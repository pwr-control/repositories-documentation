%\documentclass[11pt,a4paper]{article}
\documentclass[11pt,a4paper]{scrartcl}
%\documentclass[11pt,a4paper,oneside]{book}
\usepackage[british,UKenglish,USenglish,english,american]{babel}
%\usepackage[a4paper, total={16cm, 23cm}]{geometry}
\usepackage[tmargin = 1.25in,bmargin = 1.25in,lmargin = 0.75in,rmargin = 0.75in]{geometry}
\usepackage{tikz}
\usepackage{graphicx}
\usepackage{pgfplots}
\pgfplotsset{width=12cm,compat=1.9}
\usepackage{setspace}
\usepackage{chemmacros}
\usepackage{chemfig}
%\usepackage{ghsystem}
\usechemmodule{redox}
%\usepackage{chemnum}
%\usepackage{bohr}
%\usepackage{elements}
%\usepackage{endiagram}
%\usepackage{modiagram}
%\usepackage{chemgreek}
%\usepackage{mhchem}
\usepackage{esint}
\usepackage{tabularray}

\usepackage{makeidx}
\usepackage{epstopdf}

\usepackage{amssymb}
\usepackage{mathrsfs}
%\usepackage{minted}
\usepackage{bm}
\usepackage{amsmath}
\usepackage{enumitem}
\usepackage[english]{varioref}
\usepackage[english]{babel}
\usepackage{lipsum}
\usepackage{fancyhdr}
\pagestyle{fancy} 
\usepackage{float}
\usepackage{empheq}
\usepackage[framemethod=tikz]{mdframed}
\usepackage{epstopdf}
\numberwithin{equation}{section}
\usepackage{eso-pic}
\usepackage{calc}
\usepackage{nccmath}
\usepackage{caption}
\usepackage{subcaption}
\usepackage{gensymb}
\usepackage{amsfonts,amsthm,epsfig,epstopdf,titling,url,array}
\usepackage{siunitx}
\sisetup{input-digits = 0123456789\pi}
\usepackage[symbol]{footmisc}
\usepackage{xcolor}
\usepackage{multicol}
\usepackage{boondox-cal}
\DeclareSIUnit\atm{atm}
\setcounter{secnumdepth}{3}
\setcounter{tocdepth}{3}
\usepackage{booktabs}
\usepackage{blindtext}
\usepackage{changepage}

% \usepackage{draftwatermark}
% \SetWatermarkText{DRAFT}
% \SetWatermarkScale{5}

\DeclareSIUnit\atm{atm}

\pagestyle{fancy} 
\fancypagestyle{firstpage}{
	\rhead{
	}
}
\fancyhead[L]{\small\slshape\nouppercase{\leftmark}}
\chead{}
\rhead{
}
\lfoot{\textit{}}
\cfoot{-\ \thepage\ -}
\rfoot{\textit{}}

\DeclareMathOperator{\rank}{rank}
\DeclareMathOperator{\atantwo}{atan2}
\DeclareMathOperator{\arctantwo}{arctan2}
\DeclareMathOperator{\spn}{span}

\renewcommand{\headrulewidth}{0.4pt}
\renewcommand{\footrulewidth}{0.4pt}
\newcommand{\abs}[1]{\left|#1\right|}
\definecolor{mycolor1}{rgb}{0.97, 0.97, 0.97}
\definecolor{mycolor2}{rgb}{0.97, 0.97, 0.97}
\definecolor{tableShade}{gray}{0.9}
\newcommand{\sign}{\text{sign}}
\newcommand{\centered}[1]{\begin{tabular}{@{}l@{}} #1 \end{tabular}}
\theoremstyle{it}
\newtheorem{defn}{Definition}[section]
\newtheorem{assumption}{Assumption}[section]
\newtheorem{thm}{Theorem}[section]
\newtheorem{lemma}{Lemma}[section]
\newtheorem{corollary}{Corollary}[section]
\theoremstyle{definition}
%\theoremstyle{it}
\newtheorem{example}{Example}[section]
\let\eqrefn\eqref
\renewcommand{\eqref}[1]{Eq.~(\ref{#1})}

\newenvironment{myitemize_1}
{ \begin{itemize}[topsep=4pt]
		\setlength{\topsep}{2pt}		
		\setlength{\itemsep}{2pt}
		\setlength{\parskip}{2pt}
		\setlength{\parsep}{2pt}     }
	{ \end{itemize}                  }


\newmdenv[innerlinewidth=0.5pt, roundcorner=4pt,backgroundcolor=mycolor2, 
linecolor=mycolor1,innerleftmargin=6pt,
innerrightmargin=6pt,innertopmargin=6pt,innerbottommargin=6pt]{mybox}

\title{\textbf{ 
		\begin{LARGE}
			Control of String-Actuated:
		\end{LARGE} \\[24pt]
		\begin{Large}
			following the work of Wang/Krstic.
		\end{Large}
	}
}
\author{\textbf{Davide Bagnara}}

\begin{document}
	\thispagestyle{empty}
	\begin{mybox}
		\maketitle
		\vspace{125mm}
	\end{mybox}
	%	\let\clearpage\relax
	\newpage
	\tableofcontents%
	\listoffigures%
	\listoftables
	%	\let\clearpage\LaTeXStandardClearpage
	\newpage

\begin{onehalfspace}
\section{Description of the problem}
The present invention relates to a string-actuated transportation system comprising a carrying string to which cabins (hereinafter referred to as gondolas) are attached. The carrying string is driven by a pulley arrangement including two pulleys, one of which is operatively connected to a motor actuator, see Figures~\ref{model_derivation_1}~and~\ref{string-actuated_iii}.

\begin{figure}[H]
	\centering
	\includegraphics[width = 425pt, keepaspectratio]{figures/plant_description/line_sketch_1.eps}
	\captionsetup{width=0.75\textwidth, font=small}
	\caption{Draw of the transportation line with description of the electrical equipment: motor and electrical drive.}
	\label{line_sketch_1}
\end{figure}
The system composed by the permanent magnet synchronous motor (\textbf{psm-moto}r) and the electrical drive (\textbf{LeitDrive}) is hereinafter referred as \textbf{Direct Drive} system.

In some installations the string-actuated line is subjected to oscillations. These oscillation are intrinsically present on all string-actuated system due to its nature, they can be \textbf{dominant} or negligible. Typical dominant oscillations are shown in Figure~\ref{small_oscillation_constraints}. 
\begin{figure}[H]
	\centering
	\includegraphics[width = 500pt, keepaspectratio]{figures/plant_description/small_oscillation_constraints.eps}
	\captionsetup{width=0.75\textwidth, font=small}
	\caption{Possible dominant string oscillations of the transmission line.}
	\label{small_oscillation_constraints}
\end{figure}
Moreover, in some installation the presence of particular disturbances with pulsating nature, and where this pulsation is close to the pulsation of some dominant oscillations could degrade the intrinsically stability of the system.

\section{Approach to the problem}
\textbf{Problem description:}
\begin{myitemize_1}
	\item[--] the line of a given string-actuated starts oscillating, that means the string moves vertically ($u$-direction) with dominant pulsations;
	\item[--] the string-actuated is actuated through a motor; the motor applies a torque to the pulley, and the pulley drives the string without slipping, hence the pulley exerts a force $f$ on the string as result of the torque of the motor;
	\item[--] in order to find a control strategy which minimize the vertical oscillations of the string \textbf{a correlation} between linear displacement oscillations (vertical oscillations of the line) and the corresponding torsional displacement oscillations (torsional components acting on the shaft of the motor/pulley) \textbf{shall be derived};
	\item[--] as first approach the linear displacement oscillations will be analyzed using "small oscillation of a continuous string"; a PDE equation which can describe the vertical oscillations of the string can be assumed of the form
	\begin{equation}
		\frac{\partial^2 u(x,t)}{\partial t^2} = c^2\frac{\partial^2 u(x,t)}{\partial x^2}\quad n = 1,2,\dots;
	\end{equation}
	which results
	\item[--] in order to find solution of the above equation, separation of variables method is adopted, then the stationary spatial solutions of the PDEs assume the following form 
	\begin{equation}
		u_n(x,t) = \phi_n(x)\cos(\omega_n t + \varphi_n);
	\end{equation}
	where
	\begin{equation}
		u(x,t) = \sum_{n=1}^{\infty}\phi_n(x)\cos(\omega_n t + \varphi_n);
	\end{equation}
	\item[--] in order to create a model useful for a control system design a reduce order of normal mode components will be taken into account, as follows
	\begin{equation}
		u(x,t) \approx \sum_{n=1}^{N}\phi_n(x)\cos(\omega_n t + \varphi_n);
	\end{equation}
	\item[--] once the equivalent representation of the string as \textbf{linear string displacement model} has been defined an equivalent interpretation as \textbf{torsional string displacement model} is investigated;
	\item[--] as for the linear string model, a reduce order of normal components will be taken into account for the torsional string model, as follows
	\begin{equation}
		\vartheta(x,t) \approx \sum_{n=1}^{N}q_n(x)\cos(\omega_n t + \varphi_n);
	\end{equation} 
	\textbf{	the torsional model and the linear model will share the same dominant harmonics components};
	\item[--] \textbf{the design of a control law needs the definition of a reference model where the control can be built on}; in this context we assume that exist an \textit{isomorphism} between the equivalent \textit{linear string} representation and the \textit{rotational string} representation; the isomorphism is created by the presence of the holonomic constraints created by the pulleys mechanism:
	\begin{mybox}
		\begin{equation}
			\sum_{n=1}^{N}\phi_n(x)\cos(\omega_n t + \varphi_n) \xleftrightarrow{\text{\textbf{isomorphism}}} \sum_{n=1}^{N}q_n(x)\cos(\omega_n t + \varphi_n);
		\end{equation} 
	\end{mybox}
\end{myitemize_1}
\noindent\textbf{Remark} - the equivalent model composed by a finite number of torsional harmonics will modelized as torsional $N$-mass-$N$-spring-mass chain. \\
	
\noindent\textbf{Remark} - another aspect of the proposed invention concerns the safety of the string-actuated. The presence of intrinsically resonance (as stationary oscillation of the line) as well as the presence of deterministic monochromatic excitation\footnote{The term \textit{deterministic monochromatic excitation} refers to the presence of specific harmonic.} disturbance could bring the overall line to unbounded oscillations.    



\section{Patent description}
This section outlines the foundational principles underlying the \textbf{Advanced Speed Control} strategy. Preliminary assumptions and key concepts are below introduced, with further technical details elaborated in the subsequent sections. \\

\noindent\textbf{Statement} - The string-actuated transmission line is modeled as a oscillating string (in terms of vertical displacements) with fixed-end constraints. \\

\noindent\textbf{Statement} - Based on string theory, string-actuated operational conditions, and experimental measurements, a set of critical frequencies has been identified. From experimental results it is possible to state that the identified critical frequencies correspond to the string oscillation modes (or stationary solution of the fixed string).\\

\noindent\textbf{Statement} - The control problem for attenuating string-actuated line oscillations involves designing a controller capable of damping the vertical oscillations, particularly when they occur at specific critical frequencies. \\

\noindent\textbf{Statement} - The concept behind the control architecture for damping string-actuated line oscillations is based on the \textbf{isomorphism} between oscillating string theory and a chain of harmonic oscillators, where the oscillators are modeled as rotational masses coupled with torsional springs. \\
\begin{figure}[H]
	\centering
	\includegraphics[width = 325pt, keepaspectratio]{figures/plant_description/conceptual_idea_figure_1.eps}
	\captionsetup{width=0.75\textwidth, font=small}
	\caption{Conceptual framework and isomorphism between constrained linear string dynamics and chain of coupled oscillators, where the chain of coupled oscillators represent the foundation for the development of the control architecture.}
	\label{conceptual_idea_figure_1}
\end{figure}
The conceptual layout of the control system here developed is shown in Figure~\ref{conceptual_idea_figure_1}. In accordance with string theory, the oscillations of the string-actuated line can be approximated as a finite set of harmonic modes corresponding to a fundamental frequency (or stationary solution of the fixed ends oscillating string). These harmonic frequencies are subsequently mapped onto a chain of harmonic oscillators, within which the control strategy is developed. Each stationary oscillations of the equivalent sting, which mimic the string-actuated line, is composed by a spatial-domain terms and by a time-domain terms which are decoupled. The time-domain terms can easily mapped into a pure time-domain equivalent system based on harmonic oscillators. \\

\noindent\textbf{Remark} - the creation of a virtual model which can be described by a set of ordinary differential equations (ODEs) play a fundamental role in the overall modelization. Tools for control system design are based on ODEs. As show in Figure~\ref{isomorphism-1} as well as Figure~\ref{isomorphism-2} the initial system, approximated to a fixed ends oscillating string is basically described by PDEs, and then is mapped (via the isomorphism concept) into a set of coupled rotational oscillators described by ODEs. 
\begin{figure}[H]
	\centering
	\includegraphics[width = 375pt, keepaspectratio]{figures/plant_description/isomorphism-1.eps}
	\captionsetup{width=0.75\textwidth, font=small}
	\caption{The dominant string oscillations are mapped onto a chain of harmonic oscillators to facilitate control design; here the case of one dominant harmonic.}
	\label{isomorphism-1}
\end{figure}
\begin{figure}[H]
	\centering
	\includegraphics[width = 375pt  , keepaspectratio]{figures/plant_description/isomorphism-2.eps}
	\captionsetup{width=0.75\textwidth, font=small}
	\caption{The dominant string oscillations are mapped onto a chain of harmonic oscillators to facilitate control design; here the case of tow dominant harmonics.}
	\label{isomorphism-2}
\end{figure}

\noindent\textbf{Statement} - The control architecture is entirely built around the model based on a chain of harmonic oscillators (rotating masses) and is \textbf{primarily designed using pole placement and a state observer} techniques. \\

The internal control layout of the proposed invention is shown in Figure~\ref{conceptual_idea_figure_2}; speed regulation is performed via a proportional-integral (PI) controller, while harmonic oscillations are suppressed through state feedback control designed by pole placement. The state vector used by the state feedback is estimated by a state observer. \\

\noindent\textbf{Remark} - The overall control architecture adopts a model-based approach, whereby the system's performance is inherently dependent on the accuracy of the mathematical model relative to the actual string-actuated system. \\

\begin{figure}[H]
	\centering
	\includegraphics[width = 500pt, keepaspectratio]{figures/plant_description/conceptual_idea_figure_2.eps}
	\captionsetup{width=0.75\textwidth, font=small}
	\caption{Conceptual control layout: speed control based on PI, pole placement and state observers.}
	\label{conceptual_idea_figure_2}
\end{figure}

%\begin{figure}[H]
%	\centering
%	\includegraphics[width = 525pt, keepaspectratio]{figures/plant_description/small_oscillations_figure_1.eps}
%	\captionsetup{width=0.75\textwidth, font=small}
%	\caption{The string oscillation modes are described, highlighting that the fundamental frequency is a function of the string tension and the linear mass density.}
%	\label{small_oscillations_figure_1}
%\end{figure}
%
\pagebreak
\section{Model design}
In this section as well as in the following a technical description of the proposed invention is depicted. As already mentioned the design of a control system which will be able to actively damping the vertical oscillation of the transportation line must be designed upon an \textit{as good as possible} approximated model representative of the line dynamics.

The modelization will be outlined as follows:
\begin{myitemize_1}
	\item[--] analysis of the linear string displacement model;
	\item[--] effects of monochromatic excitation;
	\item[--] analysis of the torsional string displacement model;
	\item[--] analysis of an equivalent torsional coupled oscillators model.
\end{myitemize_1}

\subsection{Linear string displacement model}
Define
\begin{myitemize_1}
	\item[--] $L\ \Big[\SI{}{\meter}\Big]$ : length of the line;
	\item[--] $\mu\ \Big[\SI{}{\kilogram\per\meter}\Big]$ : linear density of the string;
	\item[--] $T\ \Big[\SI{}{\newton}\Big]$ : tension of the string;
\end{myitemize_1}
The partial differential equation (PDE) which describes the dynamic of the transversal string displacement $u(x,t)$ can be represented as follows 
	\begin{equation}
	\frac{\partial^2 u(x,t)}{\partial t^2} = c^2\frac{\partial^2 u(x,t)}{\partial x^2};
\end{equation}
where $c = \sqrt{\frac{T}{\mu}}$, $0 < x < L$ and $t > 0$ with the constraints $u(0,t) = 0$ and $u(L,t) = 0$. Assuming separation of variables a possible solution can be written as follows 
\begin{equation}
	u(x,t) = \sum_{n=1}^{\infty}q_n(t)\phi_n(x)
\end{equation}

which lead to a representation of the system through a set of ordinary differential equations (ODEs) expressed as function of time and space, as follows
\begin{mybox}
\noindent\textbf{Space components representation:}
\begin{equation}
	\begin{aligned}
		\phi_n^{''}(x) + k_n^2 \phi_n(x) = 0
	\end{aligned}
\end{equation}
where $k_n = {n\pi}/{L}$.
\end{mybox}

\begin{mybox}
\noindent\textbf{Time components representation:}
\begin{equation}
	\begin{aligned}
		\ddot{q}_n(t) + \omega_n^2 q_n(t) = 0
	\end{aligned}
\end{equation}
where $\omega_n = ck_n = {n\pi c}/{L}$.	
\end{mybox}

Stationary solutions can be expressed as follows
\begin{equation}
	\begin{aligned}
		\phi_n(x) = \sin\Big(\frac{n\pi}{L}\Big),\quad n=1,2,\dots
	\end{aligned}
\end{equation}
\begin{equation}
	\begin{aligned}
		q_n(t) = a_n\cos\big(\omega_nt\big) + b_n\sin\big(\omega_nt\big),\quad n=1,2,\dots
	\end{aligned}
\end{equation}
where the general solution for $u(x,0) = f(x)$ and $u_t(x,0) = g(x)$ is
\begin{equation}
	u(x,t) = \sum_{n=1}^{\infty} \sin\Big(\frac{n\pi x}{L}\Big) \Big[a_n\cos\big(\omega_nt\big) + b_n\sin\big(\omega_nt\big)\Big]
\end{equation}
with 
\begin{equation}
	\begin{aligned}
		a_n &= \frac{2}{L}\int_{0}^{L}f(x)\sin\Big(\frac{n\pi x}{L}\Big) dx \\[6pt]
		b_n &= \frac{2}{\omega_n L}\int_{0}^{L}g(x)\sin\Big(\frac{n\pi x}{L}\Big) dx
	\end{aligned}
\end{equation}

\begin{mybox}
\textbf{The model can be approximated to a finite number of normal modes, as follows}
\begin{equation}
	u(x,t) \approx \sum_{n=1}^{N} \alpha_n\sin\Big(\frac{n\pi x}{L}\Big) \cos\big(\omega_n t - \varphi_n\big)
\end{equation}
\end{mybox}

\subsection{Linear string displacement model with external input}
In this subsection the effect of monochromatic input excitation to the line is investigated. Secondary effect of the process of attaching and detaching the gondolas to the string is the generation of a monochromatic trigger, see also Figure~\ref{trigger_1}
\begin{figure}[H]
	\centering
	\includegraphics[width = 225pt, keepaspectratio]{figures/plant_description/trigger_1.eps}
	\captionsetup{width=0.75\textwidth, font=small}
	\caption{Description of the nature of the external excitation.}
	\label{trigger_1}
\end{figure}

\noindent\textbf{Remark} - the process of attaching and detaching the gondolas to the string generates a source of disturbance which is assumed of the form:
\begin{equation}
	d_f(x,t) = F\delta(x-x_0)\cos\big(\omega_0 t\big)
\end{equation}
where $F\Big[\SI{}{\newton}\Big]$ is a transversal ($u$-direction) force. Assuming 
\begin{equation}
	\delta(x-x_0) = \frac{2}{L}\sum_{n=1}^{\infty}\sin\Big(\frac{n\pi x_0}{L}\Big)\sin\Big(\frac{n\pi x}{L}\Big).
\end{equation}
\begin{mybox}
	\textbf{the time dependent components satisfy the following ODE}
	\begin{equation}
		\ddot{q}_n(t)+\omega_n^2q_n(t) = \frac{2F}{\mu L}\sin\Big(\frac{n\pi x_0}{L}\Big)\cos\big(\omega_0 t\big)
	\end{equation}
\end{mybox}
then the general solution can be assumed of the form
\begin{equation}
	u(x,t) = \sum_{n=1}^{\infty}q_n(t)\sin\Big(\frac{n\pi x}{L}\Big)
\end{equation}
where
\begin{equation}
	q_n(t) = a_n\cos\big(\omega_n t\big) + b_n\sin\big(\omega_n t\big) + \frac{2F}{\mu L}\sin\Big(\frac{n\pi x_0}{L}\Big)\frac{\cos(\omega_0 t)}{\omega_n^2 - \omega_0^2} 
\end{equation}
\begin{mybox}
\textbf{approximated stationary solution} with forcing input can be expressed as follows
	\begin{equation}
		u(x,t) \approx \sum_{n=1}^{N}  \Bigg[\alpha_n\cos\big(\omega_n t - \varphi_n\big) + \frac{2F}{\mu L}\sin\Big(\frac{n\pi x_0}{L}\Big)\frac{\cos(\omega_0 t)}{\omega_n^2 - \omega_0^2}\Bigg] \sin\Big(\frac{n\pi x}{L}\Big).
	\end{equation}
\end{mybox}

\noindent\textbf{Remark} - when $\omega_0 = \omega_n$ the damped\footnote{It is assumed that the real system is intrinsically damped.} system results in a steady state oscillation, while the undamped system results in a unbounded growth oscillations. 
\begin{figure}[H]
	\centering
	\includegraphics[width = 450pt, keepaspectratio]{figures/plant_description/conceptual_idea_figure_5.eps}
	\captionsetup{width=0.75\textwidth, font=small}
	\caption{The dominant string oscillations are mapped onto a chain of harmonic oscillators to facilitate control design.}
	\label{conceptual_idea_figure_3}
\end{figure}

\subsection{Torsional string displacement model}
From physical point of view we can assumed that the linear displacement oscillations are transmitted though the holonomic constraints of the pulleys to the shaft of the motor (actuator) inducing torsional displacement oscillations which will be visible at the motor quantities in terms of torque components and phase displacements.

From physical point of view we can assume, a kind of isomorphism between the linear displacement oscillation of the line and the torsional displacement oscillations of a virtual torsional-spring-inertia chain.

From mathematical point of view the torsional-spring-inertia chain can be modelized as follows: define $\vartheta(x,t)$ angular torsion as function of space and time of an equivalent torsional bar, the equations of the motion of $\vartheta(x,t)$ are
	\begin{equation}
	\frac{\partial^2 \vartheta(x,t)}{\partial t^2} = c^2\frac{\partial^2 \vartheta(x,t)}{\partial x^2};
\end{equation}
where $\vartheta(0,t) = \vartheta(L,t) = 0,\quad\forall t$.

A possible solution can be formulated in the form 
\begin{equation}
	\vartheta(x,t) = \sum_{n=1}^{\infty}q_n(t)\phi_n(n) 
\end{equation}
obtaining the ODEs\footnote{When the PDEs admit separation of variables, the solutions of the PDEs are obtained through the combination of solutions of the corresponding set of ODEs.} equations
\begin{mybox}
	\noindent\textbf{Phase components representation:}
	\begin{equation}
		\begin{aligned}
			\phi_n^{''}(x) + k_n^2 \phi_n(x) = 0
		\end{aligned}
	\end{equation}
	where $k_n = {n\pi}/{L}$.
\end{mybox}
\begin{mybox}
	\noindent\textbf{Time components representation:}
	\begin{equation}
		\begin{aligned}
			\ddot{q}_n(t) + \omega_n^2 q_n(t) = 0
		\end{aligned}
	\end{equation}
	where $\omega_n = ck_n = {n\pi c}/{L}$.	
\end{mybox}
where $\omega = ck$, $q(0) = q(L) = 0$, results
\begin{equation}
	q_n(x) = \sin\Big(\frac{n\pi x}{L}\Big), \quad n=1,2,3,\dots 
\end{equation}

\begin{equation}
	\vartheta_n(x,t) = a_n q_n(x)\cos(\omega_n t + \varphi_n)
\end{equation}

\begin{equation}
	\vartheta(x,t) \approx \sum_{n=1}^{N} a_n q_n(x)\cos(\omega_n t + \varphi_n)
\end{equation}
\begin{mybox}
	\noindent\textbf{Remark} - for the purpose of  the design of a control system the spatial terms can be omitted:
	\begin{equation}\label{final_modes}
		\left\lbrace 
		\begin{aligned}
			\vartheta_1(t) &= a_1 \cos(\omega_1 t + \varphi_1) \\[6pt]
			\vartheta_2(t) &= a_2 \cos(\omega_2 t + \varphi_2) \\[6pt]
			& \vdots \\[6pt]
			\vartheta_N(t) &= a_N \cos(\omega_N t + \varphi_N)
		\end{aligned}
		\right. 
	\end{equation}
The set of components shown in \eqref{final_modes} are the dominant harmonics which the controller shall be able to attenuate. 
\end{mybox}


\subsection{Representation by a finite number of coupled torsional oscillators}
These finite number of dominant oscillators expressed by  \eqref{final_modes} can be mapped into a chain of coupled torsional oscillators, as shown in Figure~\ref{conceptual_idea_figure_6a} and Figure~\ref{conceptual_idea_figure_6b}.

In this last designing step, the system modelization process tries to approximate the continuous torsional string into, \textit{as good as possible}, torsional mass-spring-mass chain. \\

\noindent\textbf{Remark} - the model represented by the chain of torsional springs and masses is the base for the design of the active damping control. \\

\noindent The model of the torsional $N$-mass-$N$-spring-mass chain can be modelized as follows. \\

Define:
\begin{myitemize_1}
	\item[--] $N$ torsional spring with elastic coefficient $k_\vartheta$; 
	\item[--] $\big(N+1\big)$ inertial masses: $J_i\quad i=1,2,\dots,N+1$; 
	\item[--] $\big(N+1\big)$ phase displacements: $\vartheta_i\quad i=1,2,\dots,N+1$; 
	\item[--] $\big(N+1\big)$ rotational speeds: $\omega_i\quad i=1,2,\dots,N+1$; 
	\item[--] $N$ torsional torques: $\tau_{i,i+1}\quad i=1,2,\dots,N$; 
\end{myitemize_1}
the system dynamic results in the following set of ODEs
\begin{equation}
\left\lbrace \begin{aligned}
		& J_1\ddot{\vartheta}_1 + k_\vartheta\big(\vartheta_1-\vartheta_2\big) = 0 \\[6pt]
		& J_i \ddot{\vartheta}_i + k_\vartheta\big(2\vartheta_i-\vartheta_{i+1}-\vartheta_{i-1}\big) = 0\quad i=2,\dots,N\\[6pt]
		& J_{N+1} \ddot{\vartheta}_{N+1} + k_\vartheta\big(2\vartheta_{N+1} - \vartheta_{N}\big) = 0 \\[6pt]
	\end{aligned} \right. 
\end{equation}
the above system describes the evolution of the state vector
\begin{equation}
	\vec{z} = \begin{bmatrix}
		\vartheta_1 & \dots & \vartheta_N & \omega_1 & \dots & \omega_N
	\end{bmatrix} ^T.
\end{equation}

For control system design purpose the following state vector shall be accounted
 \begin{equation}
 	\vec{x} = \begin{bmatrix}
 		\omega_1 & \dots & \omega_N & \tau_{12} & \dots & \tau_{N,N+1}
 	\end{bmatrix} ^T
 \end{equation}
where $\vec{x} = \mathbf{T}\vec{z}$, where $\mathbf{T}$ is a linear transformation.

The dynamic representation of the torsional $N$-mass-$N$-spring-mass respect to the state vector $\vec{x}$ can be expressed as follows
\begin{equation}\label{msm-model}
	\left\lbrace \begin{aligned}
		& J_1\dot{\omega}_1 = -\tau_{12} \\[6pt]
		& J_i \dot{\omega }_i = -\tau_{i,i+1} + \tau_{i-1,i}\quad i=2,\dots,N \\[6pt]
		& J_{N+1} \dot{\omega}_{N+1} = \tau_{N,N+1} \\[6pt]
		& \dot{\tau}_{i,i+1} = k_\vartheta \big(\omega_i - \omega_{i+1}\big) \quad i=1,\dots,N
	\end{aligned} \right. 
\end{equation}
The system expressed by \eqref{msm-model} can be represented in state space form as follows
\begin{equation}
	\left\lbrace 
	\begin{aligned}
	\dot{\vec{x}} &= \mathbf{A}\vec{x} + \mathbf{B}\tau_m + \mathbf{E}\tau_{l} \\[6pt]
	\vec{y} &= \mathbf{C}\vec{x}
	\end{aligned}
	\right. 
\end{equation}
where matrices $\mathbf{A}$, $\mathbf{B}$, $\mathbf{E}$ and $\mathbf{C}$ will be the fundamental elements for design of the control system. The terms $\tau_m$, and $\tau_l$ are. respectively, the motor torque and  he load torque (which can be seen like an external unknown disturbance).
\begin{figure}[H]
	\centering
	\includegraphics[width = 225pt, keepaspectratio]{figures/plant_description/conceptual_idea_figure_6a.eps}
	\captionsetup{width=0.75\textwidth, font=small}
	\caption{The dominant string oscillations are mapped onto a chain of harmonic oscillators to facilitate control design.}
	\label{conceptual_idea_figure_6a}
\end{figure}
\begin{figure}[H]
	\centering
	\includegraphics[width = 285pt, keepaspectratio]{figures/plant_description/conceptual_idea_figure_6b.eps}
	\captionsetup{width=0.75\textwidth, font=small}
	\caption{The dominant string oscillations are mapped onto a chain of harmonic oscillators to facilitate control design.}
	\label{conceptual_idea_figure_6b}
\end{figure}
\noindent\textbf{Remark} - in conclusion, the aim of the modelization process is to mapping the $N$-dominant harmonics affecting the transportation line into a $N$-mass-$N$-spring-mass torsional chain.  

\section{Control system design}
Once the state space representation of the dominant resonances have been defined as chain of coupled rotation springs, the design of the control system can start. In this proposal one specific control strategy is adopted. \\ 

\noindent Assuming that
\begin{myitemize_1}
	\item[--] rotational speed of the motor is measured and/or estimated by a state observer based on the model of a permanent magnet synchronous motor;
	\item[--] motor speed can be assimilated to the rotational speed of the first element of the mass-spring-mass chain, namely $\omega_m(t) = \omega_1(t)$, that means $\omega_1(t)$ is known;
	\item[--] the applied torque to the motor is estimated by the inner vector current control loops of the psm-motor control stage;
\end{myitemize_1}
the control strategy used for actively damp the the line oscillation is based on
\begin{myitemize_1}
	\item[--] a state feedback which matrix gain is designed by pole placement technique;
	\item[--] a state observer;
	\item[--] a load observer (inherently embedded into the state observer);
	\item[--] a proportional-integral control;
\end{myitemize_1}
Control architecture and layout can be seen from Figure~\ref{control_layout_1} and Figure~\ref{pi_sf_control_layout_ver_i}
\begin{figure}[H]
	\centering
	\includegraphics[width = 425pt, angle = 0, 
	keepaspectratio]{figures/msm/control_layout_1.eps}
	\captionsetup{width=0.65\textwidth, font=small}
	\caption{Control system layout.}
	\label{control_layout_1}
\end{figure}
Figure~\ref{pi_sf_control_layout_ver_i} shows the interconnection of the three control system blocks; as can be seen, fundamental step in the design of the control system lays in the identification of the matrices $\big(\mathbf{A}, \mathbf{B}, \mathbf{C}, \mathbf{E}\big)$, more details on the control system design are presented below. 
\begin{figure}[H]
	\centering
	\includegraphics[width = 510pt, angle = 0, 
	keepaspectratio]{figures/msm/pi_sf_control_layout_ver_i.eps}
	\captionsetup{width=0.65\textwidth, font=small}
	\caption{Control system layout from a more implementative point of view.}
	\label{pi_sf_control_layout_ver_i}
\end{figure}

\noindent\textbf{PI-Control} - proportional-integral control is applied to the error of motor speed tracking as follows
\begin{equation}\left\lbrace 
	\begin{aligned}
		\dot{\tau}_m^{i} &= k_i \big(\omega_1^{ref}-\omega_1\big) \\[6pt]
		{\tau}_m &= k_p \big(\omega_1^{ref}-\omega_1\big) + {\tau}_m^{i}
	\end{aligned}\right. 
\end{equation}
gains $k_p$ and $k_i$ are designed in order to have specific tracking behaviour. \\

\noindent\textbf{State Observer} - the state observer can be written as follows
\begin{equation}\left\lbrace 
	\begin{aligned}
		\dot{\hat{\vec{x}}} &= \mathbf{A}\hat{\vec{x}} + \mathbf{B}\tau_m^{ctrl} + \mathbf{L}\big(\omega_1 - \hat{\omega}_1\big) \\[6pt]
		\hat{\vec{y}} &= \mathbf{C}\hat{\vec{x}}
	\end{aligned}\right. 
\end{equation}
where state observer gain matrix $\mathbf{L}$ is designed in order the matrix $\Big(\mathbf{A} - \mathbf{L}\mathbf{C}\Big)$ has eigenvalues as per design. \\

\noindent\textbf{State Feedback} - the aim of the pole placement technique is to modify the \textit{nature} of the open loop system $(\mathbf{A},\mathbf{B})$, applying the state feedback the original system $(\mathbf{A},\mathbf{B})$ turns into the new system~$\big(\mathbf{A}-\mathbf{B}\mathbf{K}\big)$
\begin{equation}
	\begin{aligned}
		{\dot{\vec{x}}} &= \Big(\mathbf{A} - \mathbf{B}\mathbf{K}\Big)\vec{x}
	\end{aligned}
\end{equation}
hence the gain matrix $\mathbf{K}$ will be designed in order the matrix $\big(\mathbf{A} - \mathbf{B}\mathbf{K}\big)$
has eigenvalues as per design.  \\

\noindent\textbf{Remark} - the above three steps describe the whole control system architecture:
\begin{myitemize_1}
	\item[--] state feedback design;
	\item[--] state observer design;
	\item[--] PI-control design;
\end{myitemize_1}

\noindent\textbf{Remark} - \textbf{state observer} as well as \textbf{state feedback} are designed according to matrices $\mathbf{A}$, $\mathbf{B}$, $\mathbf{C}$ and $\mathbf{E}$ of the equivalent mass-spring-mass adopted as model. Output of the control design are the matrices $\mathbf{K}$, and $\mathbf{L}$.
\begin{figure}[H]
	\centering
	\includegraphics[width = 375pt, angle = 0, keepaspectratio]{figures/msm/control_layout_2.eps}
	\captionsetup{width=0.65\textwidth, font=small}
	\caption{Control system design process.}
	\label{control_layout_2}
\end{figure}

\pagebreak  
\section{Study Case}
In this section two case study are proposed. In the first case scenario the line experiences one dominant resonance and the equivalent mass-spring-mass model is built with two masses and one torsional spring. In the second case scenario, the dominant frequencies are two and the system is finally modelized by three inertial masses and two rotational springs.

\noindent\textbf{Remark} - the following nomenclature shall be accounted. 
\begin{myitemize_1}
	\item[--] $J_1$ : equivalent motor side inertia - $\Big[\SI{}{\kilogram\square\meter}\Big]$;	
	\item[--] $D_1$ : equivalent motor side passivity - $\Big[\SI{}{\newton\meter\second\per\radian}\Big]$;
	\item[--] $\omega_1$ : motor side rotational speed - $\Big[\SI{}{\radian\per\second}\Big]$;
	\item[--] $J_2$ : equivalent motor side inertia - $\Big[\SI{}{\kilogram\square\meter}\Big]$;	
	\item[--] $D_2$ : equivalent motor side passivity - $\Big[\SI{}{\newton\meter\second\per\radian}\Big]$;
	\item[--] $\omega_2$ : motor side rotational speed - $\Big[\SI{}{\radian\per\second}\Big]$;
	\item[--] $J_3$ : equivalent motor side inertia - $\Big[\SI{}{\kilogram\square\meter}\Big]$;	
	\item[--] $D_3$ : equivalent motor side passivity - $\Big[\SI{}{\newton\meter\second\per\radian}\Big]$;
	\item[--] $\omega_3$ : motor side rotational speed - $\Big[\SI{}{\radian\per\second}\Big]$;
	\item[--] $k_\vartheta$ : equivalent torsional spring of the system - $\Big[\SI{}{\newton\meter\per\radian}\Big]$;
	\item[--] $\tau_\vartheta$ : equivalent torsional torque applied between motor and line - $\Big[\SI{}{\newton\meter}\Big]$;
	\item[--] $D_\vartheta$ : equivalent torsional passivity - $\Big[\SI{}{\newton\meter\second\per\radian}\Big]$;
	\item[--] $\tau_m$ : total torque applied to the motor side - $\Big[\SI{}{\newton\meter}\Big]$;
\item[--] $\tau_l$ : total torque applied to the line side - $\Big[\SI{}{\newton\meter}\Big]$;
\end{myitemize_1}

\subsection{One modes case scenario} 
The mass-spring-mass model is a valid approximation of a flexible shaft, and for the purpose of this work will be used as approximation of the string-actuated system where a main natural frequency is dominant among the others. The 
aim of the control strategy, here proposed, is to modify the natural evolution (e.g. from a step response) of both rotor speeds ($\omega_1(t)$ and $\omega_2(t)$). Moreover a zero steady-state error from a given constant reference speed ($\omega_1^{ref}(t)$) shall be achieved despite the presence of a load disturbance applied to the system, see also Figure~\ref{flexible_shaft} where $\tau_m$ 
is the torque actuated by the control and $\tau_l$ a disturbance (or load torque).

The model presents two inertia masses ($J_1, J_2$) connected by a torsional spring ($k_{\vartheta}$) where passivity terms ($D_1$, $D_2$, and $D_{\vartheta}$) are taken into account. The two inertia masses can be assumed to be connected to the reference frame by two virtual bearings.

Mathematical model can be derived using Euler-Lagrange equations as follows
\begin{equation}\label{elagrange_1}
	\frac{d}{dt}\left(\frac{\partial E_{kin}}{\partial \dot{q}_1} \right) - 
	\frac{\partial E_{kin}}{\partial q_1} + \frac{\partial E_{pot}}{\partial 
		q_1} =\underbrace{Q_{diss}^{(12)} + Q_1}_\text{not conservative force}
\end{equation}
\begin{equation}\label{elagrange_2}
	\frac{d}{dt}\left(\frac{\partial E_{kin}}{\partial \dot{q}_2} \right) - 
	\frac{\partial E_{kin}}{\partial q_2} + \frac{\partial E_{pot}}{\partial 
		q_2} = \underbrace{Q_{diss}^{(21)} + Q_2}_\text{not conservative force}
\end{equation}
Where $Q_1$, $Q_2$, $Q_{diss}^{(12)}$ and $Q_{diss}^{(21)}$ are the generalized 
forces, like magnetic force, unknown load, disturbance and  friction or in general not conservative force\footnote{A conservative force is a force which is derived by a potential and satisfies to the following conditions:
	\begin{flalign*}
		\vec{\nabla}\times\vec{f} = \vec{0},\quad 
		\oint_{\mathscr{C}}\vec{f}\cdot d\vec{r} = 0,\quad
		\vec{f} = -\vec{\nabla}\phi &&
	\end{flalign*}
}. 

Applying Eqs.~\eqref{elagrange_1} and~\eqref{elagrange_2} for the case
\begin{equation}
	\begin{aligned}
		q_1 &= \vartheta_1 = \vartheta_m \\[6pt]	
		q_2 &= \vartheta_2 = \vartheta_l
	\end{aligned}
\end{equation}
The kinematic and potential energy can be written as follows
\begin{equation}
	E_{kin} = \frac{1}{2}J_1\dot{\vartheta}_1^2 + \frac{1}{2}J_2\dot{\vartheta}_2^2
\end{equation}
\begin{equation}
	E_{pot} = \frac{1}{2}k_{\vartheta}\left(\vartheta_1 - \vartheta_2\right)^2
\end{equation}
\begin{figure}[H]
	\centering
	\includegraphics[width = 325pt, keepaspectratio] {figures/model_derivation/flexible_shaft_1mode.eps}
	\captionsetup{width=0.5\textwidth, font=small}		
	\caption{Simplified equivalent model of a string-actuated system, for the case scenario of one dominant harmonic.}
	\label{flexible_shaft}
\end{figure}
Applying Lagrange the following equations of the motion will be obtained\footnote{
	\begin{figure}[H]
		\includegraphics[width = 235pt, keepaspectratio] {figures/model_derivation/equation_references.eps}
	\end{figure}
}
\begin{flalign}
	J_1\ddot{\vartheta}_1 + k_{\vartheta}\left( \vartheta_1 - \vartheta_2 \right)  &= 
	-D_1\dot{\vartheta}_1 - D_{\vartheta}\left(\dot{\vartheta_1} - \dot{\vartheta_2}\right) + 
	\tau_m \\[6pt]
	J_2\ddot{\vartheta}_2 + k_{\vartheta} \left( \vartheta_2 - \vartheta_1 \right)  &= 
	-D_2\dot{\vartheta}_2 - D_{\vartheta}\left(\dot{\vartheta_2}-\dot{\vartheta_1}\right) - 
	\tau_l
\end{flalign}
or 
\begin{equation}
	\left\lbrace \begin{aligned}
		\dot{\vartheta}_1 &= \omega_1 \\[6pt]
		\dot{\omega}_1 &= 
		-\frac{k_{\vartheta}}{J_1}\vartheta_1 - \frac{D_1+D_{\vartheta}}{J_1}\omega_1 
		+\frac{k_{\vartheta}}{J_1}\vartheta_2+\frac{D_{\vartheta}}{J_1}\omega_2+\frac{1}{J_1}\tau_m
		\\[6pt]
		\dot{\vartheta}_2 &= \omega_2 \\[6pt]
		\dot{\omega}_2 &= 
		\frac{k_{\vartheta}}{J_2}\vartheta_1 + \frac{D_{\vartheta}}{J_2}\omega_1 - \frac{k_{\vartheta}}{J_2}\vartheta_2 - \frac{D_2 + D_{\vartheta}}{J_2}\omega_2 -\frac{1}{J_2}\tau_l
	\end{aligned}\right. 
\end{equation}
The state space representation become
\begin{equation}
	\dot{\vec{z}}(t) = {\mathbf{G}} \, \vec{z}(t) + {\mathbf{H}} \, 
	\tau_m(t) + {\mathbf{M}} \, \tau_l(t)
\end{equation}
where $\vec{z}=\left[ \begin{matrix}
	\vartheta_1 & \omega_1 & \vartheta_2 & \omega_2
\end{matrix}\right]^T$
and 
\begin{equation}
	{\mathbf{G}} = 
	\left[ \begin{matrix}
		0&1&0&0\\[6pt]
		-\frac{k_{\vartheta}}{J_1}&-\frac{D_1+D_{\vartheta}}{J_1}&\frac{k_{\vartheta}}{J_1}&\frac{D_{\vartheta}}{J_1}\\[6pt]
		0&0&0&1\\[6pt]
		\frac{k_{\vartheta}}{J_2}&+\frac{D_{\vartheta}}{J_2}&-\frac{k_{\vartheta}}{J_2}&-\frac{D_2+D_{\vartheta}}{J_2}
	\end{matrix}\right]
\end{equation}
and ${\mathbf{H}} =\left[ \begin{matrix}
	0&\frac{1}{J_1}&0&0
\end{matrix}\right]^T$, 
${\mathbf{M}} =\left[ \begin{matrix}
	0&0&0&-\frac{1}{J_2}
\end{matrix}\right]^T$.

For the purpose to design a control system a representation without the $\vartheta_m$ and $\vartheta_l$ shall be taken into account; pursing this constraint the following state vector will be used
\begin{equation}
	\vec{x} = \mathbf{T} \, \vec{z} = \begin{bmatrix} \omega_1 \\[6pt] \omega_2 \\[6pt] \tau_{\vartheta} \end{bmatrix}
\end{equation}
where\footnote{where 
	\begin{flalign}
		\tau_{\vartheta}=k_\vartheta(\vartheta_1-\vartheta_2) &&
	\end{flalign} 
	which can be written also as 
	\begin{flalign}
		\dot{\tau}_{\vartheta}=k_\vartheta(\omega_1-\omega_2) &&
	\end{flalign} 
	which describes the dynamic of the shaft torsional torque.} $\mathbf{T}$ is 
given by
\begin{equation}
	\mathbf{T} = \left[\begin{matrix}
		0 & 1 & 0 & 0 \\[6pt]
		0 & 0 & 0 & 1 \\[6pt]
		k_{\vartheta} & 0 & -k_{\vartheta} & 0
	\end{matrix}\right]
	\quad \text{and} \quad
	\mathbf{T}^{-1} = \left[\begin{matrix}
		0 & 0 & \frac{1}{2k_{\vartheta}} \\[6pt]
		1 & 0 & 0 \\[6pt]
		0 & 0 & -\frac{1}{2k_{\vartheta}} \\[6pt]
		0 & 1 & 0
	\end{matrix}\right]
\end{equation}
Basically, the linear transformation $\mathbf{T}$ consists in a model reduction. The minimum representation of a dynamical system can be reached applying some matrix analysis results, but also considering the total number of energy storage parts of the system. In the case of mass-spring-mass model, we can assume that three components can store energy: the two inertial masses $J_1$ and $J_2$ as kinetic energy and the torsional spring and potential energy. That means the minimal representation of the mass-spring-mass model is made by a differential system equations of order three.

The new state space representation becomes
\begin{flalign}
	\dot{\vec{z}} &= {\mathbf{G}} \, \vec{z}+{\mathbf{H}} \, 
	\tau_m + {\mathbf{M}} \, \tau_l \\[6pt]
	\mathbf{T}^{-1}\dot{\vec{x}} &= 
	{\mathbf{G}}\mathbf{T}^{-1} \, \vec{x} + {\mathbf{H}} \, 
	\tau_m + {\mathbf{M}} \, \tau_l \\[6pt]
	%&\vdots \\
	\dot{\vec{x}} &= 
	\mathbf{T}{\mathbf{G}}\mathbf{T}^{-1} \, \vec{x}+\mathbf{T} {\mathbf{H}}
	\ \tau_m + \mathbf{T}{\mathbf{M}} \, \tau_l
\end{flalign}
or
\begin{mybox}
	\begin{equation}\left\lbrace 
		\begin{aligned}
			\dot{\vec{x}}(t) &= {\mathbf{A}} \, \vec{x}(t)+{\mathbf{B}} \, \tau_m(t) + {\mathbf{E}}\,\tau_l(t) \\[6pt]
			y(t) &=  \mathbf{C} \,\vec{x}(t)
		\end{aligned}\right. 
	\end{equation}
	or
	\begin{equation}\label{two_mass_2}
		\left[ \begin{matrix}
			\dot{\omega}_1 \\[6pt]
			\dot{\omega}_2 \\[6pt]
			\dot{\tau_{\vartheta}}
		\end{matrix}\right] = 
		\left[ \begin{matrix}
			-\frac{D_1 + D_{\vartheta}}{J_1} & \frac{D_{\vartheta}}{J_1} & 
			-\frac{1}{J_1}\\[6pt]
			\frac{D_{\vartheta}}{J_2} & -\frac{D_1 + D_{\vartheta}}{J_2} & 
			\frac{1}{J_2}\\[6pt]
			k_{\vartheta} & -k_{\vartheta} & 0
		\end{matrix}\right]
		\left[ \begin{matrix}
			{\omega_1} \\[6pt]
			{\omega_2} \\[6pt]
			{\tau_{\vartheta}}
		\end{matrix}\right] + 
		\left[ \begin{matrix}
			\frac{1}{J_1} \\[6pt]
			0 \\[6pt]
			0
		\end{matrix}\right] \tau_m+
		\left[ \begin{matrix}
			0 \\[6pt]
			-\frac{1}{J_2} \\[6pt]
			0
		\end{matrix}\right] \tau_l
	\end{equation}
	where 
	
	\begin{equation*}
		\vec{x} = \left[\begin{matrix} \omega_1 & \omega_2 & \tau_{\vartheta} 
		\end{matrix} \right]^T,
	\end{equation*}
	\begin{equation*}
		{\mathbf{A}} = \mathbf{T}{\mathbf{G}}\mathbf{T}^{-1} =
		\left[ \begin{matrix}
			-\frac{D_1 + D_{\vartheta}}{J_1} & \frac{D_{\vartheta}}{J_1} & 
			-\frac{1}{J_1}\\[6pt]
			\frac{D_{\vartheta}}{J_2} & -\frac{D_2 + D_{\vartheta}}{J_2} & 
			\frac{1}{J_2}\\[6pt]
			k_{\vartheta} & -k_{\vartheta} & 0
		\end{matrix}\right], \quad
		{\mathbf{B}} = \mathbf{T}{\mathbf{H}} =
		\left[ \begin{matrix}
			\frac{1}{J_1} \\[6pt]
			0 \\[6pt]
			0
		\end{matrix}\right],
		\quad
		{\mathbf{E}} = \mathbf{T}{\mathbf{M}} =
		\left[ \begin{matrix}
			0 \\[6pt]
			-\frac{1}{J_2} \\[6pt]
			0
		\end{matrix}\right].
	\end{equation*}
	\textbf{Consider the term $\tau_l(t)$ as a disturbance (or load) which perturbs 
		the system.}
	
	\vspace{5mm}
	For a more physical viewpoint the mass-spring-mass (or flexible shaft) dynamic 
	equations can also be written as follows:
	\begin{equation}
		\left\lbrace 
		\begin{aligned}
			& J_1 \dot{\omega}_1 + D_1\omega_1 + D_{\vartheta}(\omega_1 - \omega_2) = \tau_m 
			- \tau_{\vartheta} \\[6pt]
			& J_2 \dot{\omega}_2 + D_2\omega_2 + D_{\vartheta}(\omega_2 - \omega_1) = 
			\tau_{\vartheta} - \tau_l \\[6pt]
			& \dot{\tau}_{\vartheta} = k_{\vartheta}\left( \omega_1 - \omega_2\right) 
		\end{aligned}
		\right. 
	\end{equation}
\end{mybox}

Before to start with the development of the control, a short description of the pstringrties of the system shall be considered. Taking a look into the transfer function or into the characteristic polynomial of the system we can see it contains an harmonic oscillator. This is due to the presence of the torsional spring and the two inertia masses. The natural frequency of the harmonic oscillator can be analytically determined as follows

\noindent setting $b=0$ and $b_\vartheta=0$ we obtain the following characteristic polynomial
\begin{equation}
	\Big|s\mathbf{I}-{\mathbf{A}}\Big|=\begin{vmatrix}
		s & 0 & 1/J_1 \\[6pt]
		0 & s & -1/J_2 \\[6pt]
		-k_\vartheta & k_\vartheta & s 
	\end{vmatrix} = s\Big[s^2+k_\vartheta\frac{J_1 + J_2}{J_1 J_2}\Big]
\end{equation}
where the characteristic roots are
\begin{equation}
	\left\lbrace \begin{aligned}
		s_1&=0 \\[6pt]
		s_2&=j\sqrt{k_\vartheta\frac{J_1 + J_2}{J_1 J_2}} \\[6pt]
		s_3&=-j\sqrt{k_\vartheta\frac{J_1 + J_2}{J_1 J_2}}
	\end{aligned}\right. 
\end{equation}
the roots $s_{2,3}$ are the roots of the harmonic oscillator.

Figure~\ref{plant_msm_pulse_responce_figure} shows the effect of the inner harmonic oscillator when a impulse torque is applied to the system. The natural behaviour of the system is to oscillate until the internal friction dissipate the initial potential energy stored into the spring. 

\begin{figure}[H]
	\centering
	\begin{subfigure}{.5\textwidth}
		\centering
		\includegraphics[width = 250pt, keepaspectratio]{figures/msm/pulse_test_plant_evolution.eps}
		\captionsetup{width=0.75\textwidth, font=footnotesize}
		\caption{Pulse test configuration system - open loop system.}
		\label{}
	\end{subfigure}%
	\begin{subfigure}{.5\textwidth}
		\centering
		\includegraphics[width = 250pt, keepaspectratio]{figures/msm/plant_msm_pulse_responce_figure.eps}
		\captionsetup{width=0.75\textwidth, font=footnotesize}
		\caption{Pulse test response evolution.}
		\label{plant_msm_pulse_responce_figure}
	\end{subfigure}
	\captionsetup{width=0.75\textwidth, font=small}
	\caption{Pulse test analysis for the case of one dominant harmonic and without state feedback.}
	\label{}
\end{figure}

\begin{figure}[H]
	\centering
	\begin{subfigure}{.5\textwidth}
		\centering
		\includegraphics[width = 250pt, keepaspectratio]{figures/msm/pulse_test_plant_evolution_sf.eps}
		\captionsetup{width=0.75\textwidth, font=footnotesize}
		\caption{Pulse test configuration system - state feedback implemented.}
		\label{}
	\end{subfigure}%
	\begin{subfigure}{.5\textwidth}
		\centering
		\includegraphics[width = 250pt, keepaspectratio]{figures/msm/plant_msm_pulse_responce_sf_figure.eps}
		\captionsetup{width=0.75\textwidth, font=footnotesize}
		\caption{Pulse test response evolution.}
		\label{plant_msm_pulse_responce_sf_figure}
	\end{subfigure}
	\captionsetup{width=0.75\textwidth, font=small}
	\caption{Pulse test analysis for the case of one dominant harmonic and with state feedback.}
	\label{}
\end{figure}


\subsection{Two modes case scenario} 
In this subsection the case scenario of two dominant harmonics is taken into account; mathematical model can be derived using Euler-Lagrange equations as follows
\begin{equation}\label{elagrange_2modes_1}
	\frac{d}{dt}\left(\frac{\partial E_{kin}}{\partial \dot{q}_1} \right) - 
	\frac{\partial E_{kin}}{\partial q_1} + \frac{\partial E_{pot}}{\partial 
		q_1} =Q_{diss}^{(12)} + Q_1
\end{equation}
\begin{equation}\label{elagrange_2modes_2}
	\frac{d}{dt}\left(\frac{\partial E_{kin}}{\partial \dot{q}_2} \right) - 
	\frac{\partial E_{kin}}{\partial q_2} + \frac{\partial E_{pot}}{\partial 
		q_2} = Q_{diss}^{(23)} + Q_2
\end{equation}
\begin{equation}\label{elagrange_2modes_3}
	\frac{d}{dt}\left(\frac{\partial E_{kin}}{\partial \dot{q}_3} \right) - 
	\frac{\partial E_{kin}}{\partial q_3} + \frac{\partial E_{pot}}{\partial 
		q_3} = Q_{diss}^{(32)} + Q_3
\end{equation}
Where $Q_1$, $Q_2$, $Q_3$, $Q_{diss}^{(12)}$, $Q_{diss}^{(23)}$ and $Q_{diss}^{(31)}$ are the generalized 
forces, like magnetic force, unknown load, disturbance and  friction or in general not conservative force. 

Applying \eqref{elagrange_2modes_1}, \eqref{elagrange_2modes_2} and~\eqref{elagrange_2modes_3} for the case
\begin{equation}
	\begin{aligned}
		q_1& = \vartheta_1 = \vartheta_m \\[6pt]			
		q_2& = \vartheta_2 \\[6pt]	
		q_3& = \vartheta_3 = \vartheta_l
	\end{aligned}
\end{equation}
The kinematic and potential energy can be written as follows
\begin{equation}
	E_{kin} = \frac{1}{2}J_1\dot{\vartheta}_1^2 + \frac{1}{2}J_2\dot{\vartheta}_2^2 + \frac{1}{2}J_3\dot{\vartheta}_3^2
\end{equation}
\begin{equation}
	E_{pot} = \frac{1}{2}k_{\vartheta}\left(\vartheta_1 - \vartheta_2\right)^2 + \frac{1}{2}k_{\vartheta}\left(\vartheta_2 - \vartheta_3\right)^2
\end{equation}
\begin{figure}[H]
	\centering
	\includegraphics[width = 500pt, keepaspectratio] {figures/msm/flexible_shaft_2modes.eps}
	\captionsetup{width=0.5\textwidth, font=small}		
	\caption{Simplified equivalent model of a string-actuated system.}
	\label{flexible_shaft_2modes}
\end{figure}

Applying the Lagrange equations we obtain the following motion equations
\begin{flalign}
	J_1\ddot{\vartheta}_1 + k_{\vartheta}\left( \vartheta_1 - \vartheta_2 \right)  + D_1\dot{\vartheta}_1 + D_{\vartheta}\left(\dot{\vartheta_1} - \dot{\vartheta_2}\right) &= \tau_m \\[6pt]
	J_2\ddot{\vartheta}_2 + k_{\vartheta}\left( \vartheta_2 - \vartheta_1 \right) + k_{\vartheta}\left( \vartheta_2 - \vartheta_3 \right) + D_2\dot{\vartheta}_2 + D_{\vartheta}\left(\dot{\vartheta_2} - \dot{\vartheta_1}\right) + D_{\vartheta}\left(\dot{\vartheta_2} - \dot{\vartheta_3}\right) &= 0 \\[6pt]
	J_3\ddot{\vartheta}_3 + k_{\vartheta} \left( \vartheta_3 - \vartheta_2 \right) +
	D_3\dot{\vartheta}_3 + D_{\vartheta}\left(\dot{\vartheta_3} -\dot{\vartheta_2}\right) &= -\tau_l 
\end{flalign}
or 
\begin{equation}
	\left\lbrace \begin{aligned}
		\dot{\vartheta}_1 &= \omega_1 \\[6pt]
		\dot{\omega}_1 &= 
		-\frac{k_{\vartheta}}{J_1}\vartheta_1 + \frac{k_{\vartheta}}{J_1}\vartheta_2 
		- \frac{D_1 + D_{\vartheta}}{J_1}\omega_1 +\frac{D_{\vartheta}}{J_1}\omega_2 + \frac{1}{J_1}\tau_m
		\\[6pt]
		\dot{\vartheta}_2 &= \omega_2 \\[6pt]
		\dot{\omega}_2 &= 
		\frac{k_{\vartheta}}{J_2}\vartheta_1 - 2\frac{k_{\vartheta}}{J_2}\vartheta_2  + \frac{k_{\vartheta}}{J_2}\vartheta_3 + \frac{D_{\vartheta}}{J_2}\omega_1 - \frac{D_2 + D_{\vartheta}}{J_2}\omega_2 + \frac{D_{\vartheta}}{J_2}\omega_3 \\[6pt]
		\dot{\vartheta}_3 &= \omega_3 \\[6pt]
		\dot{\omega}_3 &= 
		\frac{k_{\vartheta}}{J_3}\vartheta_2 - \frac{k_{\vartheta}}{J_3}\vartheta_3 + \frac{D_{\vartheta}}{J_3}\omega_2 - \frac{D_3 + D_{\vartheta}}{J_3}\omega_3 - \frac{1}{J_3}\tau_l
	\end{aligned}\right. 
\end{equation}
The state space representation become
\begin{equation}
	\dot{\vec{z}}(t) = {\mathbf{G}} \, \vec{z}(t) + {\mathbf{H}} \, 
	\tau_m(t) + {\mathbf{M}} \, \tau_l(t)
\end{equation}
where $\vec{z}=\left[ \begin{matrix}
	\vartheta_1 & \omega_1 & \vartheta_2 & \omega_2 & \vartheta_3 & \omega_3
\end{matrix}\right]^T$
and 
\begin{equation}
	{\mathbf{G}} = 
	\left[ \begin{matrix}
		0&1&0&0&0&0\\[6pt]
		-\frac{k_{\vartheta}}{J_1} & -\frac{D_1+D_{\vartheta}}{J_1} & \frac{k_{\vartheta}}{J_1} & \frac{D_{\vartheta}}{J_1} & 0 & 0 \\[6pt]
		0&0&0&1&0&0\\[6pt]
		\frac{k_{\vartheta}}{J_2} & \frac{D_{\vartheta}}{J_2} & -\frac{2k_{\vartheta}}{J_2} & -\frac{D_2+D_{\vartheta}}{J_2} & \frac{k_{\vartheta}}{J_2} & \frac{D_{\vartheta}}{J_2} \\[6pt]
		0&0&0&0&0&1\\[6pt]
		0 & 0 & \frac{k_{\vartheta}}{J_3} & \frac{D_{\vartheta}}{J_3} & -\frac{k_{\vartheta}}{J_3} & -\frac{D_3+D_{\vartheta}}{J_3}
	\end{matrix}\right]
\end{equation}
and ${\mathbf{H}} =\left[ \begin{matrix}
	0&\frac{1}{J_m}&0&0&0&0
\end{matrix}\right]^T$, 
${\mathbf{M}} =\left[ \begin{matrix}
	0&0&0&0&0&-\frac{1}{J_l}
\end{matrix}\right]^T$.

For the purpose to design a control system a representation without the $\vartheta_m$ and $\vartheta_l$ shall be taken into account; pursing this constraint the following state vector will be used
\begin{equation}
	\vec{x} = \mathbf{T} \, \vec{z} = \begin{bmatrix} \omega_1 & \omega_2 & \omega_3 & \tau_{\vartheta}^{12} & \tau_{\vartheta}^{23} \end{bmatrix}^T
\end{equation}
where $\mathbf{T}$ is 
given by
\begin{equation}
	\mathbf{T} = \left[\begin{matrix}
		0 & 1 & 0 & 0 & 0 & 0\\[6pt]
		0 & 0 & 0 & 1 & 0 & 0 \\[6pt]
		0 & 0 & 0 & 0 & 0 & 1 \\[6pt]
		k_{\vartheta} & 0 & -k_{\vartheta} & 0 & 0 & 0 \\[6pt]
		0 & 0 & k_{\vartheta} & 0 & -k_{\vartheta} & 0
	\end{matrix}\right]
	\quad \text{and} \quad
	\mathbf{T}^{-1} = \left[\begin{matrix}
		0 & 0 & 0 & \frac{2}{3k_{\vartheta}} & \frac{1}{3k_{\vartheta}} \\[6pt]
		1 & 0 & 0 & 0 & 0 \\[6pt]
		0 & 0 & 0 & -\frac{1}{3k_{\vartheta}} & \frac{1}{3k_{\vartheta}} \\[6pt]
		0 & 1 & 0 & 0 & 0 \\[6pt]
		0 & 0 & 0 & -\frac{1}{3k_{\vartheta}} & -\frac{2}{3k_{\vartheta}} \\[6pt]
		0 & 0 & 1 & 0 & 0
	\end{matrix}\right]
\end{equation}
Basically, the linear transformation $\mathbf{T}$ consists in a model reduction. The minimum representation of a dynamical system can be reached applying some matrix analysis results, but also considering the total number of energy storage parts of the system. In the case of mass-spring-mass model, we can assume that three components can store energy: the two inertial masses $J_m$ and $J_l$ as kinetic energy and the torsional spring and potential energy. That means the minimal representation of the mass-spring-mass model is made by a differential system equations of order three.

The new state space representation becomes
\begin{flalign}
	\dot{\vec{z}} &= {\mathbf{G}} \, \vec{z}+{\mathbf{H}} \, 
	\tau_m + {\mathbf{M}} \, \tau_l \\[6pt]
	\mathbf{T}^{-1}\dot{\vec{x}} &= 
	{\mathbf{G}}\mathbf{T}^{-1} \, \vec{x} + {\mathbf{H}} \, 
	\tau_m + {\mathbf{M}} \, \tau_l \\[6pt]
	%&\vdots \\
	\dot{\vec{x}} &= 
	\mathbf{T}{\mathbf{G}}\mathbf{T}^{-1} \, \vec{x}+\mathbf{T} {\mathbf{H}}
	\ \tau_m + \mathbf{T}{\mathbf{M}} \, \tau_l
\end{flalign}
or
\begin{mybox}
	\begin{equation}\left\lbrace 
		\begin{aligned}
			\dot{\vec{x}}(t) &= {\mathbf{A}} \, \vec{x}(t)+{\mathbf{B}} \, \tau_m(t) + {\mathbf{E}}\,\tau_l(t) \\[6pt]
			y(t) &=  \mathbf{C} \,\vec{x}(t)
		\end{aligned}\right. 
	\end{equation}
	or
	\begin{equation}\label{two_mass_2}
		\left[ \begin{matrix}
			\dot{\omega}_1 \\[6pt]
			\dot{\omega}_2 \\[6pt]			
			\dot{\omega}_3 \\[6pt]
			\dot{\tau}_{\vartheta}^{12} \\[6pt]
			\dot{\tau}_{\vartheta}^{23}
		\end{matrix}\right] = 
		\begin{bmatrix}
			-\frac{D_1 + D_{\vartheta}}{J_1} & \frac{D_{\vartheta}}{J_1} & 0 & -\frac{1}{J_1} & 0 \\[6pt]
			\frac{D_{\vartheta}}{J_2} & -\frac{D_2 + D_{\vartheta}}{J_2} & \frac{D_{\vartheta}}{J_2} & \frac{1}{J_2} & -\frac{1}{J_2} \\[6pt]
			0 & \frac{D_{\vartheta}}{J_3} & -\frac{D_3 + D_{\vartheta}}{J_3} & 0 & \frac{1}{J_3} \\[6pt]
			k_{\vartheta}^{12} & -k_{\vartheta}^{12} & 0 & 0 & 0 \\[6pt]			
			0 & k_{\vartheta}^{23} & -k_{\vartheta}^{23} & 0 & 0
		\end{bmatrix}
		\left[ \begin{matrix}
			\omega_1 \\[6pt]
			\omega_2 \\[6pt]
			\omega_3 \\[6pt]
			\tau_{\vartheta}^{12} \\[6pt]
			\tau_{\vartheta}^{23}
		\end{matrix}\right] + 
		\left[ \begin{matrix}
			\frac{1}{J_1} \\[6pt]
			0 \\[6pt]			
			0 \\[6pt]
			0 \\[6pt]
			0
		\end{matrix}\right] \tau_m +
		\left[ \begin{matrix}
			0 \\[6pt]
			0 \\[6pt]
			-\frac{1}{J_3} \\[6pt]
			0 \\[6pt]
			0
		\end{matrix}\right] \tau_l
	\end{equation}
	where 
	
	\begin{equation*}
		\vec{x} = \left[\begin{matrix} \omega_1 & \omega_2 & \omega_3 & \tau_{\vartheta}^{12} & \tau_{\vartheta}^{23}
		\end{matrix} \right]^T,
	\end{equation*}
	\begin{equation*}
		\begin{aligned}
			{\mathbf{A}} &= \mathbf{T}{\mathbf{G}}\mathbf{T}^{-1} =
			\begin{bmatrix}
				-\frac{D_1 + D_{\vartheta}}{J_1} & \frac{D_{\vartheta}}{J_1} & 0 & -\frac{1}{J_1} & 0 \\[6pt]
				\frac{D_{\vartheta}}{J_2} & -\frac{D_2 + D_{\vartheta}}{J_2} & \frac{D_{\vartheta}}{J_2} & \frac{1}{J_2} & 		-\frac{1}{J_2} \\[6pt]
				0 & \frac{D_{\vartheta}}{J_3} & -\frac{D_3 + D_{\vartheta}}{J_3} & 0 & \frac{1}{J_3} \\[6pt]
				k_{\vartheta}^{12} & -k_{\vartheta}^{12} & 0 & 0 & 0 \\[6pt]			
				0 & k_{\vartheta}^{23} & -k_{\vartheta}^{23} & 0 & 0
			\end{bmatrix}, \quad
			{\mathbf{B}} = \mathbf{T}{\mathbf{H}} =
			\begin{bmatrix}
				\frac{1}{J_1} \\[6pt]
				0 \\[6pt]			
				0 \\[6pt]
				0 \\[6pt]
				0
			\end{bmatrix} \\[6pt]
			{\mathbf{E}} &= \mathbf{T}{\mathbf{M}} =
			\begin{bmatrix}
				0 \\[6pt]
				0 \\[6pt]
				-\frac{1}{J_3} \\[6pt]
				0 \\[6pt]
				0
			\end{bmatrix}.
		\end{aligned}
	\end{equation*}      
	\textbf{Consider the term $\tau_l(t)$ as a disturbance (or load) which perturbs 
		the system.}
	
	\vspace{5mm}
	For a more physical viewpoint the mass-spring-mass (or flexible shaft) dynamic 
	equations can also be written as follows:
	\begin{equation}
		\left\lbrace 
		\begin{aligned}
			& J_1 \dot{\omega}_1 + D_1\omega_1 + D_{\vartheta} (\omega_1 - \omega_2) = \tau_m - \tau_{\vartheta}^{12} \\[6pt]
			& J_2 \dot{\omega}_l + D_2\omega_2 + D_{\vartheta} (\omega_2 - \omega_1 - \omega_3) = \tau_{\vartheta}^{12} - \tau_{\vartheta}^{23} \\[6pt]
			& J_3 \dot{\omega}_3 + D_3\omega_3 + D_{\vartheta} (\omega_2 - \omega_3) = \tau_{\vartheta}^{23} - \tau_l \\[6pt]
			& \dot{\tau}_{\vartheta}^{12} = k_{\vartheta}\left( \omega_1 - \omega_2\right) \\[6pt]
			& \dot{\tau}_{\vartheta}^{23} = k_{\vartheta}\left( \omega_2 - \omega_3\right) 
		\end{aligned}
		\right. 
	\end{equation}
\end{mybox}

Before to start with the development of the control, a short description of the pstringrties of the system shall be considered. Taking a look into the transfer function or into the characteristic polynomial of the system we can see it contains an harmonic oscillator. This is due to the presence of the torsional spring and the two inertia masses. The natural frequency of the harmonic oscillator can be analytically determined as follows

\noindent setting $D_1=D_2=D_3=0$, $D_\vartheta=0$, and $J_1 = J_2 = J_3$ we obtain the following characteristic polynomial
\begin{equation}
	\Big|s\mathbf{I}-{\mathbf{A}}\Big|=\begin{vmatrix}
		s & 0 & 0 & 1/J & 0 \\[6pt]
		0 & s & 0 & -1/J & 1/J \\[6pt]
		0 & 0 & s & 0 & -1/J \\[6pt]
		-k_{\vartheta} & k_{\vartheta} & 0 & s & 0 \\[6pt]
		0 & -k_{\vartheta} & k_{\vartheta} & 0 & s-k_\vartheta 
	\end{vmatrix} = s\Big[\frac{1}{J^3}\Big(J^3s^4+4J^2k_\vartheta s^3 + 3Jk_{\vartheta}s\Big)\Big]
\end{equation}
where the characteristic roots are
\begin{equation}
\begin{aligned}
	s_1 = 0, \qquad
	s_2 = j\sqrt{\frac{k_\vartheta}{J}}, \qquad
	s_3 = -j\sqrt{\frac{k_\vartheta}{J}}, \qquad
	s_4 = j\sqrt{3}\sqrt{\frac{k_\vartheta}{J}}, \qquad
	s_5 = -j\sqrt{3}\sqrt{\frac{k_\vartheta}{J}}
\end{aligned}
\end{equation}

Figure~\ref{plant_msmsm_pulse_responce_figure} shows the effect of the inner harmonic oscillator when a impulse torque is applied to the system. The natural behaviour of the system is to oscillate until the internal friction dissipate the initial potential energy stored into the spring. 

\begin{figure}[H]
	\centering
	\begin{subfigure}{.5\textwidth}
		\centering
		\includegraphics[width = 250pt, keepaspectratio]{figures/msm/pulse_test_plant_evolution.eps}
		\captionsetup{width=0.75\textwidth, font=footnotesize}
		\caption{Pulse test configuration system - open loop system.}
		\label{pulse_test_plant_evolution}
	\end{subfigure}%
	\begin{subfigure}{.5\textwidth}
		\centering
		\includegraphics[width = 250pt, keepaspectratio]{figures/msm/plant_msmsm_pulse_responce_figure.eps}
		\captionsetup{width=0.75\textwidth, font=footnotesize}
		\caption{Pulse test response evolution.}
		\label{plant_msmsm_pulse_responce_figure}
	\end{subfigure}
	\captionsetup{width=0.75\textwidth, font=small}
	\caption{Pulse test analysis for the case of two dominant harmonics and without state feedback.}
	\label{}
\end{figure}

\begin{figure}[H]
	\centering
	\begin{subfigure}{.5\textwidth}
		\centering
		\includegraphics[width = 250pt, keepaspectratio]{figures/msm/pulse_test_plant_evolution_sf.eps}
		\captionsetup{width=0.75\textwidth, font=footnotesize}
		\caption{Pulse test configuration system - state feedback implemented.}
		\label{pulse_test_plant_evolution_sf}
	\end{subfigure}%
	\begin{subfigure}{.5\textwidth}
		\centering
		\includegraphics[width = 250pt, keepaspectratio]{figures/msm/plant_msmsm_pulse_responce_sf_figure.eps}
		\captionsetup{width=0.75\textwidth, font=footnotesize}
		\caption{Pulse test response evolution.}
		\label{plant_msmsm_pulse_responce_sf_figure}
	\end{subfigure}
	\captionsetup{width=0.75\textwidth, font=small}
	\caption{Pulse test analysis for the case of two dominant harmonics and with state feedback.}
	\label{}
\end{figure}

\clearpage
\begin{thebibliography}{99}
	\bibitem{ogata_1} 
	K. Ogata, \emph{Modern Control Engineering}. Pearson 2009.	
	
	\bibitem{ogata_2} 
	K. Ogata, \emph{Discrete-time Control Systems}. Prentice-Hall 1987.

	\bibitem{wang} 
	J. Wang, M. Krstic, \emph{PDE Control of String-Actuated Motion}. Princeton University Press, 2022.
\end{thebibliography}

\end{onehalfspace}
\end{document}