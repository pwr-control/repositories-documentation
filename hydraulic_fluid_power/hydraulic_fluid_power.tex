%\documentclass[11pt,a4paper]{article}
%\documentclass[11pt,a4paper]{scrartcl}
\documentclass[11pt,a4paper,oneside]{book}
\usepackage[table]{xcolor}
\usepackage{colortbl}
\usepackage{tabularray}
\usepackage{pgfplots}
\usepackage[british,UKenglish,USenglish,english,american]{babel}
%\usepackage[a4paper, total={16cm, 23cm}]{geometry}
\usepackage[tmargin = 1.25in,bmargin = 1.25in,lmargin = 1in,rmargin = 1in]{geometry}
\usepackage{tikz}
\usepackage{graphicx}
\usepackage[version=4]{mhchem}
\usepackage{chemformula}

\usepackage{enumitem}
\usepackage{makeidx}
\usepackage{epstopdf}
\usepackage{amssymb}
\usepackage{mathrsfs}
\usepackage{amsmath}

\usepackage[english]{varioref}
\usepackage[english]{babel}
\usepackage{lipsum}
\usepackage{fancyhdr}
\pagestyle{fancy} 
\usepackage{float}
\usepackage{empheq}
\usepackage[framemethod=tikz]{mdframed}
\usepackage{epstopdf}
\numberwithin{equation}{section}
\usepackage{eso-pic}
\usepackage{calc}
\usepackage{nccmath}
\usepackage{caption}
\usepackage{subcaption}
\usepackage{gensymb}
\usepackage{amsfonts,amsthm,epsfig,epstopdf,titling,url,array}
\usepackage{siunitx}
\usepackage{xcolor}
\usepackage{multicol}
\usepackage{boondox-cal}
\usepackage{tabularray}


\setcounter{secnumdepth}{3}
\setcounter{tocdepth}{3}
\usepackage{booktabs}

\DeclareSIUnit\atm{atm}

\pagestyle{fancy} 
\fancypagestyle{firstpage}{}
\fancyhead[L]{\slshape\nouppercase{\leftmark}}
\chead{}
\rhead{}
\lfoot{\textit{}}
\cfoot{-\ \thepage\ -}
\rfoot{\textit{}}

\DeclareMathOperator{\rank}{rank}
\DeclareMathOperator{\atantwo}{atan2}

\renewcommand{\headrulewidth}{0.4pt}
\renewcommand{\footrulewidth}{0.4pt}
\newcommand{\abs}[1]{\left|#1\right|}
\definecolor{mycolor1}{rgb}{0.95, 0.95, 0.95}
\definecolor{mycolor2}{rgb}{0.95, 0.95, 0.95}
\definecolor{tableShade}{gray}{0.9}
\newcommand{\sign}{\text{sign}}
\newcommand{\centered}[1]{\begin{tabular}{@{}l@{}} #1 \end{tabular}}
\theoremstyle{it}
\newtheorem{defn}{Definition}[section]
\newtheorem{thm}{Theorem}[section]
\newtheorem{lemma}{Lemma}[section]
\theoremstyle{definition}
%\theoremstyle{it}
\newtheorem{example}{Example}[section]

\newenvironment{myitemize_1}
{ \begin{itemize}[topsep=0pt]
		\setlength{\topsep}{2pt}		
		\setlength{\itemsep}{2pt}
		\setlength{\parskip}{2pt}
		\setlength{\parsep}{2pt}     }
	{ \end{itemize}                  }

\newmdenv[innerlinewidth=0.5pt, roundcorner=4pt,backgroundcolor=mycolor2, linecolor=mycolor1,innerleftmargin=6pt,
innerrightmargin=6pt,innertopmargin=6pt,innerbottommargin=6pt]{mybox}

\title{\textbf{Hydrostatic Fluid Power Modelization}}
\author{Davide Bagnara}

\begin{document}
	\thispagestyle{firstpage}
	\begin{mybox}
		\maketitle
		\vspace{125mm}
	\end{mybox}
	\newpage
	\tableofcontents
	\listoffigures	
	\listoftables
	\newpage
	
\chapter{Hydraulic power fluid modelization}
\section{Introduction}
The modelization consists in the development of the most significant components, for each vehicle, which affect the behaviour of the power-train and to create a full model. For each component, like pump, motor, valves, etc a mathematical representation has been derived as well as a \textit{Simscape} (a Matlab\,\textsuperscript{\tiny\textregistered}  simulation environment) model. Each Simscape model is developed in a way to be parametrizable in order to create a multi-user object. The structure of the modelization can be seen as a layered-structure.

The validation of the whole power-train model, and refining of the parameters, is carried out by the matching of the experimental data. The model fitting cover about half of the time spent around the development of the models. From the bottom we can see a fundamental model of each single component. More complex components are developed by the use of the above fundamental components and so on up to the realization of the whole physical architecture of the vehicle. As already said the context is a multi-energy domains and can be summarized as follows:
\begin{itemize}
	\item Diesel engine
	\item Hydrostatic pump
	\item Hydrostatic motor
	\item Hydraulic relief valve
	\item Hydraulic check valve
	\item Hydraulic 3/4 ways valve
	\item Hydrostatic pressure source
	\item Hydrostatic servo actuator	
	\item Hydrostatic power-train
	\item Mechanical gear
	\item Vehicle control strategy
	\item Anti-stall vehicle strategy
	\item Optimal vehicle control
	\item Electrical power-source		
	\item Electrical power-train
	\item Global control of the electrical power-train
\end{itemize} 
\chapter{Hydrostatic components}

\section{Power losses}
Hydraulic power is the product of pressure drop and flow $P_{hy}(t)=\Delta p(t)q(t)$, and in many instances this power is consumed by fluid friction and increases the internal head (energy) of the fluid. The power used by all hydraulic resistance such as orifice, valves pipes, etc, is converted into a temperature rise of the fluid. Most of the hydraulic power produced by hydrostatic motors is used as shaft work, but power used by internal regulation (valves) and cross leakages (orifices) is converter into heat. A typical working case is the abrupt expansion as shown in Figure~\ref{abrupt_expansion} where for the case $d_2\gg d_1$ the head loss become $\Delta E = \frac{1}{2}\rho u_1^2$ (where $\xi=1$) resulting in $p_1 = p_2$.
\begin{figure}[H]
	\centering
	\includegraphics[width = 0.5\textwidth, width = 275pt, angle = 0, keepaspectratio]{figures/hydrostatic_components/abrupt_expansion.eps}
	\captionsetup{width=0.75\textwidth}		
	\caption{Abrupt expansion.}
	\label{abrupt_expansion}
\end{figure}
The overall energy $\Delta E = \frac{1}{2}\rho u_1^2$ is converter into heat.

\section{Hydrostatic pump}
Hydrostatic pump is a device which converts a mechanical power into a hydraulic power. According to Figure~\ref{Hydrostatic_pump} the following dynamical model can be written:
\begin{figure}[H]
	\centering
	\includegraphics[width = 0.5\textwidth, width = 275pt, angle = 0, keepaspectratio]{figures/hydrostatic_components/hypump_scheme_1.eps}
	\captionsetup{width=0.75\textwidth}		
	\caption{Hydrostatic pump.}
	\label{Hydrostatic_pump}
\end{figure}
\begin{equation}
	\left\lbrace \begin{aligned}
		J_p\frac{d}{dt}{\omega}_p &= \tau_p-\Delta p D_p - b_p\omega_p \\[6pt]
		\frac{V}{\beta}\frac{d}{dt}{\Delta p} &= \omega_p D_p - q_1 - q_{pL} \\[6pt]
		q_{pL} &= \Big(1-\eta_{p}^{vol}\Big) \omega_p D_p \\[6pt]
		q_1 &= q_4 
	\end{aligned}\right. 
\end{equation}

\section{Hydrostatic motor}
As for the pump the hydrostatic motor is a device which converts a hydraulic power into a mechanical power. According to Figure~\ref{Hydrostatic_motor} the following dynamical model can be written:
\begin{figure}[H]
	\centering
	\includegraphics[width = 0.5\textwidth, width = 275pt, angle = 0, keepaspectratio]{figures/hydrostatic_components/hymotor_scheme_1.eps}
	\captionsetup{width=0.75\textwidth}		
	\caption{Hydrostatic motor.}
	\label{Hydrostatic_motor}
\end{figure}
\begin{equation}
	\left\lbrace \begin{aligned}
		J_m\frac{d}{dt}{\omega}_m &= \Delta p D_m +\tau_m-b_m\omega_m \\[6pt]
		\frac{V}{\beta}\frac{d}{dt}{\Delta p} &= q_2 - q_{mL} - \omega_m D_m \\[6pt]
		q_{mL} &= \Big(\frac{1}{\eta_{m}^{vol}}-1\Big) \omega_m D_m \\[6pt]
		q_2 &= q_3 
	\end{aligned}\right. 
\end{equation}
\section{Pressure relief linearized flow equation}
In this section we want to find a linearized relation between a pressure relief orifice and the corresponding flow. We start considering a valve orifice as shown in Figure~\ref{orifice}
\begin{figure}[H]
	\centering
	\includegraphics[width = 0.5\textwidth, width = 300pt, angle = 0, keepaspectratio]{figures/hydrostatic_components/orifice.eps}
	\captionsetup{width=0.75\textwidth}		
	\caption{Valve orifice used along the whole document.}
	\label{orifice}
\end{figure}
The best equation that we have available for modeling the volumetric flow rate through a control valve is the classic orifice equation 
\begin{equation}
	q = AC_d\sqrt{\frac{2}{\rho}(p-p_t)}
\end{equation}
where $A = x_v^{max}(x_v+x_v^0)$ is the discharge flow area of the valve, $C_d$ is the discharge coefficient that we consider $C_d=1$, $\rho$ is the fluid density and $p-p_t$ is the pressure drop across the valve.

We can write the linearized orifice equation as follows
\begin{equation}\label{lin_orifice_1}
	\Delta q = \Biggl[\frac{\partial q}{\partial x_v}\Biggr]_0\Delta x_v+\Biggl[\frac{\partial q}{\partial p}\Biggr]_0\Delta p
\end{equation}
where 
\begin{equation}\label{lin_orifice_2}
	\frac{\partial q}{\partial x_v} = x_v^{max}\sqrt{\frac{2}{\rho}p_\text{nom}} = k_q
\end{equation}
where $p_\text{nom}$ is the nominal pressure which the relief works
\begin{equation}\label{lin_orifice_3}
	\frac{\partial q}{\partial p} = \frac{x_v^{max}x_v^0}{2\rho\,p_\text{nom}} = k_c
\end{equation}
which results in 
\begin{equation}\label{lin_orifice_4}
	\Delta q = k_q\Delta x_v + k_c\Delta p
\end{equation}
in case we consider to linearize the orifice around $x_v(t)=0\rightarrow\Delta x_v(t) = x_v(t)$ and we set $x_v^0 = 0$ the linearized flow through the relief valve becomes as follows
\begin{equation}\label{lin_orifice_5}
	q = k_q x_v
\end{equation}
\section{Pressure controlled pump}
\subsection{Model derivation}
Pressure controlled pump is a hydraulic equipment used to generate a constant pressure source. The pump is represent by a variable wash plate displacement pump and is governed by the output of a pressure regulator where in our modelization is a proportional-integral control (PI-control).
\begin{figure}[H]
	\centering
	\includegraphics[width = 0.5\textwidth, width = 400pt, angle = 0, keepaspectratio]{figures/hydrostatic_components/pressure_controlled_pump_3.eps}
	\captionsetup{width=0.75\textwidth}		
	\caption{Pressure controlled pump.}
	\label{pcp_fig_1}
\end{figure}
The pump is connected via mechanical shaft to the primary source speed which in our case is a IC-engine. In our modelization we have considered a pump capable of two quadrant operative working range coupled with a relief valve. The pump can be selected to work also at one quadrant. Here the mathematical model

The flow through an orifice is here modelized with first order linearization 

\begin{equation}
	\begin{aligned}
		&k_q = x_{rf}^{\text{max}}\sqrt{\frac{2}{\rho\,p_{\text{nom}}}} \\[6pt]
		%	&k_c = \frac{x_{rf}^{\text{max}}}{\sqrt{\frac{2}{\rho\,p_{\text{nom}}}}}
	\end{aligned}
\end{equation}
where $x_{rf}^{\text{max}}$ is the maximum orifice opening of the relief valve, $\rho \big[\SI{}{\kilogram\per\cubic\meter}\big]$ is the density of the fluid.

%, $\beta \big[\SI{}{\pascal}\big]$ is the bulk modulus.

The PI-controller can be represented as follows
\begin{equation}
	\left\lbrace \begin{aligned}
		&\tilde{p} = \frac{1}{p^{\text{nom}}}\Big(p^{\text{ref}} - p\Big)\\[6pt]
		&v_d = k_p\,\tilde{p} + v_d^i \\[6pt]
		&\frac{dv_d^i}{dt} = k_i\,\tilde{p}	
	\end{aligned}\right. 
\end{equation}
The output of the controller $v_d(t)$ is a per-unit volumetric displacement. $p^{\text{nom}}$ is the nominal pressure of the system which is set to $p^{\text{nom}}=\SI{250}{\bar}$.

Concerning the mathematical model of the system we can start from the mechanical part as follows
\begin{equation}
	\begin{aligned}
		&J\frac{d\omega_e}{dt}=\tau_e-\omega_eb - pv_dv_d^{\text{nom}}	
	\end{aligned}
\end{equation}
where $\omega_e$ is the pump rotational speed and $v_d^{\text{nom}}$ is the nominal volumetric displacement of the pump, set to $v_d^{\text{nom}} = \SI{62.3}{\cubic\centi\meter\per\second}$ and $b$ is the friction factor extrapolate from the mechanical efficiency.
Concerning the hydraulic model we can write the following equations
\begin{equation}
	\left\lbrace \begin{aligned}
		&\frac{dp(t)}{dt} = \frac{\beta}{V}\Big[q_p(t)-q_L(t)-q(t)\Big]  \\[6pt]
		&q_p(t) = \omega_e(t)v_d^{\text{nom}} v_d(t) \\[6pt]
		&q_L(t) = p(t)C_L + x_{rf}(t)k_q
	\end{aligned}\right. 
\end{equation}
where
\begin{equation}
	\left\lbrace 
	\begin{aligned}
		&x_{rf} = x_{rf}^{\text{max}}\quad\text{if }p>p^\text{max} \\[6pt]
		&x_{rf} = 0\quad\text{if } p<p^\text{set} \\[6pt]
		&x_{rf} = \frac{x_{rf}^{\text{max}}}{p^\text{reg}}\big(p-p^\text{set}\big)\quad\text{if } p^\text{set}<p<p^\text{max}\\[6pt]
		&p^\text{max} = p^\text{set}+p^\text{reg}
	\end{aligned}\right. 
\end{equation}
and where $C_L$ is the leakage coefficient.
\begin{mybox}
	\textbf{The whole model results as follows}
	\begin{equation}
		\left\lbrace \begin{aligned}
			&\tilde{p} = \frac{1}{p^{\text{nom}}}\Big(p^{\text{ref}} - p\Big)\\[6pt]
			&v_d = k_p\,\tilde{p} + v_d^i \\[6pt]
			&\frac{dv_d^i}{dt} = k_i\,\tilde{p}\\[6pt]	
			&J\frac{d\omega_e}{dt}=\tau_e-\omega_eb - pv_dv_d^{\text{nom}}	 \\[6pt]
			&\frac{dp(t)}{dt} = \frac{\beta}{V}\Big[q_p(t)-q_L(t)-q(t)\Big]  \\[6pt]
			&q_p(t) = \omega_e(t)v_d^{\text{nom}} v_d(t) \\[6pt]
			&q_L(t) = p(t)C_L + x_{rf}(t)k_q
		\end{aligned}\right. 
	\end{equation}
\end{mybox}
The leakage coefficient $C_L$ and the friction coefficient $b$ has been derived as follows
\begin{enumerate}
	\item The leakage coefficient is calculated considering a flow leakage $q_\text{leak} = \SI{5}{\liter\per\min}$ at the nominal pressure of $p^{\text{nom}} = \SI{250}{\bar}$, which results in \begin{equation} C_L = \SI{4e-12}{\cubic\meter\per\second\per\pascal}\end{equation}
	\item The coefficient friction $b$ has been calculated considering a mechanical efficiency of $\eta_\text{mech}=0.9$ at a nominal speed which has been considered $\omega_e^\text{nom}= \SI{1900}{\per\min}$, which means \begin{equation} b=\frac{P^\text{nom}\Big(1-\eta_\text{mech}\Big)}{\big(\omega_e^\text{nom}\big)^2} = \SI{0.124}{\newton\meter\per\radian\second} \end{equation}
\end{enumerate}

%\subsection{Eaton ADU062}
%In Leitwolf equipment we consider the pump unit \textbf{Eaton ADU062} with the following characteristics
%\begin{itemize}
%	\item Nominal volumetric displacement 
%	\item Maximum operating pressure
%\end{itemize}

\section{Four-way spool valve.}
Consider the four-way spool valve shown in Figure~\ref{fourway_three_lands}. The four orifices are completely analogous to the four arms of a Wheatstone bridge as Figure~\ref{Wheatstone}.
\begin{figure}[H]
	\centering
	\includegraphics[width = 0.5\textwidth, width = 500pt, angle = 0, keepaspectratio]{figures/hydrostatic_components/fourway_three_lands_valve_1.eps}
	\captionsetup{width=0.75\textwidth}		
	\caption{Three-land-four-way spool valve.}
	\label{fourway_three_lands}
\end{figure}
\begin{figure}[H]
	\centering
	\includegraphics[width = 0.5\textwidth, width = 300pt, angle = 0, keepaspectratio]{figures/hydrostatic_components/weaston.eps}
	\captionsetup{width=0.75\textwidth}		
	\caption{Three-land-four-way spool valve - Wheatstone bridge representation.}
	\label{Wheatstone}
\end{figure}
Flow through the valving orifices are described by orifice equation as follows
\begin{equation}\label{three_way_valve_pflow_1}
	q_1^{\text{load}}(t) = \left\lbrace \begin{aligned}
		&x_vx_v^{\text{max}}\sqrt{\frac{2}{\rho}}\sqrt{p_{s}-p_1} - x_v^0x_v^{\text{max}}\sqrt{\frac{2}{\rho}}\sqrt{p_1}\quad \text{for} \quad x_v \ge 0 \\[6pt]
		&-x_vx_v^{\text{max}}\sqrt{\frac{2}{\rho}}\sqrt{p_{1}} + x_v^0x_v^{\text{max}}\sqrt{\frac{2}{\rho}}\sqrt{p_{s}-p_1} \quad \text{for} \quad x_v \le 0
	\end{aligned}\right. 
\end{equation}
\begin{equation}\label{three_way_valve_pflow_2}
	q_2^{\text{load}}(t) = \left\lbrace \begin{aligned}
		&x_vx_v^{\text{max}}\sqrt{\frac{2}{\rho}}\sqrt{p_{2}} - x_v^0x_v^{\text{max}}\sqrt{\frac{2}{\rho}}\sqrt{p_{s}-p_2} \quad \text{for} \quad x_v \ge 0 \\[6pt]
		&-x_vx_v^{\text{max}}\sqrt{\frac{2}{\rho}}\sqrt{p_s-p_2} + x_v^0x_v^{\text{max}}\sqrt{\frac{2}{\rho}}\sqrt{p_2} \quad \text{for} \quad x_v \le 0
	\end{aligned}\right. 
\end{equation}
In the vast majority of cases the valving orifices are \textit{matched} and \textit{symmetrical}. Matched orifices require that
\begin{equation}
	\begin{aligned}
		A_1 &= A_3 \\[6pt]
		A_2 &= A_4
	\end{aligned}
\end{equation}
and symmetrical orifices require that
\begin{equation}
	\begin{aligned}
		A_1(x_v) &= A_2(-x_v) \\[6pt]
		A_3(x_v) &= A_4(-x_v)
	\end{aligned}
\end{equation}
Therefore at the neutral position of the spool, all four orifice areas are equal.
\begin{equation}
	\begin{aligned}
		A_1(0) &= A_2(0) = x_v^0x_v^{\text{max}} \\[6pt]
	\end{aligned}
\end{equation}
With these restrictions on the orifice areas only one orifice area need be defined because the other areas follows from it. In fact, if the orifice areas are linear with valve stroke, as is usually the case, only one defining parameter is required: the width of the slot in the valve sleeve $x_v$. The rate of change of orifice area with stroke is called the \textit{opening} or \textit{area gradient} of the valve. If the valve is linear, then  the area gradient of each orifice (and so the whole valve) is $x_v$.
 


\section{Three-way spool valve.}
In the present section will give a brief discussion of the two types, critical center and open center, of three-way spool valves. Three-way spool valves must be used with an unequal area piston, Figure~\ref{threeway_valve_piston}, to provide direction reversal. The rod and head side areas of the piston should be such that a steady-state control pressure of about $p_s/2$ or 
\begin{equation}
	p_c(t) A_h - p_sA_r =0 
\end{equation}
that for $p_{c0} = p_s/2$ results in $A_h=2A_r$. This design relation allows the control pressure $p_c(t)$ to rise or fall and to provide equal acceleration and deceleration capability.
\begin{figure}[H]
	\centering
	\includegraphics[width = 0.5\textwidth, width = 300pt, angle = 0, keepaspectratio]{figures/hydrostatic_components/threeway_valve_monochamber_piston_4.eps}
	\captionsetup{width=0.75\textwidth}		
	\caption{Three-way valve actuated piston.}
	\label{threeway_valve_piston}
\end{figure}
Three-way valves find greatest use un hydro-mechanical position servo, and the critical center type ($x_v^0=0$) is usually preferred. Referring to Figure~\ref{threeway_valve_piston} consider the critical center case in which wither orifice 1 and 2 (not both) is active at one time. Letting $x_v^0=0$ to achieve a critical center, the equation for the pressure-flow curves can be written directly. Therefore
\begin{equation}\label{threeway_valve_flow}
	\left\lbrace \begin{aligned}
		q(t) &= x_v^{max}x_v\Big[\frac{2}{\rho}\Big(p_s-p_c(t)\Big)\Big]^{1/2} \quad\text{for}\quad x_v\ge 0 \\[6pt]
		q(t) &= -x_v^{max}x_v\Big[\frac{2}{\rho}p_c(t)\Big]^{1/2} \quad\text{for}\quad x_v\le 0 \\[6pt]		
	\end{aligned}\right. 
\end{equation}
The null operating point for a three-way valve is defined by $q=x_v=0$ and $p_c=p_s/2$ (supposing the external force are null). Evaluating the derivatives of Eq.~\eqref{threeway_valve_flow} at this point, we find that the null coefficients for the critical center three-way valve become
\begin{equation}
	K_{q0}=\frac{\partial q(t)}{\partial x_v}\Big|_0=x_v^{max}\sqrt{\frac{p_s}{\rho}}
\end{equation}
\begin{equation}
	K_{c0}=-\frac{\partial q(t)}{\partial p_c}\Big|_0=\frac{2x_v^{max}x_v\sqrt{p_s/\rho}}{p_s}\Big|_{x_v=0}=0
\end{equation}
\begin{equation}
	K_{p0}=\frac{\partial p_c}{\partial x_v}\Big|_0=\frac{p_s}{x_v}\Big|_{x_v=0}=\infty
\end{equation}
For operation within the underlap region (open center three-way valve), the pressure flow curves given by the equation
\begin{equation}
	q(t)=x_v^{max}x_v^0\sqrt{2p_s/\rho}\Bigg[\Big(1+\frac{x_v(t)}{x_v^0}\Big)\Big(1-\frac{p_c(t)}{p_s}\Big)^{1/2}-\Big(1+\frac{x_v(t)}{x_v^0}\Big)\Big(\frac{p_c(t)}{p_s}\Big)^{1/2}\Bigg]
\end{equation}
and the null coefficients are
\begin{equation}
	K_{q0}=\frac{\partial q(t)}{\partial x_v}\Big|_0=2x_v^{max}\sqrt{\frac{p_s}{\rho}}
\end{equation}
\begin{equation}
	K_{c0}=-\frac{\partial q(t)}{\partial p_c}\Big|_0=\frac{2x_v^{max}x_v^0\sqrt{p_s/\rho}}{p_s}
\end{equation}
\begin{equation}
	K_{p0}=\frac{\partial p_c}{\partial x_v}\Big|_0=\frac{p_s}{x_v^0}
\end{equation}
The leakage or center flow at null is given by the equation
\begin{equation}
	q_c(t)=x_v^{max}x_v^0\sqrt{\frac{p_s}{\rho}}
\end{equation}
Since three-way valves are usually used in hydro-mechanical servos, the source positioning the spool is electromechanical and quite stiff compared with the force load imposed by the spool. For this reason the force equation describing the spool motion is usually unimportant and flow forces are not of interest.


\section{Four-way four-land valve}
In this section the \textit{open center spool valve} (built as four-way four-land valve) is investigated (see Figure~\ref{four_way_four_land_valve}). Additional investigation are taken into account for the case study of an piston actuator (see Figure~\ref{four_way_four_land_valve_piston})
\begin{figure}[H]
	\centering
	\includegraphics[width = 0.5\textwidth, width = 350pt, angle = 0, keepaspectratio]{figures/hydrostatic_components/underlapped_fourway_fourland_valve.eps}
	\captionsetup{width=0.75\textwidth}		
	\caption{Underlapped configuration of the four-way valve.}
	\label{four_way_four_land_valve}
\end{figure}
Although this type valve is more limited in application, knowledge of its characteristics will broaden our understanding of valves.

Consider the four-way spool shown in Figure~\ref{four_way_four_land_valve}. When the valve is centered, the underlap of the supply and return ports are identical with a value of $x_v^0$. It is assumed the valve is matched and symmetrical and that its operation remains within the underlap region, that is $\abs{x_v}\le x_v^0$. The orifice areas therefore are
\begin{align}
	A_4 &= x_v^{\text{max}}\Big(x_v^0+x_v\Big)=A_5 \\[6pt]
	A_3 &= x_v^{\text{max}}\Big(x_v^0-x_v\Big)=A_6
\end{align}  
The continuity conditions relays in the following constraints
\begin{align}
	q_1 &= q_3 - q_4 \\[6pt]
	q_2 &= q_6 - q_5 \\[6pt] 	
	q_s &= q_3 + q_5 \\[6pt] 	
	q_t &= q_4 + q_6 
\end{align}  
Now we consider the case where the open center spool drivers a piston as shown in Figure~\ref{four_way_four_land_valve_piston}
\begin{figure}[H]
	\centering
	\includegraphics[width = 0.5\textwidth, width = 500pt, angle = 0, keepaspectratio]{figures/hydrostatic_components/fourway_fourland_valve_bichamber_piston_open.eps}
	\captionsetup{width=0.75\textwidth}		
	\caption{Four-way four-land valve bi-chamber piston system.}
	\label{four_way_four_land_valve_piston}
\end{figure}
The dynamical equations become as follows
\begin{equation}
	\left\lbrace \begin{aligned}
		x_v &> x_v^0 \\[6pt]
		q_3 &= (x_v^0 + x_v)x_v^{\text{max}}\sqrt{\frac{2}{\rho}}\sqrt{p_s-p_1}  \\[6pt]
		q_4 &= 0  \\[6pt]
		q_5 &= 0  \\[6pt]
		q_6 &= (x_v^0 + x_v)x_v^{\text{max}}\sqrt{\frac{2}{\rho}}\sqrt{p_2-p_t}
	\end{aligned}\right. 
\end{equation}
\begin{equation}
	\left\lbrace \begin{aligned}
		x_v &< -x_v^0 \\[6pt]
		q_3 &= 0  \\[6pt]
		q_4 &= (x_v^0 - x_v)x_v^{\text{max}}\sqrt{\frac{2}{\rho}}\sqrt{p_1-p_t}  \\[6pt]
		q_5 &= (x_v^0 - x_v)x_v^{\text{max}}\sqrt{\frac{2}{\rho}}\sqrt{p_s-p_2}  \\[6pt]
		q_6 &= 0
	\end{aligned}\right. 
\end{equation}
\begin{equation}
	\left\lbrace \begin{aligned}
		x_v^0 &\le x_v \le -x_v^0 \\[6pt]
		q_3 &= (x_v^0 + x_v)x_v^{\text{max}}\sqrt{\frac{2}{\rho}}\sqrt{p_s-p_1}  \\[6pt]
		q_4 &= (x_v^0 - x_v)x_v^{\text{max}}\sqrt{\frac{2}{\rho}}\sqrt{p_1-p_t}  \\[6pt]
		q_5 &= (x_v^0 - x_v)x_v^{\text{max}}\sqrt{\frac{2}{\rho}}\sqrt{p_s-p_2}  \\[6pt]
		q_6 &= (x_v^0 + x_v)x_v^{\text{max}}\sqrt{\frac{2}{\rho}}\sqrt{p_2-p_t}
	\end{aligned}\right. 
\end{equation}
where the dynamical equation of the piston become
\begin{equation}
	\left\lbrace \begin{aligned}
		\frac{d}{dt}p_1 &= \beta\Bigg[\frac{q_1-A_p^1v_p}{V_0 + A_p^1x_p}\Bigg] \\[6pt]
		\frac{d}{dt}p_2 &= \beta\Bigg[\frac{-q_2+A_p^2v_p}{V_0 - A_p^2x_p}\Bigg] \\[6pt]
		m_p\frac{dv_p}{dt} &= \Big(A_p^1p_1-A_p^2p_2\Big)-b_pv_p-f_p \\[6pt]
		\frac{dx_p}{dt} &= v_p
	\end{aligned}\right. 
\end{equation}

\chapter{Hydrostatic powertrain modelization and control}

\section{Model description}
The heavy duty vehicle power-train is in general performed by a diesel engine followed by an hydrostatic transmission which is composed by a variable displacement hydraulic pump and by a variable (or fixed) displacement hydraulic motor. In the following, we will consider the case where both pump and motor are equipped with a variable displacement (which means variable volumetric flow). The use of a both volumetric displacement pump and motor permits to extend the working operating area, in terms of speed and torque, of the whole transmission line.  Basically, This kind of transmission is often identified as a continuous variable gear-ratio.
\begin{figure}[H]
	\centering
	\includegraphics[width = 0.5\textwidth, width = 500pt, angle = 0, keepaspectratio]{figures/hydrostatic_powertrain/hydrostatic_transmission_3.eps}
	\captionsetup{width=0.5\textwidth}		
	\caption{Power train layout of an heavy duty tracked vehicle.}
	\label{hydro_transm_2}
\end{figure}
Figure~\ref{hydro_transm_1} depicts a typical hydrostatic power train, where the dynamic equations can be represented as follows
\begin{figure}[H]
	\centering
	\includegraphics[width = 0.5\textwidth, width = 500pt, angle = 0, keepaspectratio]{figures/hydrostatic_powertrain/hydrostatic_transmission_1.eps}
	\captionsetup{width=0.5\textwidth}		
	\caption{Hydrostatic transmission - no-leakage and no-viscosity case is taken into account.}
	\label{hydro_transm_1}
\end{figure}
the whole transmission power-train is made two, independent lines, one for the left side and one for the right side. In the following we write the model equations of the one single line:
\begin{equation}\left\lbrace 
	\begin{aligned}
		\dot{\Delta p_1}&=\frac{1}{\beta}\Big(D_p^1\omega_p^1-D_m^1\omega_m^1\Big) \\[8pt]
		\dot{\omega}_m^1&=\frac{1}{J_m}\Big(\tau_\vartheta^1-\tau_m^1\Big) \qquad\text{where}\qquad\tau_m^1=D_m^1\Delta p_1\\[8pt]
		\dot{\tau}_\vartheta^1&=k_\vartheta\Big(\omega_m^1-\omega_l^1\Big) \\[8pt]
		\dot{\omega}_p^1&=\frac{1}{J_p}\Big(\tau_\vartheta^{p1}-\tau_p^1\Big)\qquad\text{where}\qquad\tau_p^1=D_p^1\Delta p_1\\[8pt]
		\dot{\tau}_\vartheta^{p1}&=k_\vartheta\Big(\omega_e-\omega_p^1\Big)
	\end{aligned}\right. 
\end{equation}
where $D_m^1 = Vd_m^1D_m^{nom}$, $D_p^1 = Vd_p^1D_p^{nom}$ and where $D^{nom}$ means the nominal volumetric displacement of the pump or motor. Moreover,  
\begin{equation}\left\lbrace 
	\begin{aligned}
		\dot{\omega}_l^1&=\frac{1}{J_l}\Big(\tau_\vartheta^1-\tau_l^1\Big) \\[8pt]
		\dot{\omega}_e&=\frac{1}{J_e}\Big(\tau_e -\tau_\vartheta^{p1}-\tau_\vartheta^{p2}\Big)
	\end{aligned}\right. 
\end{equation}
The case study we are going to consider is a drive-train with the following data characteristics
\begin{itemize}
	\item $D_m^{nom} = \SI{252.8}{\cubic\centi\meter}$
	\item $D_p^{nom} = \SI{147.2}{\cubic\centi\meter}$
	\item $R = \SI{0.44056}{\meter}$
	\item $D_{tg} = \SI{41.4}{}$
	\item $v_{tr}^{max} = \SI{11.0}{\kilo\meter\per\hour}$
	\item $\eta_p^m=\SI{0.909}{}$
	\item $\eta_p^v=\SI{0.959}{}$
	\item $\eta_m^m=\SI{0.939}{}$
	\item $\eta_m^v=\SI{0.943}{}$
\end{itemize}
To evaluate the performance of the power-train a typical power test is made as follows
\begin{itemize}
	\item the engine is run at full open throttle
	\item the motor displacement is kept at maximum volumetric capability
	\item the pump displacement is ramp up slowly from 0\% up to 100\%
	\item a load (torque load) is applied to the end of the power train (in our case, at the shaft of the hydraulic motor) in order to control the engine speed at a given constant set-point 
	\item when the pump displacement reached the maximum value of 100\% the volumetric displacement of the motor is ramp down up to the minimum value of 36\% (of the nominal) which represent the minimum volumetric displacement of the hydraulic motor.
	\item as already said above, the power train load is applied by an external PI loop control and is continuously adapted in order to keep the engine speed at a constant set-point  
\end{itemize}
In the following we can see the performance results obtained by the above test. Figure~\ref{limit_curves1} shows the maximum torque-speed limit curve according the maximum engine power (\textit{torque $\times$ speed}) shown in the Figure~\ref{limit_curves2}
\begin{figure}[H]
	\centering
	\includegraphics[width = 0.5\textwidth, width = 300pt, angle = 0, keepaspectratio]{figures/hydrostatic_powertrain/limit_curves.eps}
	\captionsetup{width=0.5\textwidth}		
	\caption{Power train limit curves - for one line power train.}
	\label{limit_curves1}
\end{figure}
the curve shown in Figure~\ref{limit_curves1} represents the torque/speed trajectory of the hydraulic motor (or of an electrical motor in case of electrification) which drives the drive-gear as shown in Figure~\ref{travel_gear}
\begin{figure}[H]
	\centering
	\includegraphics[width = 0.5\textwidth, width = 300pt, angle = 0, keepaspectratio]{figures/hydrostatic_powertrain/travel_gear.eps}
	\captionsetup{width=0.5\textwidth}		
	\caption{drive-gear representation.}
	\label{travel_gear}
\end{figure}
\begin{figure}[H]
	\centering
	\begin{subfigure}{.5\textwidth}
		\centering
		\includegraphics[width = 0.5\textwidth, width = 200pt, angle = 0, keepaspectratio]{figures/hydrostatic_powertrain/engine_curves.eps}
		\captionsetup{width=0.5\textwidth}		
		\caption{Engine speed and torque along the maximum power limit curve.}
		\label{limit_curves2}
	\end{subfigure}%
	\begin{subfigure}{.5\textwidth}
		\centering
		\includegraphics[width = 0.5\textwidth, width = 225pt, angle = 0, keepaspectratio]{figures/hydrostatic_powertrain/volumetric_diplacements.eps}
		\captionsetup{width=0.5\textwidth}		
		\caption{Volumetric displacements trajectories in \%.}
		\label{limit_curves3}
	\end{subfigure}
	\caption{Power train limit curves - for one line.}
	\label{}
\end{figure}
Results shown in Figure~\ref{limit_curves1} can also be derived from the vehicle data. The hydraulic motor is connected to the track (undercarriages) via a gear device which is also called \textit{drive-gear}. 

Let $D_{tg}=41.4$ be the gear ratio of the \textit{drive-gear} and let $R=\SI{0.44056}{\meter}$ be its radius. Considering a maximum (nominal) vehicle speed of $v_{tr}^{1}=\SI{11}{\kilo\meter\per\hour}$ we obtain a maximum (nominal) hydraulic rotor speed of $\omega_m^{nom} = \SI{2742}{\per\minute}$. 

For design purpose we are going to consider a maximum hydraulic motor speed of (we are considering a worst case) $$\omega_m^{max} = \hat{\omega}_m = \SI{3100}{\per\minute}$$. In the matter of fact, the final $\omega_m$ speed is also affected by the set-point of the engine speed.
\begin{figure}[H]
	\centering
	\includegraphics[width = 0.5\textwidth, width = 175pt, angle = 0, keepaspectratio]{figures/hydrostatic_powertrain/track_1.eps}
	\captionsetup{width=0.5\textwidth}		
	\caption{Track representation.}
	\label{track_1}
\end{figure}
An important aspect which has to be taken into account is to evaluate the hydrostatic transmission at steady state condition, in fact, in steady state condition the following relations can be considered:
\begin{equation}
	\begin{aligned}
		\eta_p^vD_p\omega_p=\frac{D_m\omega_m}{\eta_p^v}
	\end{aligned}
\end{equation}
where $\eta_p^v$ and $\eta_m^v$ are the volumetric efficiency of the pump and motor respectively.
which results in
\begin{equation}
	\boxed{	\begin{aligned}
			\omega_m=\Bigg(\frac{D_p}{D_m}\Bigg)\omega_p\Big(\eta_p^v\eta_m^v\Big)
	\end{aligned}}
\end{equation}
and 
\begin{equation}
	\begin{aligned}
		\frac{\tau_m\omega_m}{\eta_m^m}=\eta_p^m\tau_p\omega_p
	\end{aligned}
\end{equation}
or
\begin{equation}
	\boxed{	\begin{aligned}
			\tau_p =\frac{1}{\eta_p^m\eta_m^m}\Bigg(\frac{D_p}{D_m}\Bigg)\tau_m
	\end{aligned}}
\end{equation}
where $\eta_p^m$ and $\eta_m^m$ are the mechanical efficiency of the pump and motor respectively, $D_m = D_m^{nom}V_d^m$ where $$V_d^m\in\big[0.36,\ 1\big]$$ and $D_p = D_p^{nom}V_d^p$ where $$V_d^p\in\big[-1,\ 1\big].$$
\section{Components parameters identification}

\section{Model validation}
The following document reports a comparison between the power-train experimental data of the \textbf{hydrostatic-vehicle} and the corresponding \textit{digital twin model} developed by the DTL. In order to keep the following document readable and concise, a reduced number of comparisons will be here reported. The tests which shown a relevant impact in the design and parametrization of the model are here reported
\begin{itemize}
	\item Test 28.1
	\item Test 24.1
	\item Test 14.1
	\item Test 12.1	(lifted vehicle)
\end{itemize} 
where in the first three cases the vehicle is laid on the ground, while in the last one the vehicle is suspended. 
Additional two tests were added to the document for completeness:
\begin{itemize}
	\item Test 1.1	(lifted vehicle)
	\item Test 1.2	(lifted vehicle)
\end{itemize}

Most of the parameters involved in the fitting of the experimental data come from the two main control components: pump servo control and motor servo control. Figure~\ref{pump_servo} and figure~\ref{motor_servo} show the servo control layout for both systems. 


All test comparison regarding the case \textit{laid on ground} report a good reproduction of the real behaviour, both steady-state and dynamic. On the other hand the comparison of the test~12.1 reports a misalignment between measured and simulated engine torque. This misalignment numerical evaluated in the section~\ref{pump_efficiency_comparison}. This engine torque misalignment is much less evident in Test~1.1 and Test~1.2, where remain a marginal gap only for very low driveline differential pressure.

\begin{figure}[H]
	\centering
	\includegraphics[width = 450pt, angle = 0, keepaspectratio]{figures/hydrostatic_powertrain/validation/figures/pump_servo_2.eps}
	\captionsetup{width=0.75\textwidth}		
	\caption{Pump servo layout.}
	\label{pump_servo}
\end{figure}

\begin{figure}[H]
	\centering
	\includegraphics[width = 525pt, angle = 0, keepaspectratio]{figures/hydrostatic_powertrain/validation/figures/motor_servo_3.eps}
	\captionsetup{width=0.75\textwidth}		
	\caption{Motor servo layout.}
	\label{motor_servo}
\end{figure}


\subsection{Test 28.1 - comparison}
In this test the vehicle is moved in forward and backward in a step way. In the following, the main power-train quantities are compared; solid-black from simulation and solid-gray from experimental data. Figure~\ref{dynamic_track} shown the comparison of the vehicle dynamic behaviour during the step from forward to backward. Dynamic pressure behaviour can also be seen in figures~\ref{dynamic_DP}, figures~\ref{dynamic_AP}, figures~\ref{dynamic_BP}, figures~\ref{dynamic_M4}. At the end of each section a \textit{lesson learned} section is reported to focus on the results of the comparison.

\begin{figure}[H]
	\centering
	\begin{subfigure}{.5\textwidth}
	\centering
	\includegraphics[width = 240pt, angle = 0, keepaspectratio]{figures/hydrostatic_powertrain/validation/test_28_1_fig/engine_quantities_test_28_1.eps}
	\captionsetup{width=0.75\textwidth}		
	\caption{Engine torque and speed.}
	\label{}
	\end{subfigure}%
	\begin{subfigure}{.5\textwidth}
	\centering
	\includegraphics[width = 240pt, angle = 0, keepaspectratio]{figures/hydrostatic_powertrain/validation/test_28_1_fig/abs_driveline_pressureA_test_28_1.eps}
	\captionsetup{width=0.75\textwidth}		
	\caption{Driveline pressure A.}
	\label{}
	\end{subfigure}
\caption{Simulation results Test 28.1.}
\label{}
\end{figure}

\begin{figure}[H]
	\centering
	\begin{subfigure}{.5\textwidth}
	\centering
	\includegraphics[width = 240pt, angle = 0, keepaspectratio]{figures/hydrostatic_powertrain/validation/test_28_1_fig/abs_driveline_pressureA_test_28_1_zoom.eps}
	\captionsetup{width=0.75\textwidth}		
	\caption{Driveline pressure A - dynamic behaviour .}
	\label{dynamic_AP}
	\end{subfigure}%
	\begin{subfigure}{.5\textwidth}
	\centering
	\includegraphics[width = 240pt, angle = 0, keepaspectratio]{figures/hydrostatic_powertrain/validation/test_28_1_fig/abs_driveline_pressureB_test_28_1.eps}
	\captionsetup{width=0.75\textwidth}		
	\caption{Driveline pressure B.}
	\label{}
	\end{subfigure}
\caption{Simulation results Test 28.1.}
\label{}
\end{figure}

\begin{figure}[H]
	\centering
	\begin{subfigure}{.5\textwidth}
	\centering
	\includegraphics[width = 240pt, angle = 0, keepaspectratio]{figures/hydrostatic_powertrain/validation/test_28_1_fig/abs_driveline_pressureB_test_28_1_zoom.eps}
	\captionsetup{width=0.75\textwidth}		
	\caption{Driveline pressure B - dynamic behaviour.}
	\label{dynamic_BP}
	\end{subfigure}%
\begin{subfigure}{.5\textwidth}
	\centering
	\includegraphics[width = 240pt, angle = 0, keepaspectratio]{figures/hydrostatic_powertrain/validation/test_28_1_fig/diff_driveline_pressure_test_28_1.eps}
	\captionsetup{width=0.75\textwidth}		
	\caption{Differential driveline pressure.}
	\label{}
	\end{subfigure}
\caption{Simulation results Test 28.1.}
\label{}
\end{figure}

\begin{figure}[H]
	\centering
	\begin{subfigure}{.5\textwidth}
	\centering
	\includegraphics[width = 240pt, angle = 0, keepaspectratio]{figures/hydrostatic_powertrain/validation/test_28_1_fig/diff_driveline_pressure_test_28_1_zoom.eps}
	\captionsetup{width=0.75\textwidth}		
	\caption{Differential driveline pressure - dynamic behaviour.}
	\label{dynamic_DP}
	\end{subfigure}%
	\begin{subfigure}{.5\textwidth}
	\centering
	\includegraphics[width = 240pt, angle = 0, keepaspectratio]{figures/hydrostatic_powertrain/validation/test_28_1_fig/HMM4_pressure_test_28_1.eps}
	\captionsetup{width=0.75\textwidth}		
	\caption{Hydro motor servo pressure.}
	\label{}
	\end{subfigure}
\caption{Simulation results Test 28.1.}
\label{}
\end{figure}


\begin{figure}[H]
	\centering
	\begin{subfigure}{.5\textwidth}
	\centering
	\includegraphics[width = 240pt, angle = 0, keepaspectratio]{figures/hydrostatic_powertrain/validation/test_28_1_fig/HMM4_pressure_test_28_1_zoom.eps}
	\captionsetup{width=0.75\textwidth}		
	\caption{Hydro motor servo pressure - dynamic behaviour.}
	\label{dynamic_M4}
	\end{subfigure}%
	\begin{subfigure}{.5\textwidth}
	\centering
	\includegraphics[width = 240pt, angle = 0, keepaspectratio]{figures/hydrostatic_powertrain/validation/test_28_1_fig/p_supply_test_28_1.eps}
	\captionsetup{width=0.75\textwidth}		
	\caption{Charge pressure and fan driver pressure.}
	\label{}
	\end{subfigure}
\caption{Simulation results Test 28.1.}
\label{}
\end{figure}

\begin{figure}[H]
	\centering
	\begin{subfigure}{.5\textwidth}
	\centering
	\includegraphics[width = 240pt, angle = 0, keepaspectratio]{figures/hydrostatic_powertrain/validation/test_28_1_fig/vehicle_track_speed_test_28_1.eps}
	\captionsetup{width=0.75\textwidth}		
	\caption{Tracks speed.}
	\label{}
	\end{subfigure}%
	\begin{subfigure}{.5\textwidth}
	\centering
	\includegraphics[width = 240pt, angle = 0, keepaspectratio]{figures/hydrostatic_powertrain/validation/test_28_1_fig/vehicle_track_speed_zoom_test_28_1.eps}
	\captionsetup{width=0.75\textwidth}		
	\caption{Tracks speed - dynamic behaviour.}
	\label{dynamic_track}
	\end{subfigure}
\caption{Simulation results Test 28.1.}
\label{}
\end{figure}

\begin{figure}[H]
	\centering
	\begin{subfigure}{.5\textwidth}
	\centering
	\includegraphics[width = 240pt, angle = 0, keepaspectratio]{figures/hydrostatic_powertrain/validation/test_28_1_fig/volumetric_displacement_test_28_1.eps}
	\captionsetup{width=0.75\textwidth}		
	\caption{Volumetric displacement request and actuated}
	\label{}
	\end{subfigure}%
	\begin{subfigure}{.5\textwidth}
	\centering
	\includegraphics[width = 240pt, angle = 0, keepaspectratio]{figures/hydrostatic_powertrain/validation/test_28_1_fig/efficiency_test_28_1.eps}
	\captionsetup{width=0.75\textwidth}		
	\caption{Hydraulic efficiencies of pump and motor}
	\label{}
	\end{subfigure}
\caption{Simulation results Test 28.1.}
\label{}
\end{figure}

\subsection{Lesson Learned from Test 28.1}
Test~28.1 can be considered one of the most important test-scenario, where not only the dynamic of the pump and motor servo plays an important role, but also their coordination. In particular from this test can be recognized the necessity to have a motor-servo dynamic faster then pump-servo dynamic. This test involves very low values of motor displacement coupled with high dynamic step, which results in high driveline pressure steps. The driveline differential pressure change suddenly from positive to negative (and vice-versa) and the correct setting of the driveline check valves impacts into the simulation results. A typical effects of wrong driveline check valves parameters are the presence of negative pressure inside the chamber of the motor servo control.
A correct setting of the \textbf{Shuttle Valve} also is of fundamental importance for the correct behaviour of the model, a wrong setting, again involve negative pressure in the motor servo chambers. Negative pressure inside any chamber is forbidden and simulation stops when it happen.  


\subsection{Test 24.1 - comparison}
In this test the vehicle is moved in backward and different level of speed are reached by the use of the motor displacement. This test was useful to set the boundary limits of the mechanical efficiency of the pump and motor look-up-table which is currently set between $0.35$ to $1$.
\begin{figure}[H]
	\centering
	\begin{subfigure}{.5\textwidth}
	\centering
	\includegraphics[width = 240pt, angle = 0, keepaspectratio]{figures/hydrostatic_powertrain/validation/test_24_1_fig/engine_quantities_test_24_1.eps}
	\captionsetup{width=0.75\textwidth}		
	\caption{Engine torque and speed.}
	\label{}
	\end{subfigure}%
	\begin{subfigure}{.5\textwidth}
	\centering
	\includegraphics[width = 240pt, angle = 0, keepaspectratio]{figures/hydrostatic_powertrain/validation/test_24_1_fig/abs_driveline_pressureA_test_24_1.eps}
	\captionsetup{width=0.75\textwidth}		
	\caption{Driveline pressure A.}
	\label{}
	\end{subfigure}
\caption{Simulation results Test 24.1.}
\label{}
\end{figure}

\begin{figure}[H]
	\centering
	\begin{subfigure}{.5\textwidth}
	\centering
	\includegraphics[width = 240pt, angle = 0, keepaspectratio]{figures/hydrostatic_powertrain/validation/test_24_1_fig/abs_driveline_pressureB_test_24_1.eps}
	\captionsetup{width=0.75\textwidth}		
	\caption{Driveline pressure B.}
	\label{}
	\end{subfigure}%
	\begin{subfigure}{.5\textwidth}
	\centering
	\includegraphics[width = 240pt, angle = 0, keepaspectratio]{figures/hydrostatic_powertrain/validation/test_24_1_fig/diff_driveline_pressure_test_24_1.eps}
	\captionsetup{width=0.75\textwidth}		
	\caption{Differential driveline pressure.}
	\label{}
	\end{subfigure}
\caption{Simulation results Test 24.1.}
\label{}
\end{figure}

\begin{figure}[H]
	\centering
	\begin{subfigure}{.5\textwidth}
	\centering
	\includegraphics[width = 240pt, angle = 0, keepaspectratio]{figures/hydrostatic_powertrain/validation/test_24_1_fig/HMM4_pressure_test_24_1.eps}
	\captionsetup{width=0.75\textwidth}		
	\caption{Hydro motor servo pressure.}
	\label{}
	\end{subfigure}%
	\begin{subfigure}{.5\textwidth}
	\centering
	\includegraphics[width = 240pt, angle = 0, keepaspectratio]{figures/hydrostatic_powertrain/validation/test_24_1_fig/p_supply_test_24_1.eps}
	\captionsetup{width=0.75\textwidth}		
	\caption{Charge pressure and fan driver pressure.}
	\label{}
	\end{subfigure}
\caption{Simulation results Test 24.1.}
\label{}
\end{figure}

\begin{figure}[H]
	\centering
	\begin{subfigure}{.5\textwidth}
	\centering
	\includegraphics[width = 240pt, angle = 0, keepaspectratio]{figures/hydrostatic_powertrain/validation/test_24_1_fig/vehicle_track_speed_test_24_1.eps}
	\captionsetup{width=0.75\textwidth}		
	\caption{Tracks speed.}
	\label{}
	\end{subfigure}%
	\begin{subfigure}{.5\textwidth}
	\centering
	\includegraphics[width = 240pt, angle = 0, keepaspectratio]{figures/hydrostatic_powertrain/validation/test_24_1_fig/volumetric_displacement_test_24_1.eps}
	\captionsetup{width=0.75\textwidth}		
	\caption{Volumetric displacement request and actuated}
	\label{}
	\end{subfigure}
\caption{Simulation results Test 24.1.}
\label{}
\end{figure}

\begin{figure}[H]
	\centering
	\includegraphics[width = 240pt, angle = 0, keepaspectratio]{figures/hydrostatic_powertrain/validation/test_24_1_fig/efficiency_test_24_1.eps}
	\captionsetup{width=0.75\textwidth}		
	\caption{Hydraulic efficiencies of pump and motor}
	\label{}
\end{figure}
\subsection{Lesson Learned from Test 24.1}
The Test~24.1 involves high values of acceleration and deceleration which impact into the dynamic behaviour of the \textbf{Check Valve} and in particular of the \textbf{Shuttle Valve} which a wrong setting results, again, in a presence of negative pressure in the chambers of the motor servo. Most of issues in the stability and quality of the model come from the modelization of the \textbf{Shuttle Valve}.


\subsection{Test 14.1 - comparison}
In this test the vehicle is moved in forward and different level of speed are reached by the use of the  motor displacement. The final step is brings the from the maximum speed to zero with the maximum acceleration. This test was particularly stress-full to find the correct implementation of the drive line check valve.
\begin{figure}[H]
	\centering
	\begin{subfigure}{.5\textwidth}
	\centering
	\includegraphics[width = 240pt, angle = 0, keepaspectratio]{figures/hydrostatic_powertrain/validation/test_14_1_fig/engine_quantities_test_14_1.eps}
	\captionsetup{width=0.75\textwidth}		
	\caption{Engine torque and speed.}
	\label{}
	\end{subfigure}%
	\begin{subfigure}{.5\textwidth}
	\centering
	\includegraphics[width = 240pt, angle = 0, keepaspectratio]{figures/hydrostatic_powertrain/validation/test_14_1_fig/abs_driveline_pressureA_test_14_1.eps}
	\captionsetup{width=0.75\textwidth}		
	\caption{Driveline pressure A.}
	\label{}
	\end{subfigure}
\caption{Simulation results Test 14.1.}
\label{}
\end{figure}

\begin{figure}[H]
	\centering
	\begin{subfigure}{.5\textwidth}
	\centering
	\includegraphics[width = 240pt, angle = 0, keepaspectratio]{figures/hydrostatic_powertrain/validation/test_14_1_fig/abs_driveline_pressureB_test_14_1.eps}
	\captionsetup{width=0.75\textwidth}		
	\caption{Driveline pressure B.}
	\label{}
	\end{subfigure}%
	\begin{subfigure}{.5\textwidth}
	\centering
	\includegraphics[width = 240pt, angle = 0, keepaspectratio]{figures/hydrostatic_powertrain/validation/test_14_1_fig/diff_driveline_pressure_test_14_1.eps}
	\captionsetup{width=0.75\textwidth}		
	\caption{Differential driveline pressure.}
	\label{}
	\end{subfigure}
\caption{Simulation results Test 14.1.}
\label{}
\end{figure}

\begin{figure}[H]
	\centering
	\begin{subfigure}{.5\textwidth}
	\centering
	\includegraphics[width = 240pt, angle = 0, keepaspectratio]{figures/hydrostatic_powertrain/validation/test_14_1_fig/HMM4_pressure_test_14_1.eps}
	\captionsetup{width=0.75\textwidth}		
	\caption{Hydro motor servo pressure.}
	\label{}
	\end{subfigure}%
	\begin{subfigure}{.5\textwidth}
	\centering
	\includegraphics[width = 240pt, angle = 0, keepaspectratio]{figures/hydrostatic_powertrain/validation/test_14_1_fig/p_supply_test_14_1.eps}
	\captionsetup{width=0.75\textwidth}		
	\caption{Charge pressure and fan driver pressure.}
	\label{}
	\end{subfigure}
\caption{Simulation results Test 14.1.}
\label{}
\end{figure}

\begin{figure}[H]
	\centering
	\begin{subfigure}{.5\textwidth}
	\centering
	\includegraphics[width = 240pt, angle = 0, keepaspectratio]{figures/hydrostatic_powertrain/validation/test_14_1_fig/vehicle_track_speed_test_14_1.eps}
	\captionsetup{width=0.75\textwidth}		
	\caption{Tracks speed.}
	\label{}
	\end{subfigure}%
	\begin{subfigure}{.5\textwidth}
	\centering
	\includegraphics[width = 240pt, angle = 0, keepaspectratio]{figures/hydrostatic_powertrain/validation/test_14_1_fig/volumetric_displacement_test_14_1.eps}
	\captionsetup{width=0.75\textwidth}		
	\caption{Volumetric displacement request and actuated}
	\label{}
	\end{subfigure}
\caption{Simulation results Test 14.1.}
\label{}
\end{figure}

\begin{figure}[H]
	\centering
	\includegraphics[width = 240pt, angle = 0, keepaspectratio]{figures/hydrostatic_powertrain/validation/test_14_1_fig/efficiency_test_14_1.eps}
	\captionsetup{width=0.75\textwidth}		
	\caption{Hydraulic efficiencies of pump and motor}
	\label{}
\end{figure}
\subsection{Lesson Learned from Test 14.1}
Here, same considerations for the Test~24.1.

\subsection{Test 12.1 - comparison}
In this test the vehicle lifted and was partly used to find the correct parametrization of the Turas travel gear.
For this test, which shows some minor misalignment, a quantitative comparison between experimental data and simulation about pump mechanical efficiency  is reported.
\begin{figure}[H]
	\centering
	\begin{subfigure}{.5\textwidth}
	\centering
	\includegraphics[width = 240pt, angle = 0, keepaspectratio]{figures/hydrostatic_powertrain/validation/test_12_1_fig/engine_quantities_test_12_1.eps}
	\captionsetup{width=0.75\textwidth}		
	\caption{Engine torque and speed.}
	\label{}
	\end{subfigure}%
	\begin{subfigure}{.5\textwidth}
	\centering
	\includegraphics[width = 240pt, angle = 0, keepaspectratio]{figures/hydrostatic_powertrain/validation/test_12_1_fig/abs_driveline_pressureA_test_12_1.eps}
	\captionsetup{width=0.75\textwidth}		
	\caption{Driveline pressure A.}
	\label{}
	\end{subfigure}
\caption{Simulation results Test 12.1.}
\label{}
\end{figure}

\begin{figure}[H]
	\centering
	\begin{subfigure}{.5\textwidth}
	\centering
	\includegraphics[width = 240pt, angle = 0, keepaspectratio]{figures/hydrostatic_powertrain/validation/test_12_1_fig/abs_driveline_pressureA_zoom_test_12_1.eps}
	\captionsetup{width=0.75\textwidth}		
	\caption{Dynamic behaviour - Driveline pressure A.}
	\label{}
	\end{subfigure}%
	\begin{subfigure}{.5\textwidth}
	\centering
	\includegraphics[width = 240pt, angle = 0, keepaspectratio]{figures/hydrostatic_powertrain/validation/test_12_1_fig/abs_driveline_pressureB_test_12_1.eps}
	\captionsetup{width=0.75\textwidth}		
	\caption{Driveline pressure B.}
	\label{}
	\end{subfigure}
\caption{Simulation results Test 12.1.}
\label{}
\end{figure}

\begin{figure}[H]
	\centering
	\begin{subfigure}{.5\textwidth}
	\centering
	\includegraphics[width = 240pt, angle = 0, keepaspectratio]{figures/hydrostatic_powertrain/validation/test_12_1_fig/abs_driveline_pressureB_zoom_test_12_1.eps}
	\captionsetup{width=0.75\textwidth}		
	\caption{Dynamic behaviour - Driveline pressure B.}
	\label{}
	\end{subfigure}%
	\begin{subfigure}{.5\textwidth}
	\centering
	\includegraphics[width = 240pt, angle = 0, keepaspectratio]{figures/hydrostatic_powertrain/validation/test_12_1_fig/diff_driveline_pressure_test_12_1.eps}
	\captionsetup{width=0.75\textwidth}		
	\caption{Differential driveline pressure.}
	\label{}
	\end{subfigure}
\caption{Simulation results Test 12.1.}
\label{}
\end{figure}

\begin{figure}[H]
	\centering
	\begin{subfigure}{.5\textwidth}
	\centering
	\includegraphics[width = 240pt, angle = 0, keepaspectratio]{figures/hydrostatic_powertrain/validation/test_12_1_fig/diff_driveline_pressure_zoom_test_12_1.eps}
	\captionsetup{width=0.75\textwidth}		
	\caption{Dynamic behaviour - Differential driveline pressure.}
	\label{}
	\end{subfigure}%
	\begin{subfigure}{.5\textwidth}
	\centering
	\includegraphics[width = 240pt, angle = 0, keepaspectratio]{figures/hydrostatic_powertrain/validation/test_12_1_fig/HMM4_pressure_test_12_1.eps}
	\captionsetup{width=0.75\textwidth}		
	\caption{Hydro motor servo pressure.}
	\label{}
	\end{subfigure}
\caption{Simulation results Test 12.1.}
\label{}
\end{figure}

\begin{figure}[H]
	\centering
	\begin{subfigure}{.5\textwidth}
	\centering
	\includegraphics[width = 240pt, angle = 0, keepaspectratio]{figures/hydrostatic_powertrain/validation/test_12_1_fig/HMM4_pressure_zoom_test_12_1.eps}
	\captionsetup{width=0.75\textwidth}		
	\caption{Dynamic behaviour - Hydro motor servo pressure.}
	\label{}
	\end{subfigure}%
	\begin{subfigure}{.5\textwidth}
	\centering
	\includegraphics[width = 240pt, angle = 0, keepaspectratio]{figures/hydrostatic_powertrain/validation/test_12_1_fig/p_supply_test_12_1.eps}
	\captionsetup{width=0.75\textwidth}		
	\caption{Charge pressure and fan driver pressure.}
	\label{}
	\end{subfigure}
\caption{Simulation results Test 12.1.}
\label{}
\end{figure}

\begin{figure}[H]
	\centering
	\begin{subfigure}{.5\textwidth}
	\centering
	\includegraphics[width = 240pt, angle = 0, keepaspectratio]{figures/hydrostatic_powertrain/validation/test_12_1_fig/vehicle_track_speed_test_12_1.eps}
	\captionsetup{width=0.75\textwidth}		
	\caption{Tracks speed.}
	\label{}
	\end{subfigure}%
	\begin{subfigure}{.5\textwidth}
	\centering
	\includegraphics[width = 240pt, angle = 0, keepaspectratio]{figures/hydrostatic_powertrain/validation/test_12_1_fig/vehicle_track_speed_zoom_test_12_1.eps}
	\captionsetup{width=0.75\textwidth}		
	\caption{Tracks speed - dynamic behaviour.}
	\label{}
	\end{subfigure}
\caption{Simulation results Test 12.1.}
\label{}
\end{figure}

\begin{figure}[H]
	\centering
	\begin{subfigure}{.5\textwidth}
	\centering
	\includegraphics[width = 240pt, angle = 0, keepaspectratio]{figures/hydrostatic_powertrain/validation/test_12_1_fig/volumetric_displacement_test_12_1.eps}
	\captionsetup{width=0.75\textwidth}		
	\caption{Volumetric displacement request and actuated}
	\label{}
	\end{subfigure}%
	\begin{subfigure}{.5\textwidth}
	\centering
	\includegraphics[width = 240pt, angle = 0, keepaspectratio]{figures/hydrostatic_powertrain/validation/test_12_1_fig/efficiency_test_12_1.eps}
	\captionsetup{width=0.75\textwidth}		
	\caption{Hydraulic efficiencies of pump and motor}
	\label{}
	\end{subfigure}
\caption{Simulation results Test 12.1.}
\label{}
\end{figure}

\subsection{Lesson Learned from Test 12.1}
Lifted case reports two misalignment cases:
\begin{itemize}
	\item Engine torque versus driveline differential pressure
	\item Driveline differential pressure dynamic
\end{itemize}
Concerning the engine torque misalignment, a detailed analysis is shown in the next section, even if, this effect is visible only in few tests and is not always reproducible.

Concerning the dynamic behaviour of the driveline pressure an additional description is available in Figure~\ref{DBA}. The misalignment could be due to some missing (maybe second order) modelization in the track vehicle.
\begin{figure}[H]
	\centering
	\includegraphics[width = 340pt, angle = 0, keepaspectratio]{figures/hydrostatic_powertrain/validation/test_12_1_fig/abs_driveline_pressureA_DBA_test_12_1.eps}
	\captionsetup{width=0.75\textwidth}		
	\caption{Dynamic behaviour gap between model and real vehicle in the condition of lifted case.}
	\label{DBA}
\end{figure}

\subsection{Pump efficiency comparison}\label{pump_efficiency_comparison}
In this section the misalignment between experimental data and simulation regarding the engine torque is going to be investigated.

The relation between drive-line pressure and engine torque (we suppose balanced load among both drivelines) is as follows
\begin{equation}\label{base_pump_equation}
	\tau_e = 2\ \frac{\Delta p\,V_d^p}{2\pi\ \eta_m^p}
\end{equation}
\begin{equation}\label{base_pump_eff_equation}
	\eta_m^p = 2\ \frac{\Delta p\,V_d^p}{2\pi\ \tau_e}
\end{equation}
starting from the above equations we are going to consider two different operating points (see also figure~\ref{operating_points}). From these working points we extrapolate, from experimental data, the mechanical efficiency of the pump $\eta_m^p$ and we compare it with one used for the simulation. For this calculation we consider a torque drop due to auxiliary devices (supply pumps and fan) of $\tau_e^{drop}=\SI{25}{\newton\meter}$. 
\subsubsection*{Results from experimental data - operating point A}
\begin{flalign*}
	&V_d^p = \SI{147.2}{\cubic\centi\meter\per\second}  && \\[8pt]
	&\Delta p = \SI{22.5}{\bar}  && \\[8pt]
	&\tau_e = \SI{135}{\newton\meter}  &&
\end{flalign*}
applying eq.~\ref{base_pump_eff_equation} we can estimate the efficiency of the pump. Supposing $\SI{25}{\newton\meter}$ lost for the supply pump as well as fan cooling circuit, we obtain the following pump mechanical efficiency
\begin{flalign*}
	\hat{\eta}_m^p = \frac{\Delta p\,V_d^p}{\pi\ \left( \tau_e - \tau_e^{drop}\right) }  = \SI{0.96}{} &&
\end{flalign*}
\subsubsection*{Results from simulation data - operating point A}
\begin{flalign*}
	&V_d^p = \SI{147.2}{\cubic\centi\meter\per\second}  && \\[8pt]
	&\Delta p = \SI{20.0}{\bar}  && \\[8pt]
	&\tau_e = \SI{180}{\newton\meter}  && \\[8pt]
	&\eta_m^p = \SI{0.615}{}  &&
\end{flalign*}
applying eq.~\ref{base_pump_equation} we obtain the following data:
\begin{flalign*}
	&\hat{\tau}_e = 2\frac{\Delta p\,V_d^p}{\eta_m^p 2\pi} = \SI{152}{\newton\meter}  && \\[8pt]
	&\tau_e^{drop} = \tau_e - \hat{\tau}_e =  \SI{28}{\newton\meter}  &&
\end{flalign*}
which represents the load due to supply pumps and fan cooling.
\subsubsection*{Results from experimental data - operating point B}
\begin{flalign*}
	&V_d^p = \SI{147.2}{\cubic\centi\meter\per\second}  && \\[8pt]
	&\Delta p = \SI{42.5}{\bar}  && \\[8pt]
	&\tau_e = \SI{230}{\newton\meter}  &&
\end{flalign*}
applying eq.~\ref{base_pump_eff_equation} we can estimate the efficiency of the pump. Supposing $\SI{25}{\newton\meter}$ lost for the supply pump as well as fan cooling circuit, we obtain the following pump mechanical efficiency
\begin{flalign*}
	\hat{\eta}_m^p = \frac{\Delta p\,V_d^p}{\pi\ \left( \tau_e - \tau_e^{drop}\right)}  = \SI{0.97}{} &&
\end{flalign*}
\subsubsection*{Results from simulation data - operating point B}
\begin{flalign*}
	&V_d^p = \SI{147.2}{\cubic\centi\meter\per\second}  && \\[8pt]
	&\Delta p = \SI{47.3}{\bar}  && \\[8pt]
	&\tau_e = \SI{320}{\newton\meter}  && \\[8pt]
	&\eta_m^p = \SI{0.76}{}  &&
\end{flalign*}
applying eq.~\ref{base_pump_equation} we obtain the following data:
\begin{flalign*}
	&\hat{\tau}_e = 2\frac{\Delta p\,V_d^p}{\eta_m^p 2\pi} = \SI{292}{\newton\meter}  && \\[8pt]
	&\tau_e^{drop} = \tau_e - \hat{\tau}_e =  \SI{28}{\newton\meter}  &&
\end{flalign*}
which represents the load due to supply pumps and fan cooling.
\begin{figure}[H]
	\centering
	\includegraphics[width = 500pt, angle = 0, keepaspectratio]{figures/hydrostatic_powertrain/validation/test_12_1_fig/pump_eff_comp.eps}
	\captionsetup{width=0.75\textwidth}		
	\caption{Engine vs driveline pressure comparison - Test~12.1}
	\label{operating_points}
\end{figure}
\begin{figure}[H]
	\centering
	\begin{subfigure}{.5\textwidth}
	\centering
	\includegraphics[width = 240pt, angle = 0, keepaspectratio]{figures/hydrostatic_powertrain/validation/test_12_1_fig/pump_eff_comp_2.eps}
	\captionsetup{width=0.75\textwidth}		
	\caption{Efficiencies - Test~12.1}
	\label{}
	\end{subfigure}%
	\begin{subfigure}{.5\textwidth}
	\centering
	\includegraphics[width = 240pt, angle = 0, keepaspectratio]{figures/hydrostatic_powertrain/validation/test_12_1_fig/pump_eff_comp_3.eps}
	\captionsetup{width=0.75\textwidth}		
	\caption{Volumetric displacement - Test~12.1}
	\label{}
	\end{subfigure}
\caption{Simulation results Test 12.1.}
\label{}
\end{figure}

\subsection{List of parameters used for the simulation}
In this section the list of parameters used in the simulations is reported. Considering the two case: laid on ground and lifted from ground, for each case the set of parameters have not been modified. Between laid on ground and lifted the following parameters have been changed.
\begin{itemize}
	\item vehicle weight
	\item ground viscosity
	\item motor displacement calibration
\end{itemize}
\vspace{10mm}
\texttt{\underline{EDC Danfoss H1P147}
	\begin{itemize}
		\item Servo-actuator parameters
		\begin{itemize}
			\item Compressibility effects: \textit{checked}
			\item Piston area chamber A $=\SI{0.0029235}{\square\meter}$
			\item Piston area chamber B $=\SI{0.0029235}{\square\meter}$
			\item Piston stroke (both ways)	 $=\SI{0.03647}{\meter}$ 	
			\item Spring coefficient $=\SI{40.9e3}{\newton\per\meter}$			
			\item Spring damping $=\SI{1000}{\newton\per\meter\second}$			
			\item Piston mass $=\SI{1}{\kilogram}$
		\end{itemize}
		\item Valve parameters
		\begin{itemize}
			\item Number of holes per orifice $= \SI{1}{}$
			\item P-A. Maximum orifice opening $= \SI{0.0012}{\meter}$
			\item P-B. Maximum orifice opening $= \SI{0.0012}{\meter}$
			\item T-A. Maximum orifice opening $= \SI{0.0012}{\meter}$
			\item T-B. Maximum orifice opening $= \SI{0.0012}{\meter}$
			\item Orifice opening offset P-A $=\SI{1.9e-4}{\meter}$
			\item Orifice opening offset P-B $=\SI{1.9e-4}{\meter}$
			\item Orifice opening offset T-A $=\SI{1.9e-4}{\meter}$
			\item Orifice opening offset T-B $=\SI{1.9e-4}{\meter}$
			\item Leakage area $=\SI{1e-8}{\square\meter}$
		\end{itemize}
		\item Spool control parameters
		\begin{itemize}
			\item Proportional gain $=1.014$
			\item Integral gain $=0.001$			
			\item Scaling factor $=\SI{5e-4}{}$
			\item Lag time constant (coil) $=\SI{0.009859}{\second}$
		\end{itemize}
		\item Swash-plate parameters
		\begin{itemize}
			\item Maximum angular displacement $=\SI{18}{\deg}$
		\end{itemize}
	\end{itemize}
}

\vspace{10mm}
\texttt{\underline{EDC Danfoss H1B250}
	\begin{itemize}
		\item Internal low pass filter
		\begin{itemize}
			\item Lag constant $=\SI{0.01}{\second}$
		\end{itemize}
		\item Servo-actuator parameters
		\begin{itemize}
			\item Compressibility effects: \textit{checked}
			\item Piston area chamber A $=\SI{0.0008}{\square\meter}$
			\item Piston area chamber B $=\SI{0.000425}{\square\meter}$
			\item Piston stroke	 $=\SI{0.06}{\meter}$ 	
			\item Spring coefficient $=\SI{800}{\newton\per\meter}$			
			\item Spring damping $=\SI{2000}{\newton\per\meter\second}$
			\item Viscous coefficient $=\SI{2000}{\newton\per\meter\second}$				
			\item Piston mass $=\SI{1}{\kilogram}$
		\end{itemize}
		\item 3-2 way control valve parameters
		\begin{itemize}		
			\item  Max.~orifice area P-A $=\SI{1.6e-5}{\square\meter}$
			\item  Max.~orifice opening P-A $=\SI{0.004}{\meter}$
			\item  Max.~orifice area T-A $=\SI{1e-6}{\square\meter}$
			\item  Max.~orifice opening T-A $=\SI{0.001}{\meter}$
			\item Leakage area $=\SI{1e-8}{\square\meter}$
		\end{itemize}
		\item 3-2 way high pressure cut-off valve parameters
		\begin{itemize}		
			\item  T-A. Passage maximum area $=\SI{1.44e-6}{\square\meter}$
			\item  T-A. Maximum orifice opening $=\SI{0.0012}{\meter}$
			\item  P-A. Passage maximum area $=\SI{1.44e-6}{\square\meter}$
			\item  P-A. Maximum orifice opening $=\SI{0.0012}{\meter}$
			\item  Setting pressure $=\SI{390}{\bar}$
			\item  Maximum pressure $=\SI{450}{\bar}$
			\item Leakage area $=\SI{1e-8}{\square\meter}$
		\end{itemize}
		\item Shuttle Valve (input pressure filtered at $=\SI{0.1}{\second}$ )
		\begin{itemize}		
			\item  Maximum valve orifice $=\SI{12e-3}{\meter}$
			\item  Valve cracking pressure $=\SI{1}{\bar}$
			\item  Valve maximum opening pressure $=\SI{6.25}{\bar}$
			\item Leakage area $=\SI{1e-8}{\square\meter}$
		\end{itemize}
		\item Spool control parameters
		\begin{itemize}
			\item Proportional gain $=0.75$
			\item Integral gain $=0$			
			\item Scaling factor $=\SI{-5e-4}{}$
			\item Lag time constant (coil) $=\SI{0.01}{\second}$
		\end{itemize}
		\item Swash-plate parameters
		\begin{itemize}
			\item Maximum angular displacement $=\SI{32}{\deg}$
			\item Minimum angular displacement $=\SI{6}{\deg}$
			\item Maximum volumetric displacement $=\SI{100}{}$
			\item Minimum volumetric displacement $=\SI{20}{}$
		\end{itemize}
		\item Flushing valve parameters
		\begin{itemize}
			\item Maximum spool valve opening $=\SI{0.012}{\meter}$
			\item Setting pressure $=\SI{6.5}{\bar}$
			\item Maximum pressure $=\SI{80}{\bar}$
			\item Maximum loop flushing valve opening $=\SI{2.7}{\meter}$
			\item Setting pressure loop flushing valve $=\SI{16}{\bar}$
			\item Regulation pressure loop flushing valve $=\SI{3}{\bar}$
			\item Leakage area $=\SI{1e-8}{\square\meter}$
			\item Leakage area loop flushing valve $=\SI{1e-8}{\square\meter}$
		\end{itemize}
	\end{itemize}
}

\vspace{10mm}
\texttt{\underline{Pressure Relief Valve Setting for Boost Pressure Regulation at $=\SI{32}{\bar}$}
	\begin{itemize}
		\item Valve parameters
		\begin{itemize}
			\item Non-Filtered model
			\item Valve maximum orifice $= \SI{0.010}{\meter}$
			\item Valve setting pressure $=\SI{24}{\bar}$			
			\item Valve regulation pressure $=\SI{70}{\bar}$
			\item Leakage area $=\SI{1e-8}{\square\meter}$
		\end{itemize}
	\end{itemize}
}

\vspace{10mm}
\texttt{\underline{Boost to Driveline Check Valve Setting}
	\begin{itemize}
		\item Valve parameters
		\begin{itemize}
			\item Filtered model
			\item Valve maximum orifice $= \SI{0.012}{\meter}$
			\item Valve cracking pressure $=\SI{1}{\bar}$			
			\item Valve maximum opening pressure $=\SI{6.25}{\bar}$
			\item Leakage area $=\SI{2e-8}{\square\meter}$
		\end{itemize}
	\end{itemize}
}

\vspace{10mm}
\texttt{\underline{Driveline to Boost Pressure Relief Valve Setting}
	\begin{itemize}
		\item Valve parameters
		\begin{itemize}
			\item Filtered model
			\item Valve maximum orifice $= \SI{0.012}{\meter}$
			\item Valve setting pressure $=\SI{420}{\bar}$			
			\item Valve regulation pressure $=\SI{30}{\bar}$
			\item Leakage area $=\SI{1e-8}{\square\meter}$
		\end{itemize}
	\end{itemize}
}

\vspace{10mm}
\texttt{\underline{Travel Gear Parameters Setting}
	\begin{itemize}
		\item Internal Friction
		\begin{itemize}
			\item Driver viscous friction $= \SI{0.06}{\newton\meter\second\per\radian}$
			\item Follower viscous friction $=\SI{2}{\newton\meter\second\per\radian}$			
			\item Meshing efficiency $=\SI{0.99}{}$
		\end{itemize}
	\end{itemize}
}

\texttt{\underline{hydrostatic-vehicle}
	\begin{itemize}
		\item Vehicle lifted from ground
		\begin{itemize}
			\item Vehicle mass $= \SI{230}{\kilogram}$
			\item Static friction coefficient $=\SI{0.005}{}$			
			\item Coulomb friction coefficient $=\SI{0.015}{}$
			\item Viscous friction coefficient $=\SI{100}{\newton\second\per\meter}$
			\item Breakaway friction velocity $=\SI{0.4}{\meter\per\second}$
		\end{itemize}
		\item Vehicle on ground
		\begin{itemize}
			\item Vehicle mass $= \SI{23000}{\kilogram}$
			\item Static friction coefficient $=\SI{0.005}{}$			
			\item Coulomb friction coefficient $=\SI{0.015}{}$
			\item Viscous friction coefficient $=\SI{2000}{\newton\second\per\meter}$
			\item Breakaway friction velocity $=\SI{0.4}{\meter\per\second}$
		\end{itemize}
	\end{itemize}
}
\vspace{10mm}
\texttt{\underline{Input data and others}
	\begin{itemize}
		\item Pipes are filtered with $\tau = \SI{0.25}{\second}$
		\item Input are filtered with an average filter of $\tau = \SI{0.25}{\second}$
	\end{itemize}
}
\vspace{10mm}
\texttt{\underline{Control Current vs Hydraulic Motor Displacement Transformation}
	\begin{itemize}
		\item Vehicle lifted from ground
		\begin{itemize}
			\item Current for maximum displacement (b) $= \SI{243}{\milli\ampere}$
			\item Current for minimum displacement (a) $=\SI{572}{\milli\ampere}$			
		\end{itemize}
		\item Vehicle on ground
		\begin{itemize}
			\item Current for maximum displacement (b) $= \SI{243}{\milli\ampere}$
			\item Current for minimum displacement (a) $=\SI{584}{\milli\ampere}$			
		\end{itemize}
	\end{itemize}
}
\vspace{10mm}
\texttt{\underline{Control Current vs Hydraulic Pump Displacement Transformation}
	\begin{itemize}
		\item Vehicle lifted from ground
		\begin{itemize}
			\item Current for minimum displacement (a) $= \SI{330}{\milli\ampere}$
			\item Current for maximum displacement (b) $=\SI{812}{\milli\ampere}$			
		\end{itemize}
		\item Vehicle on ground
		\begin{itemize}
			\item Current for minimum displacement (a) $= \SI{330}{\milli\ampere}$
			\item Current for maximum displacement (b) $=\SI{812}{\milli\ampere}$			
		\end{itemize}
	\end{itemize}
}

\subsection{Additional comparison}
In the following two sections, two additional comparison are presented: they concern the lifted case. In these context the issues present in the Test~12.1 are a little smoothed: the engine torque from experimental data matches the one obtained from simulation results in most of the operative conditions. The few points where the misalignment occur is for very low differential pressure, around $\SI{20}{\bar}$.  Still some gap between the dynamic behaviour of the driveline differential pressure.

\subsection{Test 1.1}
Test~1.1 results: vehicle is lifted from ground.
\begin{figure}[H]
	\centering
	\begin{subfigure}{.5\textwidth}
	\centering
	\includegraphics[width = 240pt, angle = 0, keepaspectratio]{figures/hydrostatic_powertrain/validation/test_1_1_fig/engine_quantities_test_1_1.eps}
	\captionsetup{width=0.75\textwidth}		
	\caption{Engine torque and speed.}
	\label{}
	\end{subfigure}%
	\begin{subfigure}{.5\textwidth}
	\centering
	\includegraphics[width = 240pt, angle = 0, keepaspectratio]{figures/hydrostatic_powertrain/validation/test_1_1_fig/abs_driveline_pressureA_test_1_1.eps}
	\captionsetup{width=0.75\textwidth}		
	\caption{Driveline pressure A.}
	\label{}
	\end{subfigure}
\caption{Simulation results Test 1.1.}
\label{}
\end{figure}

\begin{figure}[H]
	\centering
	\begin{subfigure}{.5\textwidth}
	\centering
	\includegraphics[width = 240pt, angle = 0, keepaspectratio]{figures/hydrostatic_powertrain/validation/test_1_1_fig/abs_driveline_pressureA_zoom_test_1_1.eps}
	\captionsetup{width=0.75\textwidth}		
	\caption{Dynamic behaviour - Driveline pressure A.}
	\label{}
	\end{subfigure}%
	\begin{subfigure}{.5\textwidth}
	\centering
	\includegraphics[width = 240pt, angle = 0, keepaspectratio]{figures/hydrostatic_powertrain/validation/test_1_1_fig/abs_driveline_pressureB_test_1_1.eps}
	\captionsetup{width=0.75\textwidth}		
	\caption{Driveline pressure B.}
	\label{}
	\end{subfigure}
\caption{Simulation results Test 1.1.}
\label{}
\end{figure}

\begin{figure}[H]
	\centering
	\begin{subfigure}{.5\textwidth}
	\centering
	\includegraphics[width = 240pt, angle = 0, keepaspectratio]{figures/hydrostatic_powertrain/validation/test_1_1_fig/abs_driveline_pressureB_zoom_test_1_1.eps}
	\captionsetup{width=0.75\textwidth}		
	\caption{Dynamic behaviour - Driveline pressure B.}
	\label{}
	\end{subfigure}%
	\begin{subfigure}{.5\textwidth}
	\centering
	\includegraphics[width = 240pt, angle = 0, keepaspectratio]{figures/hydrostatic_powertrain/validation/test_1_1_fig/diff_driveline_pressure_test_1_1.eps}
	\captionsetup{width=0.75\textwidth}		
	\caption{Differential driveline pressure.}
	\label{}
	\end{subfigure}
\caption{Simulation results Test 1.1.}
\label{}
\end{figure}

\begin{figure}[H]
	\centering
	\begin{subfigure}{.5\textwidth}
	\centering
	\includegraphics[width = 240pt, angle = 0, keepaspectratio]{figures/hydrostatic_powertrain/validation/test_1_1_fig/diff_driveline_pressure_zoom_test_1_1.eps}
	\captionsetup{width=0.75\textwidth}		
	\caption{Dynamic behaviour - Differential driveline pressure.}
	\label{}
	\end{subfigure}%
	\begin{subfigure}{.5\textwidth}
	\centering
	\includegraphics[width = 240pt, angle = 0, keepaspectratio]{figures/hydrostatic_powertrain/validation/test_1_1_fig/HMM4_pressure_test_1_1.eps}
	\captionsetup{width=0.75\textwidth}		
	\caption{Hydro motor servo pressure.}
	\label{}
	\end{subfigure}
\caption{Simulation results Test 1.1.}
\label{}
\end{figure}

\begin{figure}[H]
	\centering
	\begin{subfigure}{.5\textwidth}
	\centering
	\includegraphics[width = 240pt, angle = 0, keepaspectratio]{figures/hydrostatic_powertrain/validation/test_1_1_fig/HMM4_pressure_zoom_test_1_1.eps}
	\captionsetup{width=0.75\textwidth}		
	\caption{Dynamic behaviour - Hydro motor servo pressure.}
	\label{}
	\end{subfigure}%
	\begin{subfigure}{.5\textwidth}
	\centering
	\includegraphics[width = 240pt, angle = 0, keepaspectratio]{figures/hydrostatic_powertrain/validation/test_1_1_fig/p_supply_test_1_1.eps}
	\captionsetup{width=0.75\textwidth}		
	\caption{Charge pressure and fan driver pressure.}
	\label{}
	\end{subfigure}
\caption{Simulation results Test 1.1.}
\label{}
\end{figure}

\begin{figure}[H]
	\centering
	\begin{subfigure}{.5\textwidth}
	\centering
	\includegraphics[width = 240pt, angle = 0, keepaspectratio]{figures/hydrostatic_powertrain/validation/test_1_1_fig/vehicle_track_speed_test_1_1.eps}
	\captionsetup{width=0.75\textwidth}		
	\caption{Tracks speed.}
	\label{}
	\end{subfigure}%
	\begin{subfigure}{.5\textwidth}
	\centering
	\includegraphics[width = 240pt, angle = 0, keepaspectratio]{figures/hydrostatic_powertrain/validation/test_1_1_fig/vehicle_track_speed_zoom_test_1_1.eps}
	\captionsetup{width=0.75\textwidth}		
	\caption{Tracks speed - dynamic behaviour.}
	\label{}
	\end{subfigure}
\caption{Simulation results Test 1.1.}
\label{}
\end{figure}

\begin{figure}[H]
	\centering
	\begin{subfigure}{.5\textwidth}
	\centering
	\includegraphics[width = 240pt, angle = 0, keepaspectratio]{figures/hydrostatic_powertrain/validation/test_1_1_fig/volumetric_displacement_test_1_1.eps}
	\captionsetup{width=0.75\textwidth}		
	\caption{Volumetric displacement request and actuated}
	\label{}
	\end{subfigure}%
	\begin{subfigure}{.5\textwidth}
	\centering
	\includegraphics[width = 240pt, angle = 0, keepaspectratio]{figures/hydrostatic_powertrain/validation/test_1_1_fig/efficiency_test_1_1.eps}
	\captionsetup{width=0.75\textwidth}		
	\caption{Hydraulic efficiencies of pump and motor}
	\label{}
	\end{subfigure}
\caption{Simulation results Test 1.1.}
\label{}
\end{figure}

\subsection{Test 1.2}
Test~1.2 results: vehicle is lifted from ground.
\begin{figure}[H]
	\centering
	\begin{subfigure}{.5\textwidth}
	\centering
	\includegraphics[width = 240pt, angle = 0, keepaspectratio]{figures/hydrostatic_powertrain/validation/test_1_2_fig/engine_quantities_test_1_2.eps}
	\captionsetup{width=0.75\textwidth}		
	\caption{Engine torque and speed.}
	\label{}
	\end{subfigure}%
	\begin{subfigure}{.5\textwidth}
	\centering
	\includegraphics[width = 240pt, angle = 0, keepaspectratio]{figures/hydrostatic_powertrain/validation/test_1_2_fig/abs_driveline_pressureA_test_1_2.eps}
	\captionsetup{width=0.75\textwidth}		
	\caption{Driveline pressure A.}
	\label{}
	\end{subfigure}
\caption{Simulation results Test 1.2.}
\label{}
\end{figure}

\begin{figure}[H]
	\centering
	\begin{subfigure}{.5\textwidth}
	\centering
	\includegraphics[width = 240pt, angle = 0, keepaspectratio]{figures/hydrostatic_powertrain/validation/test_1_2_fig/abs_driveline_pressureA_zoom_test_1_2.eps}
	\captionsetup{width=0.75\textwidth}		
	\caption{Dynamic behaviour - Driveline pressure A.}
	\label{}
	\end{subfigure}%
	\begin{subfigure}{.5\textwidth}
	\centering
	\includegraphics[width = 240pt, angle = 0, keepaspectratio]{figures/hydrostatic_powertrain/validation/test_1_2_fig/abs_driveline_pressureB_test_1_2.eps}
	\captionsetup{width=0.75\textwidth}		
	\caption{Driveline pressure B.}
	\label{}
	\end{subfigure}
\caption{Simulation results Test 1.2.}
\label{}
\end{figure}

\begin{figure}[H]
	\centering
	\begin{subfigure}{.5\textwidth}
	\centering
	\includegraphics[width = 240pt, angle = 0, keepaspectratio]{figures/hydrostatic_powertrain/validation/test_1_2_fig/abs_driveline_pressureB_zoom_test_1_2.eps}
	\captionsetup{width=0.75\textwidth}		
	\caption{Dynamic behaviour - Driveline pressure B.}
	\label{}
	\end{subfigure}%
	\begin{subfigure}{.5\textwidth}
	\centering
	\includegraphics[width = 240pt, angle = 0, keepaspectratio]{figures/hydrostatic_powertrain/validation/test_1_2_fig/diff_driveline_pressure_test_1_2.eps}
	\captionsetup{width=0.75\textwidth}		
	\caption{Differential driveline pressure.}
	\label{}
	\end{subfigure}
\caption{Simulation results Test 1.2.}
\label{}
\end{figure}

\begin{figure}[H]
	\centering
	\begin{subfigure}{.5\textwidth}
	\centering
	\includegraphics[width = 240pt, angle = 0, keepaspectratio]{figures/hydrostatic_powertrain/validation/test_1_2_fig/diff_driveline_pressure_zoom_test_1_2.eps}
	\captionsetup{width=0.75\textwidth}		
	\caption{Dynamic behaviour - Differential driveline pressure.}
	\label{}
	\end{subfigure}%
	\begin{subfigure}{.5\textwidth}
	\centering
	\includegraphics[width = 240pt, angle = 0, keepaspectratio]{figures/hydrostatic_powertrain/validation/test_1_2_fig/HMM4_pressure_test_1_2.eps}
	\captionsetup{width=0.75\textwidth}		
	\caption{Hydro motor servo pressure.}
	\label{}
	\end{subfigure}
\caption{Simulation results Test 1.2.}
\label{}
\end{figure}

\begin{figure}[H]
	\centering
	\begin{subfigure}{.5\textwidth}
	\centering
	\includegraphics[width = 240pt, angle = 0, keepaspectratio]{figures/hydrostatic_powertrain/validation/test_1_2_fig/HMM4_pressure_zoom_test_1_2.eps}
	\captionsetup{width=0.75\textwidth}		
	\caption{Dynamic behaviour - Hydro motor servo pressure.}
	\label{}
	\end{subfigure}%
	\begin{subfigure}{.5\textwidth}
	\centering
	\includegraphics[width = 240pt, angle = 0, keepaspectratio]{figures/hydrostatic_powertrain/validation/test_1_2_fig/p_supply_test_1_2.eps}
	\captionsetup{width=0.75\textwidth}		
	\caption{Charge pressure and fan driver pressure.}
	\label{}
	\end{subfigure}
\caption{Simulation results Test 1.2.}
\label{}
\end{figure}

\begin{figure}[H]
	\centering
	\begin{subfigure}{.5\textwidth}
	\centering
	\includegraphics[width = 240pt, angle = 0, keepaspectratio]{figures/hydrostatic_powertrain/validation/test_1_2_fig/vehicle_track_speed_test_1_2.eps}
	\captionsetup{width=0.75\textwidth}		
	\caption{Tracks speed.}
	\label{}
	\end{subfigure}%
	\begin{subfigure}{.5\textwidth}
	\centering
	\includegraphics[width = 240pt, angle = 0, keepaspectratio]{figures/hydrostatic_powertrain/validation/test_1_2_fig/vehicle_track_speed_zoom_test_1_2.eps}
	\captionsetup{width=0.75\textwidth}		
	\caption{Tracks speed - dynamic behaviour.}
	\label{}
	\end{subfigure}
\caption{Simulation results Test 1.2.}
\label{}
\end{figure}

\begin{figure}[H]
	\centering
	\begin{subfigure}{.5\textwidth}
	\centering
	\includegraphics[width = 240pt, angle = 0, keepaspectratio]{figures/hydrostatic_powertrain/validation/test_1_2_fig/volumetric_displacement_test_1_2.eps}
	\captionsetup{width=0.75\textwidth}		
	\caption{Volumetric displacement request and actuated}
	\label{}
	\end{subfigure}%
	\begin{subfigure}{.5\textwidth}
	\centering
	\includegraphics[width = 240pt, angle = 0, keepaspectratio]{figures/hydrostatic_powertrain/validation/test_1_2_fig/efficiency_test_1_2.eps}
	\captionsetup{width=0.75\textwidth}		
	\caption{Hydraulic efficiencies of pump and motor}
	\label{}
	\end{subfigure}
\caption{Simulation results Test 1.2.}
\label{}
\end{figure}





\chapter{Hydrostatic servo actuator modelization and control}
In what follows we shell present how to model and implement an state feedback control for an hydro-static servo actuator. Figure~\ref{servo_desc_2b} shows the structure of the system, where the main components are as follows:
\begin{itemize}
	\item An hydraulic servo consisting of two chambers, one piston with symmetrical area coupled to a springs-damper system
	\item An hydraulic four ways valve
	\item An linear actuator based on permanent magnet which moves the spool of the hydraulic valve 
\end{itemize}
The whole system perform the hydraulic valve piston. The control input is the voltage $u_c(t)$ applied to the linear motor. The output, the quantity we want to control is the position of the piston. 

By the four-way valve, we can regulate the flow of the hydraulic fluid from a generic ideal pressure source $p_s$ toward the corresponding chamber (depending by the sign of valve position $x_v$). The hydraulic fluid flowing into the chamber will increase the internal pressure and when the force actuated by the delta pressure over the piston areas overcomes the spring force the piston will move. To stand still in a position the valve orifice must be closed back in order to bring to zero the flows $q_1(t)$ and $q_2(t)$. 

For our analysis we suppose that the spool of the valve is governed by an electromagnetic linear actuator based on a permanent magnet, this permit to us to correlate the actuated force as linear function of the coil current $f_v^m=k_f\,i_c$. The spool of the valve is connected to a coil (with $N_c$ turns) which is immersed in a constant magnetic field $B^M$ generated by a permanent magnet. When a current is applied to the electromagnetic circuit a force is generated by the iteration of current and magnetic field resulting in a movement of the spool of the valve.
\begin{figure}[H]
	\centering
	\includegraphics[width = 0.5\textwidth, width = 400pt, angle = 0, keepaspectratio]{figures/servo/fourway_valve_bichamber_piston_open_2.eps}
	\captionsetup{width=0.75\textwidth}		
	\caption{Four-way valve actuated piston during position changing.}
	\label{servo_desc_2b}
\end{figure}
\begin{figure}[H]
	\centering
	\includegraphics[width = 0.5\textwidth, width = 300pt, angle = 0, keepaspectratio]{figures/servo/spool_valve_actuator.eps}
	\captionsetup{width=0.75\textwidth}		
	\caption{Permanent magnet spool valve actuator.}
	\label{servo_desc_3}
\end{figure}

\section{Model derivation}

\subsection{Permanent magnet actuator}
In this section we are using the moving coil model (as shown in Figure~\ref{servo_desc_3}) as linear permanent magnet motor. This approximation, of a more complex electromechanical actuator, is used for didactic way because the physics behind the system is still the same for a real pm-linear motor. The moving bar is consisting of $N_c$ turns. The $N_c$ turns are immersed into a constant and isotropic magnetic field $\vec{B}^M$ which we suppose generated by a permanent magnet. When the moving coil is carrying a current $i_c(t)$ the moving coil is subject to a force. In real implementation a linear motor is, in general, fed by a sin-wave current, but its transformation into the spool reference frame (like DQ transformation for rotating system) brings the overall model into a system where the final force applied to the spool of the valve is proportional to a current component, resulting in the form $f_m^v(t)=k_f\,i_c(t)$, $f_m^v(t)$ is the force applied to the spool.

To modelize a linear permanent magnet actuator we can start writing the Faraday's law 
\begin{equation}\label{feynmann_1}
	\oint_\mathcal{C}\vec{E}\cdot d\vec{l} = -\frac{d}{dt}\int_\mathcal {S}\vec{B}\cdot\hat{n}\,da
\end{equation}
considering that the curve $\mathcal{C}$ and the surface $\mathcal{S}$ are fixed in space, the right term can also be written as follows
\begin{equation}\label{feynmann_2}
	\frac{d}{dt}\int_\mathcal {S}\vec{B}\cdot\hat{n}\,da = \frac{d}{dt}\psi(i_c,x_v)
\end{equation}
The term $\psi(i_c,x_v)$ is the total flux linked to the circuit which can also be written as
\begin{equation}\label{feynmann_3}
	\begin{aligned}
		\psi(i_c,x_v) = N_c\phi(i_c,x_v)
	\end{aligned}
\end{equation}
where $N_c$ is an equivalent number of turns. 

According to \cite{p6} \textit{the “flux rule”~—~that the emf in a circuit is equal to the rate of change of the magnetic flux through the circuit~—~applies whether the flux changes because the field changes or because the circuit moves} we can exploit the flux derivative as follows
\begin{equation}\label{feynmann_4}
	\begin{aligned}
		\frac{d}{dt}\phi(i_c,x_v)= \frac{\partial\phi(i_c,x_v)}{\partial i_c}\frac{d i_c(t)}{d t} + \frac{\partial\phi(i_c,x_v)}{\partial x_v}\frac{d x_v(t)}{d t}
	\end{aligned}
\end{equation}
the first term at the right side can be written as follows 
\begin{equation}\label{feynmann_5}
	\begin{aligned}
		\frac{\partial\phi(i_c,x_v)}{\partial x_v}\frac{d x_v(t)}{d t} = \frac{\partial {B}x_v h}{\partial x_v} v_v(t) \approx B^Mhv_v(t)
	\end{aligned}
\end{equation}
where $h$ is the height of the coil and we suppose the magnetic field due to the current $i_c(t)$ is negligible respect to ${B}^{\,M}$ which is the magnitude of the permanent magnet magnetic field and it is considered a constant parameter. 

The second term become 
\begin{equation}\label{feynmann_6}
	\begin{aligned}
		\frac{\partial\phi(i_c,x_v)}{\partial i_c}\frac{d i_c(t)}{d t} =L_c'(x_v) \frac{di_c(t)}{dt} \approx L_c'\frac{di_c(t)}{dt}
	\end{aligned}
\end{equation}
Turning back to Eq.~\eqref{feynmann_1}  we can write
\begin{equation}\label{}
	\begin{aligned}
		\underbrace{\int_{(-)}^{(+)}\vec{E}\cdot\,d\vec{l}}_{\text{source}} \, + \underbrace{\int\vec{E}\cdot\,d\vec{l}}_{\text{resistor}}\, = - N_cB^Mhv_v(t) - L_c\frac{di_c(t)}{dt}
	\end{aligned}
\end{equation} 
where $L_c = N_cL_c'$ and where
\begin{equation}\label{}
	-u_c = \underbrace{\int_{(-)}^{(+)}\vec{E}\cdot\, d\vec{l}}_{\text{source}}
\end{equation} 
\begin{equation}\label{}
	i_c\,R_c = \underbrace{\int\vec{E}\cdot\,d\vec{l}}_{\text{resistor}}
\end{equation} 
hence, we obtain the Kirchhoff's voltage law applied to the circuit as follows
\begin{equation}\label{linact_7}
	\begin{aligned}
		u_c(t) - R_ci_c(t)-L_c\frac{di_c(t)}{dt}-N_cB^Mhv_v(t)=0
	\end{aligned}
\end{equation}
To complete the linear pm-actuator model we must add the mechanical equations using the Newton's law. To apply the Newton's law we first must calculate the corresponding force actuated by the iteration between the current $i_c$ and the magnetic field $\vec{B}^M$.


The expression of the force due to the current $i_c$ in the magnetic field $\vec{B}^M$ is evaluated as follows 
\begin{equation}\label{}
	\vec{f}(t) = i(t)\int_{\text{wire}}d\vec{l}\times\vec{B}^{\text{ext}}
\end{equation}
where $\vec{B}^{\text{ext}}$ is an external field. Applying the integration of Eq.~\eqref{force} it results in the following equation (\textbf{integration path follows the current direction})
\begin{equation}\label{}
	\begin{aligned}
		f_v^m(t) = N_c\,i_c(t)\int_{h}^{0}d\vec{l}\times\vec{B}^M=N_c\,B^M\,h\,i_c(t)\quad \text{positive x-direction}
	\end{aligned}
\end{equation}
where $N_c$ is the number of turns linkage to the magnetic field.
The complete set of system equations becomes
\begin{equation}\label{}
	\left\lbrace \begin{aligned}
		\frac{dx_v(t)}{dt} &= v_v(t) \\[6pt]
		\frac{dv_v(t)}{dt} &= N_c\frac{B^Mh}{m_v} i_c(t)-\frac{b_v}{m_v} v_v(t)-\frac{1}{m_v}f^e_v(t) \\[6pt]
		\frac{di_c(t)}{dt} &= -\frac{R_c}{L_c}i_c(t)-\frac{1}{L_c}N_cB^Mhv_v(t)+\frac{1}{L_c}u_c(t)
	\end{aligned}\right. 
\end{equation}
the first two equations represent the dynamic motion of the bar where the force $f_v^m(t)=N_cB^Mh\,i(t)$ is generated by the interaction of the current $i_c(t)$ and the magnetic field $\vec{B}^M$. The third equation represents the Kirchhoff's voltage law of the electrical circuit which generates the current $i_c(t)$ by applying the voltage $u_c(t)$. The equivalent auto-inductance $L_c$ and the back-emf term, $N_cB^Mhv_v(t)$ represent the effect of the Faraday's law.

\subsection{Four-ways valve piston actuated}
As reported in Figure~\ref{servo_desc_2b} the system we are going to describe is made by the coupling of a four-way valve and a double chambers piston. For model representation we are considering the overall leakage and offset orifices to be zero, moreover we are supposing the system is symmetrical $A_1=A_2$, then, we can write the following set of continuity equations
\begin{equation}\label{continuity_1}
	\left\lbrace \begin{aligned}
		& q_1(t) = \frac{dV_1(t)}{dt} + \frac{V_1(t)}{\beta}\frac{dp_1(t)}{dt}\\[6pt]
		& -q_2(t) = \frac{dV_2(t)}{dt} + \frac{V_2(t)}{\beta}\frac{dp_2(t)}{dt}
	\end{aligned}\right. 
\end{equation} 
The volume of the piston chambers may be written as follows (we are assuming the piston is centered)
\begin{equation}\label{continuity_2}
	\left\lbrace \begin{aligned}
		& V_1(t) = V_0+A_1x_p(t)\\[6pt]
		& V_2(t) = V_0-A_2x_p(t)
	\end{aligned}\right. 
\end{equation} 
considering $A_1=A_2=A$ we obtain
\begin{equation}\label{continuity_3}
	\left\lbrace \begin{aligned}
		& V_1(t) = V_0+Ax_p(t)\\[6pt]
		& V_2(t) = V_0-Ax_p(t)
	\end{aligned}\right. 
\end{equation} 
and 
\begin{equation}\label{continuity_4}
	\left\lbrace \begin{aligned}
		& \frac{dV_1(t)}{dt} = Av_p(t)\\[6pt]
		& \frac{dV_2(t)}{dt} = -Av_p(t)
	\end{aligned}\right. 
\end{equation} 
which bring to the following state equations
\begin{equation}\label{piston_eq_1}
	\left\lbrace \begin{aligned}
		&\frac{dp_1(t)}{dt} = \beta\Bigg[\frac{q_1(t)-Av_p(t)}{V_0+Ax_p(t)}\Bigg]  \\[6pt]
		&\frac{dp_2(t)}{dt} = \beta\Bigg[\frac{Av_p(t) - q_2(t)}{V_0+A\big(-x_p(t)\big)}\Bigg] \\[6pt]
		&\frac{dv_p(t)}{dt} = \frac{1}{m_p}\Bigg[A\big(p_1(t)-p_2(t)\big)-b_pv_p(t)-k_px_p(t)-f^e_p(t)\Bigg] \\[6pt]
		&\frac{dx_p(t)}{dt} = v_p(t)
	\end{aligned}\right. 
\end{equation} 
where $\rho = \SI{850}{\kilogram\per\cubic\meter}$ is the density of the fluid, $\beta = \SI{1.2e9}{\pascal}$ is the bulk modulus, $x_v$ is the valve orifice where we suppose the orifice area is $x_v\cdot x_v^{\text{max}}$, $k_p$ is the spring coefficient, $b_p$ the damping coefficient and $m_p$ the mass of the piston. Like in most of the hydraulic system the ratio between force and mass is very high that means the inertia effects are in general less significant.

Due to compressibility effect the flow $q_1(t)$ and the flow $q_2(t)$ are different as consequence of the conservation of the mass flow between chamber~1 and chamber~2. 

The first two equations of Eq.~\eqref{piston_eq_1} represent the state equation of the pressure of the chambers. The flow $q_1(t)$ is function of the delta pressure between a orifice, hence we can write the following relations
\begin{equation}\label{piston_eq_2a}
	q_1(t) = \left\lbrace \begin{aligned}
		&x_vx_v^{\text{max}}\sqrt{\frac{2}{\rho}}\sqrt{|p_{s}-p_1|}\,\sign\big(p_{s}-p_1\big) \quad \text{for} \quad x_v > 0 \\[6pt]
		&x_vx_v^{\text{max}}\sqrt{\frac{2}{\rho}}\sqrt{|p_{1}-p_\text{tank}|}\,\sign\big(p_{1}-p_\text{tank}\big) \quad \text{for} \quad x_v < 0
	\end{aligned}\right. 
\end{equation}
while $q_2(t)$ is given the following relations
\begin{equation}\label{piston_eq_2b}
	q_2(t) = \left\lbrace \begin{aligned}
		&x_vx_v^{\text{max}}\sqrt{\frac{2}{\rho}}\sqrt{|p_{2}-p_\text{tank}|}\,\sign\big(p_{2}-p_\text{tank}\big) \quad \text{for} \quad x_v > 0 \\[6pt]
		&x_vx_v^{\text{max}}\sqrt{\frac{2}{\rho}}\sqrt{|p_{2}-p_s|}\,\sign\big(p_{s}-p_2\big) \quad \text{for} \quad x_v < 0
	\end{aligned}\right. 
\end{equation}
Moreover, according Figure~\ref{servo_desc_2b} we have to consider the following relations:
\begin{equation}
	x_v>0\Rightarrow\left\lbrace \begin{aligned}
		&q_s(t) = q_1(t) \\[6pt]
		&q_t(t) = q_2(t) \\[6pt]
	\end{aligned}\right. 
\end{equation}
\begin{equation}
	x_v<0\Rightarrow\left\lbrace \begin{aligned}
		&q_s(t) = -q_2(t) \\[6pt]
		&q_t(t) = -q_1(t) \\[6pt]
	\end{aligned}\right. 
\end{equation}
and
\begin{equation}
	x_v=0\Rightarrow\left\lbrace \begin{aligned}
		&q_s(t) = 0 \\[6pt]
		&q_t(t) = 0 \\[6pt]
	\end{aligned}\right. 
\end{equation}
The pressure source $p_s$ is generated by a pressure controlled pump. This model is not here described but is available in the released simscape library.   

\subsection{The whole model}
The final system is made the union between the pm-magnet actuator and the double chamber piston and it consists of seven ordinary differential equations, as follows
\begin{equation}\label{linact_10}
	\left\lbrace \begin{aligned}
		&\frac{dx_v(t)}{dt} = v_v(t) \\[6pt]
		&\frac{dv_v(t)}{dt} = N_c\frac{B^Mh}{m_v} i_c(t)-\frac{b}{m_v} v_v(t) \\[6pt]
		&\frac{di_c(t)}{dt} = -\frac{R_c}{L_c}i_c(t)-\frac{1}{L_c}N_cB^Mhv_v(t)+\frac{1}{L_c}u_c(t)\\[6pt]
		&\frac{dp_1(t)}{dt} = \beta\Bigg[\frac{q_1(t)-Av_p(t)}{V_0+Ax_p(t)}\Bigg]  \\[6pt]
		&\frac{dp_2(t)}{dt} = \beta\Bigg[\frac{Av_p(t) - q_2(t)}{V_0+A\big(-x_p(t)\big)}\Bigg] \\[6pt]
		&\frac{dv_p(t)}{dt} = \frac{1}{m_p}\Bigg[A\Big(p_1(t)-p_2(t)\Big)-b_pv_p(t)-k_px_p(t)-f^e_p(t)\Bigg] \\[6pt]
		&\frac{dx_p(t)}{dt} = v_p(t)
	\end{aligned}\right. 
\end{equation}
where 
\begin{equation}\label{}
	q_1(t) = \left\lbrace \begin{aligned}
		&x_vx_v^{\text{max}}\sqrt{\frac{2}{\rho}}\sqrt{|p_{s}-p_1|}\,\sign\big(p_{s}-p_1\big) \quad \text{for} \quad x_v > 0 \\[6pt]
		&x_vx_v^{\text{max}}\sqrt{\frac{2}{\rho}}\sqrt{|p_{1}-p_\text{tank}|}\,\sign\big(p_{1}-p_\text{tank}\big) \quad \text{for} \quad x_v < 0
	\end{aligned}\right. 
\end{equation}
\begin{equation}\label{}
	q_2(t) = \left\lbrace \begin{aligned}
		&x_vx_v^{\text{max}}\sqrt{\frac{2}{\rho}}\sqrt{|p_{2}-p_\text{tank}|}\,\sign\big(p_{2}-p_\text{tank}\big) \quad \text{for} \quad x_v > 0 \\[6pt]
		&x_vx_v^{\text{max}}\sqrt{\frac{2}{\rho}}\sqrt{|p_{2}-p_s|}\,\sign\big(p_{s}-p_2\big) \quad \text{for} \quad x_v < 0
	\end{aligned}\right. 
\end{equation}
the disturbance $$f^e_v(t)$$ applied to the spool of the valve has been omitted because is considered negligible, while the disturbance $$f^e_p(t)$$ applied to the piston is taken into account.

While the pm-actuator is represented by a liner time invariant system, the double chamber piston is a non-linear system. In order to implement a state feedback control strategy we have to linearize the model around a steady state point. Due to the complexity of the model, the resulting linearized model is still complex. A possible different approach could be to evaluate the time response of the system when is driven by a step input signal. This approach will be carried out in the next section.  

\subsubsection{System parameters}
\begin{itemize}
	\item Permanent magnet actuator
	\begin{itemize}
		\item $B^M = \SI{1}{\weber\per\square\meter}$
		\item $m_v = \SI{0.05}{\kilogram}$
		\item $L_c = \SI{10}{\milli\henry}$
		\item $R_c = \SI{10}{\ohm}$
		\item $b_v = \SI{2e3}{\newton\second\per\meter}$
		\item $h = \SI{0.05}{\meter}$
		\item $x_v^{\text{max}} = \SI{2.75}{\milli\meter}$
		\item $N_c = \SI{1000}{}$		
		\item $u_c^{\text{max}} = \SI{25}{\volt}$
	\end{itemize}
	\item Four-way valve bi-chambers piston with load
	\begin{itemize}
		\item $\beta = \SI{1.2e9}{\pascal}$
		\item $\rho = \SI{850}{\kilogram\per\cubic\meter}$
		\item $D_1 = \SI{0.1}{\meter}$
		\item $D_2 = \SI{0.1}{\meter}$
		\item $L_p = \SI{1}{\meter}$
		\item $m_p = \SI{25}{\kilogram}$
		\item $k_p = \SI{100}{\kilo\newton\meter}$
		\item $b_p = \SI{200}{\newton\second\per\meter}$
	\end{itemize}
\end{itemize}

\subsection{System approximation}
In the following section we try to find a way to represent the physical system with a simpler model which cover its main dynamic. Applying test signal to the input $u_c(t)$ we evaluate the behaviour of the output $x_p(t)$ in such a way to try to write down a possible approximated linear representation  of the form $X_p(s) = H_s(s)U_c(s)$ where $U_c(s)$ is the applied voltage in the Laplace domain and $X_p(s)$ is the piston position also in the Laplace domain.

In order to identify the system a set of different test signals are applied. As shown in the Figure~\ref{test_signals}, from the time response of the physical system we can approximate the seven order model by a simpler double integrator, hence the transfer function the system could be modelized as follows
\begin{equation}\label{linact_11}
	X_p(s) = k\frac{1}{s^2}U_c(s)
\end{equation}
where the value of the parameters $k$ can be determined experimentally.

Once the parameters $k$ have been identified the model representation is ended. In order to implement a state feedback control with a state observer the state space representation of the model must be taken into account.

The transfer function $$H(s) = k\frac{1}{s^2}$$ can be represented in state space form using the controllable canonical form as follows
\begin{equation}\label{ssr1}
	\tilde{\mathbf{A}} = \begin{bmatrix}
		0 & 1 \\[6pt]
		0 & 0
	\end{bmatrix} \qquad
	\tilde{\mathbf{B}} = \begin{bmatrix} 0\\[6pt] 1\end{bmatrix} \qquad
	\mathbf{C} = \begin{bmatrix}
		k & 0
	\end{bmatrix}
\end{equation}
Resulting in the following system
\begin{equation}\label{ssr2}
	\begin{aligned}
		\dot{\vec{x}}(t) &=\tilde{\mathbf{A}} \vec{x}(t) + \tilde{\mathbf{B}} u(t) \\[6pt]
		y(t) &= \mathbf{C}\vec{x}(t)
	\end{aligned}
\end{equation}
where $u(t)=u_c(t)$ and $y(t) = x_p(t)$.
\begin{figure}[H]
	\centering
	\includegraphics[width = 0.5\textwidth, width = 400pt, angle = 0, keepaspectratio]{figures/servo/test_signals_2.eps}
	\captionsetup{width=0.75\textwidth}		
	\caption{Description of the test signals used for system identification and linearization.}
	\label{test_signals}
\end{figure}
\section{Position control implementation}
In this laboratory course we want to show how to implement a controller in both time domains: continuous and discrete time. The discrete time control implementation will be done using a direct C-code using the C-caller function. The C-caller function permit to use the proper code implementation and can be a very useful strategy to generate a very robust code with minimum of bugs. As already mention the control system strategy is based on state feedback control and state observer as shown in Figure~\ref{control_layout_1} for the continuous time domain and as shown in Figure~\ref{control_layout_3} for the discrete time domain.
\begin{figure}[H]
	\centering
	\includegraphics[width = 0.5\textwidth, width = 450pt, keepaspectratio]{figures/servo/nlplant_state_fb_state_obs_ct.eps}
	\captionsetup{width=0.75\textwidth}		
	\caption{Control system layout based on state feedback and state observer for the continuous time domain.}
	\label{control_layout_1}
\end{figure}
In the next sections we will present the design of these controllers.
\begin{figure}[H]
	\centering
	\includegraphics[width = 0.5\textwidth, width = 450pt, keepaspectratio]{figures/servo/nlplant_state_fb_state_obs_dt.eps}
	\captionsetup{width=0.75\textwidth}		
	\caption{Control system layout based on state feedback and state observer for the discrete time domain.}
	\label{control_layout_3}
\end{figure}

\subsection{Continuous-time domain}
The first implementation will be carried out in continuous time domain. This is a good method to verify the feasibility of our control strategy. The control strategy layout is reported in Figure~\ref{control_layout_2} where the physical system is represented with its linear approximation. The linear approximation is only used to design the controller. Once the controller is designed it will be implemented in the full-order system shown in Figure~\ref{control_layout_1}.

Pole placement problem can be solved easily with \textbf{Matlab} by two commands: \texttt{acker()} or \texttt{place()} for the computation of the feedback-gain matrix $\mathbf{K}$. The command \texttt{acker()} is based on Ackermann's formula. This command applies to single input system only. The desired closed-loop poles can include multiple poles (pole located at the same place). 

If the system involves multiple inputs, for a specified set of closed-loop poles the state-feedback gain matrix $\mathbf{K}$ is not unique and we have an additional freedom to choose $\mathbf{K}$. There are many approaches to constructively utilize this additional freedom to determine $\mathbf{K}$. One common use is to maximize the stability margin. The pole placement based on this approach is called the robust pole placement. The matlab command for the robust pole placement is \texttt{place()}. 

Although the command \texttt{place} can be used for both single-input and multiple-input systems, this command requires that the multiplicity of poles in the desired closed-loop poles be no greater than the rank of $\mathbf{B}$. That is, if matrix $\mathbf{B}$ is an $n\times 1$ matrix, the command \texttt{place()} requires that there be no multiple poles in the set of desired close-loop poles.

For single input systems, the commands \texttt{acker()} and \texttt{place()} yield the same $\mathbf{K}$.

To use the command \texttt{acker}, we first enter the following matrices 
\begin{equation*}
	\mathbf{A}\qquad\mathbf{B}\qquad\mathbf{P}_{sf}
\end{equation*}
where $\mathbf{P}_{sf}$ is the matrix consisting of the desired closed-loop poles such that 
\begin{equation*}
	\mathbf{P}_{sf}=\begin{bmatrix}p_1^{sf} & p_2^{sf} & \cdots & p_n^{sf}\end{bmatrix}
\end{equation*}
The matrix $\mathbf{K}$ can be derived by the command
$$\texttt{K = acker(A, B, Psf);}$$

Once the feedback-gain matrix $\mathbf{K}$ has been design, the scaling matrix $N$ shall be calculated.
\begin{figure}[H]
	\centering
	\includegraphics[width = 0.5\textwidth, width = 400pt, keepaspectratio]{figures/servo/ct_state_fb_ctrl_theory.eps}
	\captionsetup{width=0.75\textwidth}		
	\caption{State feedback control.}
	\label{Statefeedbackcontrol}
\end{figure}
Considering Figure~\ref{Statefeedbackcontrol} we can write the following equation
\begin{equation}\label{stfb_1}
	\dot{\vec{x}}(t) = \tilde{\mathbf{A}}\vec{x}(t) + \tilde{\mathbf{B}}\Big(Nr(t)-\mathbf{K}\vec{x}(t)\Big)
\end{equation}
In steady state condition we can suppose $\dot{\vec{x}}(t)=0$ and Eq.~\eqref{stfb_1} becomes
\begin{equation}\label{stfb_2}
	\begin{aligned}
		0 = \tilde{\mathbf{A}}\vec{x}_{\infty} + \tilde{\mathbf{B}}\Big(Nr_{\infty}-\mathbf{K}\vec{x}_{\infty}\Big)
	\end{aligned}
\end{equation}
or
\begin{equation}\label{stfb_3}
	\begin{aligned}
		\vec{x}_{\infty} = -\Big(\tilde{\mathbf{A}}-\tilde{\mathbf{B}}\mathbf{K}\Big)^{-1}\tilde{\mathbf{B}}Nr_{\infty}
	\end{aligned}
\end{equation}
and in steady state we want to satisfy the following condition $y_{\infty}=r_{\infty}$
\begin{equation}\label{stfb_4}
	\begin{aligned}
		y_{\infty}=\mathbf{C}\vec{x}_{\infty} = -\mathbf{C}\Big(\tilde{\mathbf{A}}-\tilde{\mathbf{B}}\mathbf{K}\Big)^{-1}\tilde{\mathbf{B}}Nr_{\infty}=r_{\infty}
	\end{aligned}
\end{equation}
which brings to the following definition of $N$ ($N$ is scalar)
\begin{equation}\label{stfb_5}
	\begin{aligned}
		N = -\frac{1}{\mathbf{C}\Big(\tilde{\mathbf{A}}-\tilde{\mathbf{B}}\mathbf{K}\Big)^{-1}\tilde{\mathbf{B}}}
	\end{aligned}
\end{equation}
\begin{figure}[H]
	\centering
	\includegraphics[width = 0.5\textwidth, width = 400pt, keepaspectratio]{figures/servo/state_fb_state_obs_ct}
	\captionsetup{width=0.75\textwidth}		
	\caption{Control system layout of the approximated model in continuous time domain.}
	\label{control_layout_2}
\end{figure}
Regarding the observer block we can consider the following system equation:
\begin{equation}
	\begin{aligned}
		\dot{\hat{\vec{x}}}(t)  &= \tilde{\mathbf{A}} \,\hat{\vec{x}}(t) +\tilde{\mathbf{B}} \,u(t)+\tilde{\mathbf{L}} \left( \vec{y}(t) - \hat{\vec{y}}(t) \right)  \\[6pt]
		\hat{\vec{y}}(t)  &= \mathbf{C} \,\hat{\vec{x}}(t)
	\end{aligned}
\end{equation}
where $\tilde{\mathbf{L}}$ matrix gain is obtained by the pole placement procedure. 

Let $\mathbf{P}_{obs}$ be the desired poles regarding the state observer where
\begin{equation*}
	\mathbf{P}_{obs}=\begin{bmatrix}p_1^{obs} & p_2^{obs} & \cdots & p_n^{obs}\end{bmatrix}
\end{equation*}
the matrix gain $\tilde{\mathbf{L}}$ can be determined by the \texttt{acker} command in matlab: 
$$\texttt{L = (acker(A', C', Pobs))';}$$

%Scope of the laboratory is to show different control implementation and for this purpose an C-code implementation will be also taken into account. The C-code implementation has been carried out using the Simulink\textsuperscript{\tiny\textregistered} function \textbf{C-caller}. Necessary condition to implement a controller in C-code is to consider the system in a discrete-time representation. For this purpose the discrete time representation will be in the following section explored.


\subsection{Discrete-time domain}
In this section we shall explain how to converter a continuous time domain system into a discrete time domain. This transformation is of fundamental importance for the implementation of the control into digital micro-controller. Currently most of the control implementation are based on micro-controller or FPGA. Analog controller, which emulate a real continuous time domain, are in minority. Moreover direct code generation is also very common in particular for modern advanced control system. In section \ref{discretization} a short review concerning the concept of the discretization of state space system is reported.

The state space representation in the discrete-time domain of the system
\begin{equation}\label{}
	\begin{aligned}
		\dot{\vec{x}}(t) &=\tilde{\mathbf{A}} \vec{x}(t) + \tilde{\mathbf{B}} u(t) \\[6pt]
		y(t) &= \mathbf{C}\vec{x}(t)
	\end{aligned}
\end{equation}
where 
\begin{equation}\label{dssr_ref}
	\tilde{\mathbf{A}} = \begin{bmatrix}
		0 & 1 \\[6pt]
		0 & 0
	\end{bmatrix} \qquad
	\tilde{\mathbf{B}} = \begin{bmatrix} 0\\[6pt] 1\end{bmatrix} \qquad
	\mathbf{C} = \begin{bmatrix}
		k & 0
	\end{bmatrix}
\end{equation}
is the following (consider  $t_s$ as sampling time) 
\begin{equation}\label{dssr1}
	\begin{aligned}
		{\vec{x}}(k+1) &={\mathbf{A}} \vec{x}(k) + {\mathbf{B}} u(k) \\[6pt]
		y(k) &= \mathbf{C}\vec{x}(k)
	\end{aligned}
\end{equation}
where $u(k) = u_c(k)$ and $y(k) = x_p(k)$ and where
\begin{equation}\label{dssr2}
	\begin{aligned}
		\mathbf{A} &=  \mathbf{I}+\tilde{\mathbf{A}}t_s\\[6pt]
		\mathbf{B} &=  \tilde{\mathbf{B}}t_s
	\end{aligned}
\end{equation}
Equation~(\ref{dssr_ref}) and equation~(\ref{dssr1}) represent the same physical model where the first one is in the continuous-time domain while the second one is in the discrete-time domain.
\begin{figure}[H]
	\centering
	\includegraphics[width = 0.5\textwidth, width = 300pt, keepaspectratio]{figures/servo/discretization_statespace_2.eps}
	\captionsetup{width=0.75\textwidth}		
	\caption{State space representation of a same system: \textit{(b)} continuous-time domain case; \textit{(a)} discrete-time domain case.}
	\label{ss_representation}
\end{figure}

Regarding the observer block we can consider the following system equation:
\begin{equation}
	\begin{aligned}
		\hat{\vec{x}}(k+1)  &= \mathbf{A} \,\hat{\vec{x}}(k) +\mathbf{B} \,u(k)+\mathbf{L} \left( \vec{y}(k) - \hat{\vec{y}}(k) \right)  \\[6pt]
		\hat{\vec{y}}(k)  &= \mathbf{C} \,\hat{\vec{x}}(k)
	\end{aligned}
\end{equation}
where $\mathbf{L}$ matrix gain is obtained by the \texttt{acker} command in Matlab by a proper transformation of the desired poles from $s$~plane to $z$~plane, by the formula $$\mathbf{P}_{obs}^d = \exp\big({\mathbf{P}_{obs}t_s}\big)$$ and then apply the command:
$$\texttt{Ld = (acker(Ad', C', Pdobs))';}$$
For the discrete-time domain the poles used for state observer must be located inside the unit circle of the z-plane, see also Figure~\ref{poles}, in fact, the discretization process maps stable poles located into the left hand side of the $s$~plane into stable poles located inside the unit circle of the $z$~plane.
\begin{figure}[H]
	\centering
	\includegraphics[width = 0.5\textwidth, width = 300pt, keepaspectratio]{figures/servo/discretization_statespace_conformal_map.eps}
	\captionsetup{width=0.75\textwidth}		
	\caption{\textit{(a)} Region bounded by lines $\omega=\omega_1$, $\omega=-\omega_2$, $\sigma = -\sigma_1$ and $\sigma=-\sigma_2$ in the $s$-plane; \textit{(b)} the corresponding region in the $z$~plane.}
	\label{poles}
\end{figure}
The realized control system is shown in Figure~\ref{dt_control_system} 
\begin{figure}[H]
	\centering
	\includegraphics[width = 0.5\textwidth, width = 400pt, keepaspectratio]{figures/servo/state_fb_state_obs_dt}
	\captionsetup{width=0.75\textwidth}		
	\caption{Control system layout for the discrete-time domain case.}
	\label{dt_control_system}
\end{figure}

\section{Servo sharing control}

\section{Head losses of the valve actuated piston.}
In this section the power losses across a valve which controls a servo-piston is taken into analysis. The system valve-chamber can be approximated as shown in Figure~\ref{head_loss_1} where valve is modelled as a smooth nozzle and the chamber as an infinite sudden open. We also assume that the effect of the height (for head calculation) is negligible and we assume the flow speed into the chamber is almost null in comparison to the flow speed of the valve. 
\begin{figure}[H]
	\centering
	\includegraphics[width = 0.5\textwidth, width = 400pt, keepaspectratio]{figures/power_losses/head_loss_1.eps}
	\captionsetup{width=0.75\textwidth}		
	\caption{Head load approximation into a chamber valve controlled.}
	\label{head_loss_1}
\end{figure}
From theory it is known that an infinite sudden open generate a total head loss of 
\begin{equation}
	p_2 + \frac{1}{2}\rho v_2^2 \rightarrow p_2
\end{equation} 
where the energy density term $\frac{1}{2}\rho v_2^2$ is totally dissipated into the fluid.

The total power losses, of the valve-chamber system, can be approximated to
\begin{equation}
P_{loss} = (p_1-p_2)v_2
\end{equation} 

\chapter{Electromechanical    actuators}

\section{Solenoid actuated valve.}
\subsection{Solenoid actuator}
In order to identify the fundamental equations which govern this component we consider the magnetic circuit without losses ($\mu\rightarrow\infty$), see also Figure~\ref{solenoid_magnetic_1}. That means the equation 
\begin{equation}\label{lossless_1}
	\vec{B}=\mu\vec{H}
\end{equation}
reduces to 
\begin{equation}\label{lossless_2}
	\vec{B}=\mu_0\vec{H}
\end{equation}
Thus the only non-zero $\vec{H}$ occurs in the air-gap $g$ and $x$.

To derive the electromagnetic equations we have to consider two different integration paths: the first circulation path concern the Ampere's law (see Eq.~\eqref{ampere_law_1}) and is shown Figure~\ref{solenoid_magnetic_1} with the symbol (1):
\begin{equation}\label{ampere_law_1}
	\oint_{C}\vec{H}\cdot d\vec{l}=\int_{S}\vec{J}\cdot\hat{n}\,da
\end{equation}
which results in the following equation
\begin{equation}\label{ampere_law_2}
	H_1g+H_2x=N\,i_c
\end{equation}
The second law which we have to apply is
\begin{equation}\label{gauss_law_magnetic_1}
	\oint_{S}\vec{B}\cdot\hat{n}\,da=0
\end{equation}
where the integration is done over the closed surface $S$ which is shown in Figure~\ref{solenoid_magnetic_2} by the symbol (2). 
In order to apply Eq.~\eqref{gauss_law_magnetic_1} additional geometric information are necessary according Figure~\ref{solenoid_mech_2}. Equation~\ref{gauss_law_magnetic_1} results in the following equation (see Figure~\ref{solenoid_magnetic_2})
\begin{equation}\label{gauss_law_magnetic_2}
	\mu_0H_1\Big(2wd\Big)-mu_0H_2\Big(2wd\Big)=0
\end{equation}
Combining Eq.~\eqref{gauss_law_magnetic_2} and Eq.~\eqref{ampere_law_2} we obtain
\begin{equation*}\label{}
	H_1=H_2=\frac{N\,i_c}{g+x}
\end{equation*}
\begin{figure}[H]
	\centering
	\begin{subfigure}{.5\textwidth}
	\centering
	\includegraphics[width = 0.5\textwidth, width = 240pt, angle = 0, keepaspectratio]{figures/electromechanical/electromagnetic_actuator_flow_small.eps}
	\captionsetup{width=0.75\textwidth}		
	\caption{Lossless magnetic circuit of the solenoid actuator.}
	\label{solenoid_magnetic_1}
	\end{subfigure}%
	\begin{subfigure}{.5\textwidth}
	\centering
	\includegraphics[width = 0.5\textwidth, width = 240pt, angle = 0, keepaspectratio]{figures/electromechanical/surface_encloses_pluger_small.eps}
	\captionsetup{width=0.75\textwidth}		
	\caption{Hydraulic linear actuator during position changing.}
	\label{solenoid_magnetic_2}
	\end{subfigure}
\caption{Solenoid actuator.}
\label{}
\end{figure}
\begin{figure}[H]
	\centering
	\begin{subfigure}{.5\textwidth}
	\centering
	\includegraphics[width = 0.5\textwidth, width = 200pt, angle = 0, keepaspectratio]{figures/electromechanical/electromagnetic_actuator_size_small.eps}
	\captionsetup{width=0.75\textwidth}		
	\caption{Geometrical data.}
	\label{solenoid_mech_2}
	\end{subfigure}%
	\begin{subfigure}{.5\textwidth}
	\centering
	\includegraphics[width = 0.5\textwidth, width = 200pt, angle = 0, keepaspectratio]{figures/electromechanical/equivalent_circuit.eps}
	\captionsetup{width=0.75\textwidth}		
	\caption{Solenoid equivalent electrical circuit.}
	\label{equivalent_circuit_1}
	\end{subfigure}
\caption{Solenoid actuator.}
\label{}
\end{figure}
The flux through the center leg of the core is simply the flux crossing the air gap $x$ and is
\begin{equation}\label{flow_1}
	\phi=\mu_0H_2\Big(2wd\Big)=\frac{2wd\mu_0Ni_c}{g+x}
\end{equation}
Supposing $\mu\rightarrow\infty$ and $x$ small in comparison with $w$ we can suppose the leakage flux is negligible therefore the flux linkage to the $N$-turn coil is as follows
\begin{equation}\label{flow_2}
	\psi=N\phi=\mu_0H_2\Big(2wd\Big)=\frac{2wd\mu_0N^2i_c}{g+x}
\end{equation}
We can see that $\psi$ is a linear function of $i_c$, hence the system is electrically linear and we can write:
\begin{equation*}\label{}
	\psi = L(d)i_c
\end{equation*}
where
\begin{equation}\label{inductance_solenoid}
	L(x) = \frac{2wd\mu_0N^2}{g+x} = \frac{L_0}{1+x/g}
\end{equation}
where $L_0=(2wd\mu_0N^2)/g$. 

Assuming that $i_c=i_c(t)$, $x=x(t)$ and $\psi=\psi(i_c,d)$ we can write the following Kirchhoff's voltage law
\begin{equation}\label{electrical_equation_1}
	\boxed{u_c(t)-R_ci_c(t)-\frac{d\psi(i_c,x)}{dt}=0}
\end{equation}
where 
\begin{equation}\label{electrical_equation_2}
	\begin{aligned}
		\frac{d\psi(i_c,x)}{dt}&=\frac{\partial \psi(i_c,x)}{\partial i_c}\frac{di_c(t)}{dt} + \frac{\partial \psi(i_c,x)}{\partial x}\frac{dx(t)}{dt} \\[6pt]
		&= -\frac{2wd\mu_0N^2}{\big(g+x\big)^2}i_c(t)\frac{dx(t)}{dt} + \frac{2wd\mu_0N^2}{g+x}\frac{di_c(t)}{dt} \\[6pt]
		&= -\frac{L_0}{g(1+x/g)^2}i_c(t)\frac{dx(t)}{dt} + \frac{L_0}{1+x/g}\frac{di_c(t)}{dt} \\[6pt]
		&= -u_{\text{emf}}(i_c,x) + \frac{L_0}{1+x/g}\frac{di_c(t)}{dt}
	\end{aligned}
\end{equation}
hence Eq.~\eqref{electrical_equation_1} can be rewritten as follows
\begin{equation}\label{electrical_equation_3}
	u_c(t)-R_ci_c(t)+\underbrace{\frac{L_0}{g(1+x/g)^2}i_c(t)v_v(t)}_{u_{\text{emf}}(i_c,x)} - \frac{L_0}{1+x/g}\frac{di_c(t)}{dt}=0
\end{equation}
where $v(t) = d\,x(t)/dt$. Eq.~\eqref{electrical_equation_3} which results in the following equivalent electrical circuit (see Figure~\ref{equivalent_circuit_1})
To complete the solenoid actuator model we must add the mechanical equations using the Newton's equation. To apply the Newton's law we first must calculate the corresponding force actuated by the solenoid.

We first write the \textbf{magnetic coenergy} as
\begin{equation}\label{solenoid_mech_eq_1}
	\begin{split}
		W_m'=\int_{0}^{i_c}\psi(i,x)\,di &= \int_{0}^{i_c}\mu_0H_2\Big(2wd\Big)=\frac{L_0}{1+x/g}i\,di\\[6pt]
		&=\frac{1}{2}\frac{L_0}{1+x/g}i_c^2
	\end{split}
\end{equation}
Hence the force applied to the spool due to the solenoid is as follows
\begin{equation}\label{solenoid_mech_eq_2}
	\begin{split}
		f^v(t)=\frac{\partial W_m'}{\partial x} = -\frac{1}{2}\frac{L_0}{g(1+x/g)^2}i_c^2(t)
	\end{split} 
\end{equation}
The mechanical equations becomes
\begin{equation}\label{solenoid_mech_eq_3}
	\left\lbrace \begin{aligned}
		\frac{dx(t)}{dt} &= v(t) \\[6pt]
		\frac{dv(t)}{dt} &= \frac{1}{2m_v}\frac{L_0}{g\Big(1+x/g\Big)^2}i_c^2(t)-\frac{b}{m_v} v(t)-\frac{k_v}{m_v}\,x(t)
	\end{aligned}\right. 
\end{equation}
Hence the full electromechanical equations of the solenoid actuator become as follows
\begin{equation}\label{solenoid_mech_eq_4}
	\left\lbrace \begin{aligned}
		\frac{dx(t)}{dt} &= v(t) \\[6pt]
		\frac{dv(t)}{dt} &= \frac{1}{2m_v}\frac{L_0}{g\Big(1+\frac{x(t)}{g}\Big)^2}i_c^2(t)-\frac{b}{m_v} v(t) - -\frac{k_v}{m_v}\,x(t)\\[6pt]
		\frac{di_c(t)}{dt} &=-\frac{R_c}{L_0}\Big(1+\frac{x(t)}{g}\Big)i_c(t)+\frac{i_c(t)}{g\Big(1+\frac{x(t)}{g}\Big)}v(t) +\frac{\Big(1+\frac{x(t)}{g}\Big)}{L_0}u_c(t)
	\end{aligned}\right. 
\end{equation}
Now we have to consider two solenoid actuator where:
\begin{equation*}\label{}
	\left\lbrace \begin{aligned}
		x_f &= x_0 - x_v(t) \quad\text{forward solenoid} \quad u_c^f(t)>0,\,u_c^r(t)=0\\[6pt]
		x_r &= x_0 + x_v(t) \quad\text{reverse solenoid} \quad u_c^f(t)=0,\,u_c^r(t)>0
	\end{aligned}\right. 
\end{equation*}
where $x_0=x_v^{\text{max}}$.

The equations which govern the forward solenoid can be represented as follows 
\begin{equation}\label{solenoid_mech_eq_5}
	\left\lbrace \begin{aligned}
		u_c^f(t)&\ge0 \quad \Rightarrow \quad x_v(t)\ge 0 \\[6pt]
		u_c^r(t)&=0\\[6pt]
		\frac{dx_v(t)}{dt} &= v_v(t) \\[6pt]
		\frac{dv_v(t)}{dt} &= \frac{1}{2m_v}\frac{L_0}{g\Big(1+\frac{x_0 - x_v(t)}{g}\Big)^2}\Big[i_c^f(t)\Big]^2-\frac{b}{m_v} v_v(t) - \frac{k_v}{m_v}\,x_v(t)\\[6pt]
		\frac{di_c^f(t)}{dt} &=-\frac{R_c}{L_0}\Big(1+\frac{x_0 - x_v(t)}{g}\Big)i_c^f(t)+\frac{i_c^f(t)}{g\Big(1+\frac{x_0 - x_v(t)}{g}\Big)}v_v(t) +\frac{\Big(1+\frac{x_0 - x_v(t)}{g}\Big)}{L_0}u_c^f(t)
	\end{aligned}\right. 
\end{equation}
while the equations which govern the reverse solenoid can be represented as follows 
\begin{equation}\label{solenoid_mech_eq_6}
	\left\lbrace \begin{aligned}
		u_c^f(t)&=0 \\[6pt]
		u_c^r(t)&>0 \quad \Rightarrow \quad x_v(t)\le 0 \\[6pt]
		\frac{dx_v(t)}{dt} &= v_v(t) \\[6pt]
		\frac{dv_v(t)}{dt} &= -\frac{1}{2m_v}\frac{L_0}{g\Big(1+\frac{x_0 + x_v(t)}{g}\Big)^2}\Big[i_c^r(t)\Big]^2-\frac{b}{m_v} v_v(t) - \frac{k_v}{m_v}\,x_v(t)\\[6pt]
		\frac{di_c^r(t)}{dt} &=-\frac{R_c}{L_0}\Big(1+\frac{x_0 + x_v(t)}{g}\Big)i_c^r(t)+\frac{i_c^r(t)}{g\Big(1+\frac{x_0 + x_v(t)}{g}\Big)}v_v(t) +\frac{\Big(1+\frac{x_0 + x_v(t)}{g}\Big)}{L_0}u_c^r(t)
	\end{aligned}\right. 
\end{equation}
\subsection{Control implementation}
The derived model represented in Eqs~\ref{linact_11} and \ref{linact_12} are strongly non linear to implement a control based on a linearisation around a stable point. The control strategy which has been implemented is a cascade control which try to decouple the dynamic of the solenoid with the dynamic of the servo. 
\begin{figure}[H]
	\centering
	\includegraphics[width = 0.5\textwidth, width = 550pt, angle = 90, keepaspectratio]{figures/electromechanical/solenoid_servo_control_2.eps}
	\captionsetup{width=0.75\textwidth}		
	\caption{Description of the control architecture for the position of the servo.}
	\label{solenoid_servo_control}
\end{figure}
in fact as shown in Figure~\ref{solenoid_servo_control} the control is made by two loops. The first loop, the inner, is a PI control with feedback linearization (the term $u_{\text{emf}}$ is added to the output of the PI control in order to compensate its counterpart in the plant); it controls the current in the coil and must be designed with a very high bandwidth. This loop is responsible to generate the force $f_v(t)$ which govern the dynamic of the hydraulic valve. The reference force $f_v^*(t)$ comes from the first loop control and its converted into the $i_c^*(t)$ current reference by the proper non linear relation.   

The second loop is a proportional control and apply a the reference force $f_v^*(t)$ in order as function of the position error of the servo piston.

\chapter{Diesel engine model and control}
\section{Hydrostatic-Vehicle.}
Here we describe how the IC-engine has been modelized. We start to consider two curves
\begin{itemize}
	\item Torque curve: maximum torque available for a given rotor speed and maximum dispacement of the throttle.
	\item Torque friction curve: torque generated by the friction for a given rotor speed. This friction is considered always present, for any value of the throttle. See also Figure~\ref{engine_curves}
\end{itemize}
\begin{figure}[H]
	\centering
	\includegraphics[width = 0.5\textwidth, width = 260pt, angle = 0, keepaspectratio]{figures/engine/engine_torque_curve}
	\captionsetup{width=0.75\textwidth}		
	\caption{Engine model and control architecture.}
	\label{engine_curves}
\end{figure}
The mechanical model of the engine can be described as follows
\begin{equation}
	J\frac{d\omega_e}{dt} = \theta_f(t)\tau^{nom}(\omega_e) + \tau^{b}(\omega_e) - \tau_{load}
\end{equation}
where $\tau^{nom}(\omega_e)$ and $\tau^{b}(\omega_e)$ are shown in Figure~\ref{engine_curves} while $\theta_f(t)$ is the throttle which is generated by the external speed loop control as shown in Figure~\ref{engine_ctrl_1} and $J=\SI{7.5}{\kilogram\per\square\meter}$.

The speed control is performed by a PI-controller and it adapts the throttle displacement in order to keep the request speed tracked.

An additional second order filter is taken into account in order to modelize additional dynamics.

The controller can be described as follows
\begin{equation}
	\left\lbrace \begin{aligned}
		&\tilde{\omega}_e(t) = \frac{1}{\omega_e^{\text{nom}}}\Big(\omega^{\text{ref}}_e - \omega_e(t)\Big)\\[6pt]
		&\theta(t) = k_p\,\tilde{\omega}_e(t) + \theta^i(t) \\[6pt]
		&\frac{d\theta^i}{dt}(t) = k_i\,\tilde{\omega}_e(t)	
	\end{aligned}\right. 
\end{equation}
The control output $\theta$ is passed through a second order filter:
\begin{equation}
	\begin{aligned}
		\theta_f(s) = \frac{\omega_0^2}{s^2+2\zeta\omega_0s+\omega_0^2}\theta(s)
	\end{aligned}
\end{equation}
where $\zeta=1$ and $\omega_0=2\pi\SI{25}{\hertz}$.
\begin{figure}[H]
	\centering
	\includegraphics[width = 0.5\textwidth, width = 480pt, angle = 0, keepaspectratio]{figures/engine/diesel_engine_ctrl_2.eps}
	\captionsetup{width=0.75\textwidth}		
	\caption{Engine model and control architecture.}
	\label{engine_ctrl_1}
\end{figure}

\subsubsection{Simulation comparison}
In the following we can see the simulation comparison with \textbf{test-28.1} in order to evaluate the performance of the new model.

\begin{figure}[H]
	\centering
	\begin{subfigure}{.5\textwidth}
	\centering
	\includegraphics[width = 240pt, angle = 0, keepaspectratio]{figures/engine/test_28_1_fig/engine_quantities_test_28_1.eps}
	\captionsetup{width=0.75\textwidth}		
	\caption{Engine torque and speed.}
	\label{}
	\end{subfigure}%
	\begin{subfigure}{.5\textwidth}
	\centering
	\includegraphics[width = 240pt, angle = 0, keepaspectratio]{figures/engine/test_28_1_fig/abs_driveline_pressureA_test_28_1.eps}
	\captionsetup{width=0.75\textwidth}		
	\caption{Driveline pressure A.}
	\label{}
	\end{subfigure}
\caption{Simulation results.}
\label{}
\end{figure}

\begin{figure}[H]
	\centering
	\begin{subfigure}{.5\textwidth}
		\centering
	\includegraphics[width = 240pt, angle = 0, keepaspectratio]{figures/engine/test_28_1_fig/abs_driveline_pressureA_test_28_1_zoom.eps}
	\captionsetup{width=0.75\textwidth}		
	\caption{Driveline pressure A - dynamic behaviour .}
	\label{dynamic_APb}
	\end{subfigure}%
\begin{subfigure}{.5\textwidth}
	\centering
	\includegraphics[width = 240pt, angle = 0, keepaspectratio]{figures/engine/test_28_1_fig/abs_driveline_pressureB_test_28_1.eps}
	\captionsetup{width=0.75\textwidth}		
	\caption{Driveline pressure B.}
	\label{}
	\end{subfigure}
\caption{Simulation results.}
\label{}
\end{figure}

\begin{figure}[H]
	\centering
	\begin{subfigure}{.5\textwidth}
	\centering
	\includegraphics[width = 240pt, angle = 0, keepaspectratio]{figures/engine/test_28_1_fig/abs_driveline_pressureB_test_28_1_zoom.eps}
	\captionsetup{width=0.75\textwidth}		
	\caption{Driveline pressure B - dynamic behaviour.}
	\label{dynamic_BPb}
	\end{subfigure}%
\begin{subfigure}{.5\textwidth}
	\centering
	\includegraphics[width = 240pt, angle = 0, keepaspectratio]{figures/engine/test_28_1_fig/diff_driveline_pressure_test_28_1.eps}
	\captionsetup{width=0.75\textwidth}		
	\caption{Differential driveline pressure.}
	\label{}
	\end{subfigure}
\caption{Simulation results.}
\label{}
\end{figure}

\begin{figure}[H]
	\centering
	\includegraphics[width = 240pt, angle = 0, keepaspectratio]{figures/engine/test_28_1_fig/HMM4_pressure_test_28_1_zoom.jpg}
	\captionsetup{width=0.75\textwidth}		
	\caption{Hydro motor servo pressure - dynamic behaviour.}
	\label{dynamic_M4b}
\end{figure}

\subsubsection{List of parameters used for the simulation}
In this section the list of parameters used in the simulations is reported. Considering the two case: laid on ground and lifted from ground, for each case the set of parameters have not been modified. Between laid on ground and lifted the following parameters have been changed.
\begin{itemize}
	\item vehicle weight
	\item ground viscosity
	\item motor displacement calibration
\end{itemize}
\vspace{10mm}
\texttt{\underline{EDC Danfoss H1P147}
	\begin{itemize}
		\item Servo-actuator parameters
		\begin{itemize}
			\item Compressibility effects: \textit{checked}
			\item Piston area chamber A $=\SI{0.0029235}{\square\meter}$
			\item Piston area chamber B $=\SI{0.0029235}{\square\meter}$
			\item Piston stroke (both ways)	 $=\SI{0.03647}{\meter}$ 	
			\item Spring coefficient $=\SI{40.9e3}{\newton\per\meter}$			
			\item Spring damping $=\SI{1000}{\newton\per\meter\second}$			
			\item Piston mass $=\SI{1}{\kilogram}$
		\end{itemize}
		\item Valve parameters
		\begin{itemize}
			\item Number of holes per orifice $= \SI{1}{}$
			\item P-A. Maximum orifice opening $= \SI{0.0012}{\meter}$
			\item P-B. Maximum orifice opening $= \SI{0.0012}{\meter}$
			\item T-A. Maximum orifice opening $= \SI{0.0012}{\meter}$
			\item T-B. Maximum orifice opening $= \SI{0.0012}{\meter}$
			\item Orifice opening offset P-A $=\SI{1.9e-4}{\meter}$
			\item Orifice opening offset P-B $=\SI{1.9e-4}{\meter}$
			\item Orifice opening offset T-A $=\SI{1.9e-4}{\meter}$
			\item Orifice opening offset T-B $=\SI{1.9e-4}{\meter}$
			\item Leakage area $=\SI{1e-8}{\square\meter}$
		\end{itemize}
		\item Spool control parameters
		\begin{itemize}
			\item Proportional gain $=1.014$
			\item Integral gain $=0.001$			
			\item Scaling factor $=\SI{5e-4}{}$
			\item Lag time constant (coil) $=\SI{0.009859}{\second}$
		\end{itemize}
		\item Swash-plate parameters
		\begin{itemize}
			\item Maximum angular displacement $=\SI{18}{\deg}$
		\end{itemize}
	\end{itemize}
}


\end{document} 