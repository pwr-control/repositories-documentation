%\documentclass[11pt,a4paper]{article}
\documentclass[11pt,a4paper]{scrartcl}
%\documentclass[11pt,a4paper,oneside]{book}
\usepackage[british,UKenglish,USenglish,english,american]{babel}
%\usepackage[a4paper, total={16cm, 23cm}]{geometry}
\usepackage[tmargin = 1.25in,bmargin = 1.25in,lmargin = 0.75in,rmargin = 0.75in]{geometry}
\usepackage{tikz}
\usepackage{graphicx}
\usepackage{pgfplots}
\pgfplotsset{width=12cm,compat=1.9}
\usepackage{setspace}
\usepackage{chemmacros}
\usepackage{chemfig}
%\usepackage{ghsystem}
\usechemmodule{redox}
%\usepackage{chemnum}
%\usepackage{bohr}
%\usepackage{elements}
%\usepackage{endiagram}
%\usepackage{modiagram}
%\usepackage{chemgreek}
%\usepackage{mhchem}
\usepackage{esint}
\usepackage{tabularray}

\usepackage{makeidx}
\usepackage{epstopdf}

\usepackage{amssymb}
\usepackage{mathrsfs}
%\usepackage{minted}
\usepackage{bm}
\usepackage{amsmath}
\usepackage{enumitem}
\usepackage[english]{varioref}
\usepackage[english]{babel}
\usepackage{lipsum}
\usepackage{fancyhdr}
\pagestyle{fancy} 
\usepackage{float}
\usepackage{empheq}
\usepackage[framemethod=tikz]{mdframed}
\usepackage{epstopdf}
\numberwithin{equation}{section}
\usepackage{eso-pic}
\usepackage{calc}
\usepackage{nccmath}
\usepackage{caption}
\usepackage{subcaption}
\usepackage{gensymb}
\usepackage{amsfonts,amsthm,epsfig,epstopdf,titling,url,array}
\usepackage{siunitx}
\sisetup{input-digits = 0123456789\pi}
\usepackage[symbol]{footmisc}
\usepackage{xcolor}
\usepackage{multicol}
\usepackage{boondox-cal}
\DeclareSIUnit\atm{atm}
\setcounter{secnumdepth}{3}
\setcounter{tocdepth}{3}
\usepackage{booktabs}
\usepackage{blindtext}
\usepackage{changepage}

% \usepackage{draftwatermark}
% \SetWatermarkText{DRAFT}
% \SetWatermarkScale{5}

\DeclareSIUnit\atm{atm}

\pagestyle{fancy} 
\fancypagestyle{firstpage}{
\rhead{}
}
\fancyhead[L]{\small\slshape\nouppercase{\leftmark}}
\chead{}
\rhead{}
\lfoot{\textit{}}
\cfoot{-\ \thepage\ -}
\rfoot{\textit{}}

\DeclareMathOperator{\rank}{rank}
\DeclareMathOperator{\atantwo}{atan2}
\DeclareMathOperator{\arctantwo}{arctan2}
\DeclareMathOperator{\spn}{span}

\renewcommand{\headrulewidth}{0.4pt}
\renewcommand{\footrulewidth}{0.4pt}
\newcommand{\abs}[1]{\left|#1\right|}
\definecolor{mycolor1}{rgb}{0.97, 0.97, 0.97}
\definecolor{mycolor2}{rgb}{0.97, 0.97, 0.97}
\definecolor{tableShade}{gray}{0.9}
\newcommand{\sign}{\text{sign}}
\newcommand{\centered}[1]{\begin{tabular}{@{}l@{}} #1 \end{tabular}}
\theoremstyle{it}
\newtheorem{defn}{Definition}[section]
\newtheorem{assumption}{Assumption}[section]
\newtheorem{thm}{Theorem}[section]
\newtheorem{lemma}{Lemma}[section]
\newtheorem{corollary}{Corollary}[section]
\theoremstyle{definition}
%\theoremstyle{it}
\newtheorem{example}{Example}[section]
\let\eqrefn\eqref
\renewcommand{\eqref}[1]{Eq.~(\ref{#1})}

\newenvironment{myitemize_1}
{ \begin{itemize}[topsep=4pt]
		\setlength{\topsep}{2pt}		
		\setlength{\itemsep}{2pt}
		\setlength{\parskip}{2pt}
		\setlength{\parsep}{2pt}     }
	{ \end{itemize}                  }


\newmdenv[innerlinewidth=0.5pt, roundcorner=4pt,backgroundcolor=mycolor2, 
linecolor=mycolor1,innerleftmargin=6pt,
innerrightmargin=6pt,innertopmargin=6pt,innerbottommargin=6pt]{mybox}

\title{\textbf{ 
	\begin{LARGE}
		Moving Coil:
	\end{LARGE} \\[24pt]
	\begin{Large}
		Mathematical Model and Control
	\end{Large}}
}
\author{\textbf{Davide Bagnara}}

\begin{document}
	\thispagestyle{empty}
	\begin{mybox}
		\maketitle
		\vspace{150mm}
	\end{mybox}
%	\let\clearpage\relax
	\newpage
	\tableofcontents%
	\listoffigures%
	\listoftables
%	\let\clearpage\LaTeXStandardClearpage
	\newpage
	
\begin{onehalfspace}
	
	\pagebreak
\section{Moving Coil: mathematical model}\label{moving_coil_jackson_section}
The moving coil was already studied at the beginning of this document. Here we 
want to represent the moving coil model using a different approach, we would 
like to use a more mathematical (and less physical) approach to show a more 
formal representation. Most of the following material has been taken from the 
monumental work of \cite{furlani}.

We define two reference frames: one stationary $\mathcal{O}$ and one which lies 
on the moving coil $\mathcal{O'}$ and we define the following meaning of the 
electrical field $\vec{E}$ induced across the coil
\begin{itemize}
	\item $\vec{E}'$ is the induced electrical field observed from the moving 
	reference frame $\mathcal{O'}$.
	\item $\vec{E}$ is the induced electrical field observed from the 
	stationary reference frame $\mathcal{O}$.	
\end{itemize}  
The Faraday's law applied to such that circuit can be written as follows
\begin{equation}\label{jackson_1}
	\oint_\mathcal{C}\vec{E}'\cdot d\vec{l} = -\frac{d}{dt}\int_\mathcal 
	{S}\vec{B}\cdot\hat{n}\,da
\end{equation}
where $\vec{E}'$ is the electrical field applied to the coil respect the moving 
reference frame. 
\begin{figure}[H]
	\centering
	\includegraphics[width = 300pt, angle = 0, 
	keepaspectratio]{figures/moving_coil_3.eps}
	\captionsetup{width=0.75\textwidth}		
	\caption{Moving coil actuator.}
	\label{moving_coil_3a}
\end{figure}

The circuit is moving at a certain speed $\vec{v}$, hence, the total time 
derivation of the linkage flux at the right hand side of Eq.~\ref{jackson_1} 
shall take into account that the linkage flux can change due to change of 
$\vec{B}$ and due to the change of the contour $\mathcal{C}$ (or $\mathcal{S}$) 
due to the movement of the coil. 

%The total time derivation of the linkage flux at the right hand side of 
%Eq.~\ref{jackson_1} shall take into account that the linkage flux can change 
%due to change of $\vec{B}$ and due to the change of the position of the coil. 

The total time derivative\footnote{
	The concept of the convective derivative can be used \begin{equation*}
		\frac{d}{dt} = \frac{\partial}{\partial t} + \vec{v}\cdot\vec{\nabla}
	\end{equation*}
	we obtain that
	\begin{equation*}
		\frac{d\vec{B}}{dt} = \frac{\partial\vec{B}}{\partial t} + 
		\Big(\vec{v}\cdot\vec{\nabla}\Big)\vec{B} = \frac{\partial 
			\vec{B}}{\partial 
			t}+\vec{\nabla}\times\Big(\vec{B}\times\vec{v}\Big)+\vec{v}\Big(\vec{\nabla}\cdot\vec{B}\Big)
	\end{equation*}
	where $\vec{v}$ is considered a constant vector. Applying Stokes theorem we 
	obtain Eq.~\ref{jackson_2}
} of the flux through the moving coil is (see \cite{p14})
\begin{equation}\label{jackson_2}
	\frac{d}{dt}\int_\mathcal{S}\vec{B}\cdot\hat{n}\,da = 
	\int_\mathcal{S}\frac{\partial\vec{B}}{\partial t}\cdot\hat{n}\,da + 
	\oint_\mathcal{C}\Big(\vec{B}\times\vec{v}\Big)\cdot d\vec{l}
\end{equation}
Hence Eq.~\ref{jackson_1} can now be written in the form
\begin{equation}\label{jackson_3}
	\oint_\mathcal{C}\Big[\vec{E}'-\vec{v}\times\vec{B}\Big]\cdot d\vec{l} = 
	\int_\mathcal{S}\frac{\partial\vec{B}}{\partial t}\cdot\hat{n}\,da
\end{equation}

Eq.~\ref{jackson_3} can also be obtained as follows (see also 
Figure~\ref{panofky})
\begin{equation}\label{silvestrini_1}
	\oint_\mathcal{C}\vec{E}'\cdot d\vec{l} = -\frac{d}{dt}\int_\mathcal 
	{S}\vec{B}\cdot\hat{n}\,da = 
	-\frac{1}{dt}\Bigg[\int_{\mathcal{S}(t+dt)}\vec{B}(t+dt)\cdot\hat{n}\,da-\int_{\mathcal{S}(t)}\vec{B}(t)\cdot\hat{n}\,da\Bigg]
\end{equation}
\begin{figure}[H]
	\centering
	\includegraphics[width = 150pt, angle = 0, 
	keepaspectratio]{figures/panofsky.eps}
	\captionsetup{width=0.75\textwidth}		
	\caption{Evaluation of the $d\vec{B}/dt$.}
	\label{panofky}
\end{figure}
where 
\begin{equation}
	\vec{B}(t+dt)=\vec{B}(t)+\frac{\partial\vec{B}}{\partial t}
\end{equation}
The second term of Eq.~\ref{silvestrini_1} become
\begin{equation}\label{silvestrini_2}
	\begin{aligned}
		& 
		-\frac{1}{dt}\Bigg[\int_{\mathcal{S}(t+dt)} 
		\Big(\vec{B}(t)+\frac{\partial\vec{B}}{\partial 
			t}\Big)\cdot\hat{n}\,da-\int_{\mathcal{S}(t)} 
		\vec{B}(t)\cdot\hat{n}\,da\Bigg] = \\[8pt]
		& -\frac{1}{dt}\Bigg[\int_{\mathcal{S}(t)} 
		\vec{B}(t+dt)\cdot\hat{n}\,da 
		-\int_{\mathcal{S}(t)}\vec{B}(t) 
		\cdot\hat{n}\,da\Bigg]-\int_{\mathcal{S}(t+dt)} 
		\frac{\partial\vec{B}}{\partial
			t}\cdot\hat{n}\,da
	\end{aligned}
\end{equation}
The first term at the right side of the equation represents the variation of 
the flux of $\vec{B}$ due to the only change of the surface, and we can write
\begin{equation}\label{silvestrini_3}
	\begin{aligned}
		-\frac{1}{dt}\Bigg[\int_{\mathcal{S}(t)}\vec{B}(t+dt)\cdot\hat{n}\,da 
		-\int_{\mathcal{S}(t)}\vec{B}(t)\cdot\hat{n}\,da\Bigg] = 
		\oint_\mathcal{C}\Big(\vec{v}\times\vec{B}\Big)\cdot d\vec{l}
	\end{aligned}
\end{equation}
while the second term represent the variation of the flux due to the change of 
the vector $\vec{B}$
\begin{equation}\label{silvestrini_4}
	\begin{aligned}
		\int_{\mathcal{S}(t+dt)}\frac{\partial\vec{B}}{\partial 
			t}\cdot\hat{n}\,da = 
		\int_{\mathcal{S}(t)}\frac{\partial\vec{B}}{\partial 
			t}\cdot\hat{n}\,da\qquad dt\rightarrow0
	\end{aligned}
\end{equation}
and we obtain the Eq.~\ref{jackson_3}.

Now we apply the Faraday's law to the circuit respect the stationary reference 
frame $\mathcal{O}$ and by the fact that the circuit $\mathcal{C}$ and the 
surface $\mathcal{S}$ are fixed respect to the same stationary reference frame 
$\mathcal{O}$ the Faraday's law can be written in the form
\begin{equation}\label{jackson_3b}
	\oint_\mathcal{C}\vec{E}\cdot d\vec{l} = -\int_\mathcal {S}\frac{\partial 
		\vec{B}}{\partial t}\cdot\hat{n}\,da
\end{equation}
Galilean invariance Eq.~\ref{jackson_3} and Eq.~\ref{jackson_3b} must be equal, 
hence results that the electrical field of the coil measured respect to 
$\mathcal{O}$ can be expressed as (see also section 
\ref{moving_reference_section})
\begin{equation}\label{jackson_4}
	\boxed{	\vec{E} = \vec{E}'-\vec{v}\times\vec{B}}
\end{equation}
Hence the Faraday's law applied to the moving coil 
\begin{equation}\label{}
	\boxed{	\begin{aligned}
			&\oint_\mathcal{C} \vec{E}'\cdot\,d\vec{l}=-\frac{d}{d 
				t}\int_\mathcal{S}\vec{B}\cdot\hat{n}\,da
	\end{aligned}}
\end{equation} 
where the circuit $\mathcal{C}$ and the surface $\mathcal{S}$ are not 
considered fixed, can be written in the form
\begin{equation}\label{jackson_5}
	\boxed{ \begin{aligned} 
			\oint_\mathcal{C}\Big[\vec{E}'-\vec{v}\times\vec{B}\Big]\cdot d\vec{l} = 
			\int_\mathcal{S}\frac{\partial\vec{B}}{\partial t}\cdot\hat{n}\,da
	\end{aligned}}
\end{equation}
where the circuit $\mathcal{C}$ and the surface $\mathcal{S}$ can be considered 
fixed but where   we already account the speed of the coil.

From another, but equivalent, point of view, we can suppose to freeze, at a 
certain instant, the circuit and to define the contour $\mathcal{C}$  and its 
surface $\mathcal{S}$ as fixed respect to the stationary reference frame. Even 
if the contour $\mathcal{C}$ and the surface $\mathcal{S}$ are fixed in the 
space, the charge present into the coil is subjected to a vector speed 
$\vec{v}$ and hence is subjected to a force $\vec{f}= 
q(\vec{E}+\vec{v}\times\vec{B})$ respect to the reference frame $\mathcal{O}$, 
from Galilean invariance we can impose that  $\vec{f}=\vec{f}'=q\vec{E}'$ which 
results in $\vec{E}'=\vec{E}+\vec{v}\times\vec{B}$. That means, the induced 
electrical field along the coil seen from a stationary reference frame is given 
by 
\begin{equation}\label{silvestrini_5}
	\vec{E}=\vec{E}'-\vec{v}\times\vec{B}
\end{equation}
For ideal conductive coil we can write $$\vec{E}=-\vec{v}\times\vec{B}.$$
\begin{figure}[H]
	\centering
	\includegraphics[width = 350pt, angle = 0, 
	keepaspectratio]{figures/moving_coil_4.eps}
	\captionsetup{width=0.75\textwidth}		
	\caption{Lorentz force applied to the coil charge due to effect of the 
		speed.}
	\label{moving_coil_4}
\end{figure}


Now we can write the vector $\vec{B}$ as a superposition of two fields:  the 
$\vec{B}_i$ due to the current $i(t)$ and the external field $\vec{B}^M$ which 
is supposed to be constant and isotropic, 
\begin{equation*}
	\vec{B}(t)=\vec{B}_i(t)+\vec{B}^M
\end{equation*}
Therefore, the last term in Eq.~\ref{jackson_5} can be rewritten as
\begin{equation}\label{jackson_6}
	\begin{aligned}
		\int_\mathcal{S}\frac{\partial\vec{B}}{\partial t}\cdot\hat{n}\,da &= 
		\int_\mathcal{S}\frac{\partial \vec{B}_i}{\partial t}\cdot\hat{n}\, da 
		+ \int_\mathcal{S}\frac{\partial \vec{B}^M}{\partial t}\cdot\hat{n}\, da
	\end{aligned}
\end{equation} 
where 
\begin{equation}\label{jackson_6b}
	\begin{aligned}
		\frac{\partial \vec{B}^M}{\partial t} = 0
	\end{aligned}
\end{equation} 
\begin{equation}\label{jackson_6c}
	\begin{aligned}
		\frac{\partial \vec{B}_i}{\partial t} = \frac{\partial 
			\vec{B}_i(i)}{\partial i}\,\frac{d i(t)}{dt}
	\end{aligned}
\end{equation} 
Hence Eq.~\ref{jackson_6} can be written as
\begin{equation}\label{jackson_6d}
	\begin{aligned}
		\int_\mathcal{S}\frac{\partial\vec{B}}{\partial t}\cdot\hat{n}\,da = 
		\int_\mathcal{S}\Big[\frac{\partial \vec{B}_i(i)}{\partial i}\frac{d 
			i(t)}{dt}\Big]\cdot\hat{n}\,da = \frac{\partial \phi(i)}{\partial i} 
		\frac{di(t)}{dt} = L\frac{di(t)}{dt}
	\end{aligned}
\end{equation} 
where $\mathcal{S}$ is considered constant and $\partial \phi(i)/\partial i = 
L$ is the definition of inductance.

Now we apply Eq.~\ref{jackson_5} in clockwise around the contour $\mathcal{C}$ 
of the Figure~\ref{moving_coil_3a} and we consider the case that the coil is 
composed of $N_c$ turns
\begin{equation}\label{}
	\begin{aligned}
		&\psi(t) = N_c\phi(t) = N_c \int_{S}^{}\vec{B}\cdot d\vec{s} \\[6pt]
		&L_c = N_c\,L
	\end{aligned}
\end{equation}
we obtain 
\begin{equation}\label{jackson_7}
	\begin{aligned}
		\underbrace{\int_{(-)}^{(+)}\vec{E}\cdot\,d\vec{l}}_{\text{source}} \, 
		+ \underbrace{\int\vec{E}\cdot\,d\vec{l}}_{\text{resistor}} \, + 
		\underbrace{\int\vec{E}'\cdot\,d\vec{l}}_{\text{coil}} = - 
		\frac{d}{dt}\int_S\vec{B}\cdot\hat{n}\,da.
	\end{aligned}
\end{equation} 
or
\begin{equation}\label{jackson_8}
	\begin{aligned}
		\underbrace{\int_{(-)}^{(+)}\vec{E}\cdot\,d\vec{l}}_{\text{source}} \, 
		+ \underbrace{\int\vec{E}\cdot\,d\vec{l}}_{\text{resistor}} \, + 
		\underbrace{\int\Big[\vec{E}' - 
			\vec{v}_v\times\vec{B}^MN_c\Big]\cdot\,d\vec{l}}_{\text{coil}} = - 
		L_c\frac{di_c(t)}{dt}.
	\end{aligned}
\end{equation} 
The integrals over the fixed components (source and resistor) follow from 
Equations
\begin{equation}\label{jackson_9}
	-u_c = \underbrace{\int_{(-)}^{(+)}\vec{E}\cdot\, d\vec{l}}_{\text{source}}
\end{equation} 
\begin{equation}\label{jackson_9b}
	i_c\,R_c = \underbrace{\int\vec{E}\cdot\,d\vec{l}}_{\text{resistor}}
\end{equation} 
Concerning the coil we consider the case where the whole resistance is 
accounted in $R_c$ and we set $\vec{E}'= \vec{J}/\sigma = 0$, we obtain
\begin{equation}\label{jackson_10}
	\begin{aligned}
		u_c(t) - i_c(t)\,R_c-L_c 
		\frac{di_c}{dt}+\int_{\text{coil}}\big(\vec{v}_v\times\vec{B}^MN_c\big)\cdot
		d\vec{l}=0
	\end{aligned}
\end{equation} 
We are considering the case where $\vec{v}_v=v_v\hat{x}$, 
$\vec{B}\,^M=-B^M\,\hat{z}$ and $d\vec{l}=dy\hat{y}$. Therefore 
($\hat{x}\times\hat{z}=-\hat{y}$), 
\begin{equation}\label{jackson_11}
	\begin{aligned}
		\int_{\text{coil}}\big(\vec{v}_v\times\vec{B}^MN_c\big)\cdot d\vec{l} = 
		\int_{h}^{0}v_v\,B^MN_cdy = -N_cB^Mhv_v(t)
	\end{aligned}
\end{equation} 
we obtain
\begin{equation}\label{jackson_12}
	\boxed{	\begin{aligned}
			u_c(t) - i_c(t)R_c-L_c\frac{di_c(t)}{dt}-N_cB^Mhv_v(t)=0
	\end{aligned}}
\end{equation}

To complete the linear pm-actuator model we must add the mechanical equations 
using the Newton's equation. To apply the Newton's law we first must calculate 
the corresponding force actuated by the iteration between the current $i_c$ and 
the magnetic field $\vec{B}^M$.


The expression of the force due to the current $i_c$ in the magnetic field 
$\vec{B}^M$ is evaluated as follows 
\begin{equation}\label{force}
	\vec{f}(t) = i(t)\int_{\text{wire}}d\vec{l}\times\vec{B}^{\text{ext}}
\end{equation}
where $\vec{B}^{\text{ext}}$ is an external field. Applying the integration of 
Eq.~\ref{force} it results in the following equation (\textbf{integration path 
	follows the current direction})
\begin{equation}\label{linact_8}
	\begin{aligned}
		f_v^m(t) = 
		N_c\,i_c(t)\int_{h}^{0}d\vec{l}\times\vec{B}^M=N_c\,B^M\,h\,i_c(t)\quad 
		\text{positive x-direction}
	\end{aligned}
\end{equation}
where $N_c$ by the number of turns linkage to the magnetic field.
The complete set of system equations becomes
\begin{equation}\label{linact_9}
	\left\lbrace \begin{aligned}
		\frac{dx_v(t)}{dt} &= v_v(t) \\[6pt]
		\frac{dv_v(t)}{dt} &= N_c\frac{B^Mh}{m_v} i_c(t)-\frac{b_v}{m_v} 
		v_v(t)-\frac{1}{m_v}f^e_v(t) \\[6pt]
		\frac{di_c(t)}{dt} &= 
		-\frac{R_c}{L_c}i_c(t)-\frac{1}{L_c}N_cB^Mhv_v(t)+\frac{1}{L_c}u_c(t)
	\end{aligned}\right. 
\end{equation}
the first two equations represent the dynamic motion of the bar where the force 
$f_v^m(t)=N_cB^Mh\,i(t)$ is generated by the interaction of the current 
$i_c(t)$ and the magnetic field $\vec{B}^M$. The third equation represents the 
Kirchhoff's voltage law of the electrical circuit which generates the current 
$i_c(t)$ by applying the voltage $u_c(t)$. The equivalent auto-inductance $L_c$ 
and the back-emf term, $N_cB^Mhv_v(t)$ represent the effect of the Faraday's 
law.

\section{Moving Coil: control system design}
In the following project we are going to consider a linear actuator consisting 
of a conductive bar of mass $m$ in sliding contact with a pair of stationary 
conducting rails as shown in Figure~\ref{figure_moving_coil_1}. The rails are 
connected to a voltage source $u(t)$. The bar is moving trough a constant 
uniform $\vec{B}^M$ magnetic field with a time dependent velocity $v(t)$ 
relative to the rails. We can consider the rails an inertial reference frame. 
This model could be useful for modeling a plotter or in general a linear 
actuator.
\begin{figure}[H]
	\centering
	\includegraphics[width = 360pt, 
	keepaspectratio]{figures/moving_coil_2.eps}
	\captionsetup{width=0.5\textwidth, font=small}		
	\caption{Linear actuator.}
	\label{figure_moving_coil_1}
\end{figure}
\subsection{Model Derivation}
Consider
\begin{equation}\label{linact_1}
	\begin{aligned}
		\psi(t) = \int_{S}^{}\vec{B}\cdot d\vec{s}
	\end{aligned}
\end{equation}
the flux linkage to the circuit. At this point we have to point out that the 
magnetic field $\vec{B}$ include two main contributes: one constant term which 
comes from the external permanent magnet $\vec{B}^{\,M}$ and one which is 
generated by the flowing current $i$, that means 
$\vec{B}=\vec{B}_i+\vec{B}^{\,M}$.\\\\
Applying the general Kirchhoff's law we obtain
\begin{equation}\label{linact_2}
	\begin{aligned}
		u(t) - Ri(t)+\frac{d\psi(t)}{dt}=0
	\end{aligned}
\end{equation}
The term ${d\psi(t)/dt}$ represents the voltage induced in the circuit by a 
time rate of change of magnetic flux through the circuit. The change in flux 
can be due to a change in circuit current and/or the movement of the bar. 
Hence, the magnetic flux can be written as function of space and current 
$\psi(i,x)$
\begin{equation}\label{linact_3}
	\begin{aligned}
		{\psi(t)}=\int_{S}^{}\vec{B}\cdot d\vec{s}=\int_{S}^{}\left( 
		\vec{B}_i+\vec{B}^M\right) \cdot d\vec{s}=\psi(i,x)
	\end{aligned}
\end{equation}
Hence the total time derivative of $\psi(i,x)$ can be written as follows
\begin{equation}\label{linact_4}
	\begin{aligned}
		\frac{d\psi(i,x)}{dt}= \frac{\partial\psi(i,x)}{\partial i(t)}\frac{d 
			i(t)}{d t} + \frac{\partial\psi(i,x)}{\partial x(t)}\frac{d x(t)}{d t}
	\end{aligned}
\end{equation}
where $h$ (length of the bar) and ${B}^{\,M}$ (magnitude of the permanent 
magnet magnetic field) are considered constant parameters
\begin{equation}\label{linact_5}
	\begin{aligned}
		\frac{\partial\psi(i,x)}{\partial x(t)}\frac{d x(t)}{d t} = 
		\frac{\partial {B}x h}{\partial x} v(t) \approx B^Mhv(t)
	\end{aligned}
\end{equation}
and 
\begin{equation}\label{linact_6}
	\begin{aligned}
		\frac{\partial\psi(i,x)}{\partial i(t)}\frac{d i(t)}{d t} =L(x) 
		\frac{di}{dt} \approx L\frac{di}{dt}
	\end{aligned}
\end{equation}
Eq. \ref{linact_2} becomes
\begin{equation}\label{linact_7}
	\begin{aligned}
		u(t) - Ri(t)-L\frac{di(t)}{dt}-B^Mhv(t)=0
	\end{aligned}
\end{equation}
The expression of the force due to the current $i$ in the magnetic field 
$\vec{B}^M$ is given as follows (\textbf{integration path follows the current 
	direction})
\begin{equation}\label{linact_8}
	\begin{aligned}
		f_m=i(t)\int_{h}^{0}d\vec{l}\times\vec{B}^M=B^Mh\,i(t)\quad 
		\text{positive x-direction}
	\end{aligned}
\end{equation}
The complete set of system equations becomes
\begin{equation}\label{linact_9}
	\left\lbrace \begin{aligned}
		\frac{dx(t)}{dt} &= v(t) \\[6pt]
		\frac{dv(t)}{dt} &= \frac{B^Mh}{m} i(t)-\frac{b}{m} 
		v(t)-\frac{1}{m}f_l(t) \\[6pt]
		\frac{di(t)}{dt} &= -\frac{R}{L}i(t)-\frac{1}{L}B^Mhv(t)+\frac{1}{L}u(t)
	\end{aligned}\right. 
\end{equation}
the first two equations represent the dynamic motion of the bar where the force 
$f_m(t)=B^Mh\,i(t)$ is generated by the interaction of the current $i(t)$ and 
the magnetic field $\vec{B}^M$). The third equation represents the dynamic of 
the electrical circuit which generates the current $i(t)$ by applying the 
voltage $u(t)$. The equivalent auto-inductance $L$ represents the effect of the 
Faraday's law.\\ 

The current $i(t)$ and the position $x(t)$ are measured, hence, the state space 
representation can be represented as follows
\begin{mybox}
	\begin{equation*}
		\left\lbrace \begin{aligned}
			\dot{\vec{x}}(t) &= \tilde{\mathbf{A}} \vec{x}(t) + 
			\tilde{\mathbf{B}}u(t) + \tilde{\mathbf{E}}d(t)\\[6pt]
			{\vec{y}}(t) &= \tilde{\mathbf{C}} \vec{x}(t)
		\end{aligned}\right. 
	\end{equation*}\\
	\begin{equation*}
		\left[ 
		\begin{matrix}
			\frac{dx(t)}{dt} \\[6pt]
			\frac{dv(t)}{dt} \\[6pt]
			\frac{di(t)}{dt}
		\end{matrix} \right] = \left[ 
		\begin{matrix}
			0 & 1 & 0\\[6pt]
			0 & -\frac{b}{m} & \frac{B^{M}h}{m}\\[6pt]
			0 & -\frac{B^{M}h}{L} & -\frac{R}{L}
		\end{matrix} \right] \left[ 
		\begin{matrix}
			x(t) \\[6pt]
			v(t)\\[6pt]
			i(t)
		\end{matrix} \right] + 
		\left[ 
		\begin{matrix}
			0 \\[6pt]
			0 \\[6pt]
			\frac{1}{L}
		\end{matrix} \right] \ u(t) + 
		\left[ 
		\begin{matrix}
			0 \\[6pt]
			-\frac{1}{m} \\[6pt]
			0
		\end{matrix} \right] \ f_l(t)
	\end{equation*}\\
	\begin{equation*}
		\begin{aligned}
			\tilde{\mathbf{A}} &= \begin{bmatrix}
				0 & 1 & 0\\[6pt]
				0 & -\frac{b}{m} & \frac{B^{M}h}{m}\\[6pt]
				0 & -\frac{B^{M}h}{L} & -\frac{R}{L}
			\end{bmatrix}, 
			\quad 
			\tilde{\mathbf{B}} = \begin{bmatrix}
				0 \\[6pt]
				0 \\[6pt]
				\frac{1}{L}
			\end{bmatrix} \\[6pt]
			\tilde{\mathbf{E}} &= \begin{bmatrix}
				0 \\[6pt]
				-\frac{1}{m} \\[6pt]
				0
			\end{bmatrix}, \quad \tilde{\mathbf{C}} = 
			\begin{bmatrix}
				1 & 0 & 0 \\[6pt]
				0 & 0 & 1
			\end{bmatrix}
		\end{aligned}
	\end{equation*}
\end{mybox}
\subsection{Control Problem}
Given the following data: 
\begin{enumerate}
	\item Rails friction $b=\SI{20}{\newton\second\per\meter}$
	\item Length of the bar $h=\SI{0.2}{\meter}$	
	\item Mass of the bar $m=\SI{0.03}{\kilogram}$
	\item Magnitude of the permanent magnet magnetic field 
	$B^M=\SI{1.2}{\tesla}$
	\item Equivalent circuit inductance $L=\SI{1}{\milli\henry}$
	\item Equivalent circuit resistance $R=\SI{1}{\ohm}$
	\item $f_l(t)=b_2v(t)$ is an unmodelized viscosity, where 
	$b_2=\SI{0.1}{\newton\second\per\meter}$ for case (1) and 
	$b_2~=~\SI{1}{\newton\second\per\meter}$ for case (2)
\end{enumerate}
we want to design a controller which is able to position the bar (plotter head) 
from a initial point to a given final point without any residual error.
\subsection{Control Design without Load Estimator}
To meet the control requirements a possible implementation is a state feedback 
control with integrator. Obviously the full state is not measurable, hence a 
full state observer will be also implemented.

The selected control  layout is reported in Figure~\ref{figure_ctrl_coil_1}. To 
implement the control structure we have to calculate the state feedback vector 
which includes the gain corresponding to the integral block 
$\begin{bmatrix}\mathbf{K}_x&k_i\end{bmatrix}$ and a state observer. For both 
purposes we have to calculate the corresponding $\mathbf{A}$ and $\mathbf{B}$ 
matrices in discrete time domain as follows
\begin{equation}
	\mathbf{A} = \mathbf{I} + \tilde{\mathbf{A}} \ t_s
\end{equation}
\begin{equation}
	\mathbf{B} = \tilde{\mathbf{B}} \ t_s
\end{equation}
\begin{figure}[H]
	\centering
	\includegraphics[width = 460pt, angle = 0, 
	keepaspectratio]{figures/servo_linear_actuator_1.eps}
	\captionsetup{width=0.5\textwidth, font=small}		
	\caption{Control architecture.}
	\label{figure_ctrl_coil_1}
\end{figure}
Once the matrices $\mathbf{A}$ and $\mathbf{B}$ are obtained we can proceed 
implementing the extended system to calculate 
$\begin{bmatrix}K_I&\mathbf{K}_x\end{bmatrix}$ as follows
\begin{equation*}
	\begin{aligned}
		\mathbf{A'} &= 
		\begin{bmatrix}
			\mathbf{A} & \mathbf{B} \\[6pt]  
			\mathbf{0} & 0
		\end{bmatrix} = \text{$(n+1)\times(n+1)$ matrix} \\[6pt]
		\mathbf{B'} &= 
		\begin{bmatrix}
			\mathbf{0} \\[6pt]  
			1
		\end{bmatrix}= \text{$(n+1)$ column vector }
	\end{aligned}
\end{equation*}
Obtained $\mathbf{A'}$ and $\mathbf{B'}$ we can proceed calculating the state 
feedback vector $\mathbf{K'}$ from the Ackermann's formula
\begin{equation*}
	\begin{aligned}
		\mathbf{K'} &= \begin{bmatrix} 0&0&...&0&1 \end{bmatrix} 
		\mathbf{M'}^{-1}q(\mathbf{A'})
	\end{aligned}
\end{equation*}
where $\mathbf{M'} = \begin{bmatrix} 
	\mathbf{B'}&\mathbf{A'}\mathbf{B'}&...&\left( \mathbf{A'}\right) 
	^{n+m-1}\mathbf{B'} \end{bmatrix}$. Ackermann's formula uses the 
\textbf{Cayley–Hamilton} theorem and $q(z)$ is the desired final polynomial 
characteristics.

The \textbf{Cayley-Hamilton} theorem states that $\mathbf{A'}$ satisfies its 
own characteristic equation, hence
\begin{equation} \label{eq:13}
	q(\mathbf{A'}) = 
	\mathbf{A'}^{\,n}+\alpha_1\mathbf{A'}^{\,n-1}+...+\alpha_{n-1}\mathbf{A'}+\alpha_n\mathbf{I}
	= \mathbf{0}
\end{equation}
Once obtained the matrix $\mathbf{K'}$ we can use the formula
\begin{equation}
	\begin{aligned}
		\begin{bmatrix} 
			\mathbf{K}_x & k_i
		\end{bmatrix} = \left\lbrace \mathbf{K'}+
		\begin{bmatrix} 
			0&0&0&1
		\end{bmatrix}\right\rbrace 
		\begin{bmatrix} 
			\mathbf{A}-\mathbf{I}_4 & \mathbf{B}  \\[6pt] 
			\mathbf{C}\mathbf{A} & \mathbf{C}\mathbf{B} 
		\end{bmatrix}^{-1}
	\end{aligned}
\end{equation}
to obtain the state feedback gains 
$\begin{bmatrix}\mathbf{K}_x&k_i\end{bmatrix}$.

Regarding the state observer we have just to calculate the matrix $\mathbf{L}$. 
In this case the system is SIMO because the output vector $\vec{y}$ and 
obviously $\hat{\vec{y}}$ have dimension two and we have to use the Matlab 
command \texttt{place()}. The observer block becomes
\begin{equation}
	\begin{aligned}
		\hat{\vec{x}}(k+1)  &= {\mathbf{A}} \,\hat{\vec{x}}(k) +{\mathbf{B}} 
		\,u(k)+{\mathbf{L}} \left( \vec{y}(k) - \hat{\vec{y}}(k) \right)  
		\\[6pt]
		\hat{\vec{y}}(k)  &= \mathbf{C} \,\hat{\vec{x}}(k)
	\end{aligned}
\end{equation}
as reported in Figure~\ref{figure_ctrl_coil_1}
\subsubsection{Simulation results}
\begin{figure}[H]
	\centering
	\begin{subfigure}{.5\textwidth}
		\centering
		\includegraphics[width = 225pt, angle = 0, 
		keepaspectratio]{figures/track_x_1.eps}
		\captionsetup{width=0.75\textwidth}
		\caption{Tracking performance for a given position set-point. The 
			position reference $x^{ref}$ is in dashed. Case (1).}
		\label{figure_ctrl_coil_sim_results_1}
	\end{subfigure}%
	\begin{subfigure}{.5\textwidth}
		\centering
		\includegraphics[width = 250pt, angle = 0, 
		keepaspectratio]{figures/observer_1.eps}
		\captionsetup{width=0.75\textwidth}
		\caption{Observer performance. Case (1).}
		\label{figure_ctrl_coil_sim_results_2}
	\end{subfigure}
	\caption{Simulation results.}
	\label{}
\end{figure}
\textbf{Limits of the model} - 
The observer we have implemented presents some limits in particular in case (2) 
where unmodeled friction becomes relevant the estimation of the speed ($v(t)$) 
accumulate an error as function of the load as reported in 
Figure~\ref{figure_ctrl_coil_sim_results_4} where also the tracking performance 
are affected as shown in Figure~\ref{figure_ctrl_coil_sim_results_3}
\begin{figure}[H]
	\centering
	\begin{subfigure}{.5\textwidth}
		\centering
		\includegraphics[width = 225pt, angle = 0, 
		keepaspectratio]{figures/track_x_2.eps}
		\captionsetup{width=0.75\textwidth}
		\caption{Tracking performance for a given position set-point. The 
			position reference $x^{ref}$ is in dashed. Case (2).}
		\label{figure_ctrl_coil_sim_results_3}
	\end{subfigure}%
	\begin{subfigure}{.5\textwidth}
		\centering
		\includegraphics[width = 250pt, angle = 0, 
		keepaspectratio]{figures/observer_2.eps}
		\captionsetup{width=0.75\textwidth}
		\caption{Observer performance in case of high level of load force. Note 
			the residual error in the estimation of the load force. Case (2).}
		\label{figure_ctrl_coil_sim_results_4}
	\end{subfigure}
	\caption{Simulation results.}
	\label{}
\end{figure}

To avoid this degradation if the performance we would like to extend our state 
observer including an approximation of the dynamic of the load.

\subsection{Control Design with Load Estimator}
In this section we want to extend the observer including the dynamic of the 
load and try to increase the robustness against friction degradation of the 
rails. Let's start with a possible load model.

We can suppose a load of the form (the nature of the load force is friction, 
which means, a force which depends by speed):
\begin{equation*}
	\begin{aligned}
		{f}_l(t)  = {v}(t)
	\end{aligned}
\end{equation*}
where its dynamic can be expressed as
\begin{equation*}
	\begin{aligned}
		\dot{f}_l(t)  = \dot{v}(t)
	\end{aligned}
\end{equation*}
From Eq.~\ref{linact_9} we can write 
\begin{equation*}
	\begin{aligned}
		\dot{f}_l(t)  = \dot{v}(t) = \frac{B^Mh}{m} i(t)-\frac{b}{m} 
		v(t)-\frac{1}{m}f_l(t) 
	\end{aligned}
\end{equation*}
\begin{figure}[H]
	\centering
	\includegraphics[width = 460pt, angle = 0, 
	keepaspectratio]{figures/servo_linear_actuator_2.eps}
	\captionsetup{width=0.5\textwidth, font=small}		
	\caption{Control architecture with load estimator.}
	\label{ctrl_coil_2}
\end{figure}

The extended (open loop) observer model can be described as follows
\begin{equation}\label{extended_1}
	\begin{bmatrix}
		\frac{d\hat{x}(t)}{dt} \\[6pt]
		\frac{d\hat{v}(t)}{dt} \\[6pt]
		\frac{d\hat{i}(t)}{dt} \\[6pt]
		\frac{d\hat{f}_l(t)}{dt} 
	\end{bmatrix} = 
	\begin{bmatrix}
		0 & 1 & 0 & 0\\[6pt]
		0 & -\frac{b}{m} & \frac{B^{M}h}{m} & -\frac{1}{m} \\[6pt]
		0 & -\frac{B^{M}h}{L} & -\frac{R}{L} & 0 \\[6pt]
		0 & -\frac{b}{m} & \frac{B^{M}h}{m} & -\frac{1}{m}
	\end{bmatrix}
	\begin{bmatrix}
		{\hat{x}(t)} \\[6pt]
		{\hat{v}(t)} \\[6pt]
		{\hat{i}(t)} \\[6pt]
		\hat{f}_l(t) 
	\end{bmatrix} + 
	\begin{bmatrix}
		0 \\[6pt]
		0 \\[6pt]
		\frac{1}{L} \\[6pt]
		0
	\end{bmatrix} u(t)
\end{equation}
where the closed loop form is
\begin{equation}
	\begin{aligned}
		\hat{\vec{x}}_{le}(t)  &= \tilde{\mathbf{A}}_{le} 
		\,\hat{\vec{x}}_{le}(t) +\tilde{\mathbf{B}}_{le} 
		\,\tau_1(t)+\tilde{\mathbf{L}}_{le} \left( \vec{y}(t) - 
		\hat{\vec{y}}(t) \right)  \\[6pt]
		\hat{\vec{y}}(t)  &= \mathbf{C}_{le} \,\hat{\vec{x}}_{le}(t)
	\end{aligned}
\end{equation}
where $\vec{x}_{le} = \begin{bmatrix} x(t) & v(t) & i(t) & f_l(t) 
\end{bmatrix}^T$, $\mathbf{C}_{le} = \begin{bmatrix} 1 & 0 & 0 & 0 \\ 0 & 0 & 1 
	& 0 \end{bmatrix}$
\begin{equation}
	\tilde{\mathbf{A}}_{le} = 
	\begin{bmatrix}
		0 & 1 & 0 & 0\\[6pt]
		0 & -\frac{b}{m} & \frac{B^{M}h}{m} & -\frac{1}{m} \\[6pt]
		0 & -\frac{B^{M}h}{L} & -\frac{R}{L} & 0 \\[6pt]
		0 & -\frac{b}{m} & \frac{B^{M}h}{m} & -\frac{1}{m}
	\end{bmatrix}
\end{equation}
The system $(\mathbf{A}_{le},\mathbf{C}_{le})$ is fully observable.
After the discretization 
\begin{equation}
	\begin{aligned}
		\tilde{\mathbf{A}}_{le} &\rightarrow {\mathbf{A}}_{le} \\[6pt]
		\tilde{\mathbf{B}}_{le} &\rightarrow {\mathbf{B}}_{le}
	\end{aligned}
\end{equation}
\begin{equation}
	\mathbf{A}_{le} = \mathbf{I} + \tilde{\mathbf{A}}_{le} \ t_s
\end{equation}
or
\begin{equation}
	\mathbf{B}_{le} = \tilde{\mathbf{B}}_{le} \ t_s
\end{equation}
we obtain the following state observer system
\begin{equation}
	\begin{aligned}
		\hat{\vec{x}}_{le}(k+1)  &= {\mathbf{A}}_{le} \,\hat{\vec{x}}_{le}(k) 
		+{\mathbf{B}}_{le} \,\tau_1(t)+{\mathbf{L}}_{le} \left( \vec{y}(k) - 
		\hat{\vec{y}}(k) \right)  \\[6pt]
		\hat{\vec{y}}(k)  &= \mathbf{C}_{le} \,\hat{\vec{x}}_{le}(k)
	\end{aligned}
\end{equation}
Where matrix $\mathbf{L}_{le}$ is calculated applying \texttt{place()} formula 
considering the system $(\mathbf{A}_{le},\mathbf{C}_{le})$.
\subsubsection{Simulation Results}
\begin{figure}[H]
	\centering
	\begin{subfigure}{.5\textwidth}
		\centering
		\includegraphics[width = 245pt, angle = 0, 
		keepaspectratio]{figures/track_x_3.eps}
		\captionsetup{width=0.75\textwidth}
		\caption{Tracking performance for a given position set-point. The 
			position reference $x^{ref}$ is in dashed. Case (2).}
		\label{figure_moving_coil_2}
	\end{subfigure}%
	\begin{subfigure}{.5\textwidth}
		\centering
		\includegraphics[width = 250pt, angle = 0, 
		keepaspectratio]{figures/observer_3.eps}
		\captionsetup{width=0.75\textwidth}
		\caption{Observer performance in case of high level of load force. Note 
			the residual error in the estimation of the load force. Case (2).}
		\label{figure_moving_coil_3}
	\end{subfigure}
	\caption{Simulation results.}
	\label{}
\end{figure}
\textbf{Summary} - In this example we have seen an application of the state 
feedback with integral control. This kind of approach is useful when it is 
necessary to achieve a zero error steady state tracking from a constant 
reference and the plant doesn't contain an integrator. In additional, we have 
implemented a state observer, at first glance, without including the dynamic of 
the load, which is here considered as a disturbance or plant parameter 
deviation (like ageing in the rails viscosity). We have seen that this plant 
deviation can degrade the performance of the servo and increasing the time to 
set, because the load enter in the system like a disturbance and the controller 
has to compensate it, spending \textquotedblleft time\textquotedblright  and 
\textquotedblleft energy\textquotedblright  which results in a modification of 
the global dynamics.

In the second part we tried to modelize the load dynamics supposing it comes from 
a not modelized viscosity of the rails. Increasing the observer (in terms of states elements), namely, including the 
dynamics of the load, we were able to estimate the load and to reduce, at least 
partially, the degradation of the tracking performance. 

	\newpage
	\begin{thebibliography}{9}
	\bibitem{furlani} 
	Edward P. Furlani, \emph{Permanent magnet and electromagnetical devices}. Academic Press 
	2001.
	\bibitem{woodson1} 
	H.H. Woodson, J.R. Melcher, \emph{Electromechanical Dynamics, Part I: Discrete Systems}. John Wiley 1968.
	\bibitem{ogata_1} 
	K. Ogata, \emph{Modern Control Engineering}. Pearson 2009.	
	\bibitem{ogata_2} 
	K. Ogata, \emph{Discrete-time Control Systems}. Prentice-Hall 1987.
	\end{thebibliography}
	
	%*********************************************************
	% Print bibliography (If bibliography is empty nothing happens)
	%*********************************************************
	%\newpage
	%\printbibliography[heading=bibintoc]
\end{onehalfspace}
\end{document} 